\documentclass[8pt,aspectratio=169]{beamer}
\usetheme{Madrid}
\usepackage{graphicx}
\usepackage{booktabs}
\usepackage{adjustbox}
\usepackage{multicol}
\usepackage{amsmath}
\usepackage{amssymb}

\definecolor{mlblue}{RGB}{0,102,204}
\definecolor{mlpurple}{RGB}{51,51,178}
\definecolor{mllavender}{RGB}{173,173,224}
\definecolor{mllavender2}{RGB}{193,193,232}
\definecolor{mllavender3}{RGB}{204,204,235}
\definecolor{mllavender4}{RGB}{214,214,239}
\definecolor{mlorange}{RGB}{255,127,14}
\definecolor{mlgreen}{RGB}{44,160,44}
\definecolor{mlred}{RGB}{214,39,40}
\definecolor{mlgray}{RGB}{127,127,127}
\definecolor{lightgray}{RGB}{240,240,240}
\definecolor{midgray}{RGB}{180,180,180}

\setbeamercolor{palette primary}{bg=mllavender3,fg=mlpurple}
\setbeamercolor{palette secondary}{bg=mllavender2,fg=mlpurple}
\setbeamercolor{palette tertiary}{bg=mllavender,fg=white}
\setbeamercolor{palette quaternary}{bg=mlpurple,fg=white}
\setbeamercolor{structure}{fg=mlpurple}
\setbeamercolor{title}{fg=mlpurple}
\setbeamercolor{frametitle}{fg=mlpurple,bg=mllavender3}
\setbeamercolor{block title}{bg=mllavender2,fg=mlpurple}
\setbeamercolor{block body}{bg=mllavender4,fg=black}

\setbeamertemplate{navigation symbols}{}
\setbeamertemplate{itemize items}[circle]
\setbeamertemplate{enumerate items}[default]
\setbeamersize{text margin left=5mm,text margin right=5mm}

\newcommand{\bottomnote}[1]{\vfill\vspace{-2mm}\textcolor{mllavender2}{\rule{\textwidth}{0.4pt}}\vspace{1mm}\footnotesize\textbf{#1}}
\newcommand{\compactlist}{\setlength{\itemsep}{0pt}\setlength{\parskip}{0pt}\setlength{\parsep}{0pt}}
\newcommand{\chartplaceholder}[2][5cm]{\begin{center}\begin{adjustbox}{max width=0.95\textwidth, max height=#1}\framebox[\textwidth][c]{\rule{0pt}{#1}\textcolor{midgray}{[#2]}}\end{adjustbox}\end{center}}

% Shared notation macros for AI-Based Detection of Hedge Fund Fraud
% Include this file in all Beamer slide decks via % Shared notation macros for AI-Based Detection of Hedge Fund Fraud
% Include this file in all Beamer slide decks via % Shared notation macros for AI-Based Detection of Hedge Fund Fraud
% Include this file in all Beamer slide decks via \input{notation}

% Performance metrics
\newcommand{\auc}{\ensuremath{\mathrm{AUC}}}
\newcommand{\fone}{\ensuremath{F_1}}
\newcommand{\shap}{\text{SHAP}}
\newcommand{\lime}{\text{LIME}}

% Autocorrelation
\newcommand{\rhoone}{\ensuremath{\rho_1}}

% Financial abbreviations
\newcommand{\nav}{\text{NAV}}
\newcommand{\aum}{\text{AUM}}

% Regulatory
\newcommand{\euaiact}{EU AI Act}
\newcommand{\aifmd}{AIFMD}
\newcommand{\sec}{SEC}
\newcommand{\edgar}{EDGAR}

% Key numbers from the paper (for consistency)
\newcommand{\aumdollar}{\$4.5\text{ trillion}}
\newcommand{\aucdeg}{10.6\%}
\newcommand{\numop}{10}
\newcommand{\numfraudcases}{50\text{--}100}
\newcommand{\systematicpapers}{105}


% Performance metrics
\newcommand{\auc}{\ensuremath{\mathrm{AUC}}}
\newcommand{\fone}{\ensuremath{F_1}}
\newcommand{\shap}{\text{SHAP}}
\newcommand{\lime}{\text{LIME}}

% Autocorrelation
\newcommand{\rhoone}{\ensuremath{\rho_1}}

% Financial abbreviations
\newcommand{\nav}{\text{NAV}}
\newcommand{\aum}{\text{AUM}}

% Regulatory
\newcommand{\euaiact}{EU AI Act}
\newcommand{\aifmd}{AIFMD}
\newcommand{\sec}{SEC}
\newcommand{\edgar}{EDGAR}

% Key numbers from the paper (for consistency)
\newcommand{\aumdollar}{\$4.5\text{ trillion}}
\newcommand{\aucdeg}{10.6\%}
\newcommand{\numop}{10}
\newcommand{\numfraudcases}{50\text{--}100}
\newcommand{\systematicpapers}{105}


% Performance metrics
\newcommand{\auc}{\ensuremath{\mathrm{AUC}}}
\newcommand{\fone}{\ensuremath{F_1}}
\newcommand{\shap}{\text{SHAP}}
\newcommand{\lime}{\text{LIME}}

% Autocorrelation
\newcommand{\rhoone}{\ensuremath{\rho_1}}

% Financial abbreviations
\newcommand{\nav}{\text{NAV}}
\newcommand{\aum}{\text{AUM}}

% Regulatory
\newcommand{\euaiact}{EU AI Act}
\newcommand{\aifmd}{AIFMD}
\newcommand{\sec}{SEC}
\newcommand{\edgar}{EDGAR}

% Key numbers from the paper (for consistency)
\newcommand{\aumdollar}{\$4.5\text{ trillion}}
\newcommand{\aucdeg}{10.6\%}
\newcommand{\numop}{10}
\newcommand{\numfraudcases}{50\text{--}100}
\newcommand{\systematicpapers}{105}


\title{Appendix B: Feature Engineering Details}
\subtitle{AI-Based Detection of Hedge Fund Fraud}
\author{Comprehensive Feature Specifications}
\date{}

\begin{document}

% ==================== SLIDE 1: Title ====================
\begin{frame}
\titlepage
\end{frame}

% ==================== SLIDE 2: Feature Overview ====================
\begin{frame}{Feature Engineering Overview}

\begin{block}{Critical Role}
Feature quality and informativeness directly determine model performance in hedge fund fraud detection systems.
\end{block}

\vspace{2mm}

\begin{columns}[T]

\column{0.58\textwidth}
\textbf{Five Feature Categories:}

\begin{enumerate}\compactlist
    \item \textcolor{mlpurple}{\textbf{Statistical Features:}} Return properties, risk metrics, distributional characteristics
    \vspace{1mm}

    \item \textcolor{mlpurple}{\textbf{Benford Features:}} Conformity to mathematical laws of digit distributions
    \vspace{1mm}

    \item \textcolor{mlpurple}{\textbf{Textual Features:}} Disclosure characteristics, readability, sentiment
    \vspace{1mm}

    \item \textcolor{mlpurple}{\textbf{Network Features:}} Relationship structures, centrality measures
    \vspace{1mm}

    \item \textcolor{mlpurple}{\textbf{Temporal Features:}} Time-varying patterns, regime changes, calendar effects
\end{enumerate}

\column{0.38\textwidth}
\begin{center}
\fcolorbox{mlpurple}{mllavender4}{%
\begin{minipage}{0.9\linewidth}
\centering
\textbf{Typical Feature Space} \\[2mm]
\Large \textcolor{mlpurple}{\textbf{50--200}} \\[1mm]
\normalsize features in practice \\[3mm]

\footnotesize
Dimensionality varies by:
\begin{itemize}\compactlist
    \item Data availability
    \item Model complexity
    \item Regulatory context
    \item Interpretability needs
\end{itemize}
\end{minipage}
}
\end{center}
\end{columns}

\vspace{3mm}

\begin{center}
\textcolor{mlpurple}{\textbf{Multi-faceted approach captures different aspects of fund behavior to identify fraud patterns}}
\end{center}

\bottomnote{Feature engineering is the cornerstone of effective fraud detection systems.}
\end{frame}

% ==================== SLIDE 3: Statistical Features ====================
\begin{frame}{Statistical Features}

\begin{block}{Return-Based Characteristics}
Statistical properties of return time series reveal anomalies indicative of manipulation or fraud.
\end{block}

\vspace{1mm}

\begin{table}
\centering
\small
\begin{tabular}{@{}p{4.5cm}p{5.5cm}p{3cm}@{}}
\toprule
\textbf{Feature} & \textbf{Formula/Description} & \textbf{Fraud Signal} \\
\midrule
\textcolor{mlpurple}{\textbf{First-order autocorrelation}} & $\rho_1 = \text{Corr}(r_t, r_{t-1})$ & High $\rho_1$ indicates return smoothing \\
\addlinespace
\textcolor{mlpurple}{\textbf{Sharpe ratio}} & $SR = \bar{r}/\sigma_r$ & Abnormally high SR suggests fabricated returns \\
\addlinespace
\textcolor{mlpurple}{\textbf{Maximum drawdown}} & $MDD = \max_{t}\left(\frac{\max_{s \leq t} P_s - P_t}{\max_{s \leq t} P_s}\right)$ & Unusually low MDD indicates smoothing \\
\addlinespace
\textcolor{mlpurple}{\textbf{Kurtosis}} & Excess kurtosis: $\text{Kurt}(r) - 3$ & Extreme kurtosis signals manipulation \\
\addlinespace
\textcolor{mlpurple}{\textbf{Skewness}} & Third moment of distribution & Asymmetry patterns reveal reporting bias \\
\addlinespace
\textcolor{mlpurple}{\textbf{Discontinuity at zero}} & Kink in return distribution at zero (Bollen-Pool) & Avoidance of negative returns \\
\addlinespace
\textcolor{mlpurple}{\textbf{Hurst exponent}} & $H$ via rescaled range (R/S) analysis & Long-range dependence, non-randomness \\
\bottomrule
\end{tabular}
\end{table}

\vspace{2mm}

\textbf{Data Requirements:} Minimum 24--36 months of monthly return data for reliable estimation

\bottomnote{Getmansky et al. (2004), Brown et al. (2009), Bollen \& Pool (2009), Diaz et al. (2013).}
\end{frame}

% ==================== SLIDE 4: Benford Features ====================
\begin{frame}{Benford's Law Features}

\begin{block}{Mathematical Law Conformity}
Naturally occurring financial data follows Benford's Law; deviations suggest manipulation or fabrication.
\end{block}

\vspace{2mm}

\begin{columns}[T]

\column{0.48\textwidth}
\textbf{Benford's Law:}
\begin{itemize}\compactlist
    \item Leading digit distribution is logarithmic
    \item Digit ``1'' appears ~30\% of the time
    \item Digit ``9'' appears ~4.6\% of the time
    \item Applies to many naturally occurring datasets
\end{itemize}

\vspace{3mm}

\textbf{Expected Probability:}
\[
P(d) = \log_{10}\left(1 + \frac{1}{d}\right)
\]
for first digit $d \in \{1, 2, \ldots, 9\}$

\column{0.48\textwidth}
\textbf{Feature Specifications:}

\begin{table}
\centering
\footnotesize
\begin{tabular}{@{}p{3.5cm}p{3cm}@{}}
\toprule
\textbf{Test} & \textbf{Formula} \\
\midrule
\textcolor{mlpurple}{\textbf{First-digit test}} & $\chi^2 = \sum_{d=1}^{9} \frac{(O_d - E_d)^2}{E_d}$ \\
\addlinespace
\textcolor{mlpurple}{\textbf{Second-digit test}} & Benford's law on second digit \\
\addlinespace
\textcolor{mlpurple}{\textbf{Summation test}} & Cumulative conformity across positions \\
\bottomrule
\end{tabular}
\end{table}

\vspace{2mm}

\textbf{Applications:}
\begin{itemize}\compactlist
    \item Return series
    \item NAV values
    \item Fee disclosures
    \item Transaction amounts
\end{itemize}
\end{columns}

\vspace{2mm}

\bottomnote{Amiram et al. (2015), Jorion et al. (2015). High $\chi^2$ indicates fabricated data.}
\end{frame}

% ==================== SLIDE 5: Textual Features ====================
\begin{frame}{Textual Features}

\begin{block}{Disclosure Analysis}
Textual characteristics of regulatory filings, annual letters, and offering documents reveal deception patterns.
\end{block}

\vspace{1mm}

\begin{table}
\centering
\small
\begin{tabular}{@{}p{4cm}p{6cm}p{2.5cm}@{}}
\toprule
\textbf{Feature} & \textbf{Formula/Description} & \textbf{Fraud Signal} \\
\midrule
\textcolor{mlpurple}{\textbf{Fog index}} & $0.4[(w/s) + 100(c/w)]$ where $w$=words, $s$=sentences, $c$=complex words & High readability difficulty obscures risk \\
\addlinespace
\textcolor{mlpurple}{\textbf{FinBERT sentiment}} & BERT-based financial sentiment score $\in [-1,1]$ & Overly positive sentiment masks problems \\
\addlinespace
\textcolor{mlpurple}{\textbf{Boilerplate deviation}} & $1 - \text{CosSim}(\text{doc}, \text{template})$ & Unusual language suggests concealment \\
\addlinespace
\textcolor{mlpurple}{\textbf{Topic modeling}} & LDA or BERT-based topic distributions & Topic shifts indicate strategic framing \\
\addlinespace
\textcolor{mlpurple}{\textbf{Disclosure length}} & Word count, section lengths & Excessive length obscures key information \\
\bottomrule
\end{tabular}
\end{table}

\vspace{2mm}

\textbf{Data Sources:}
\begin{itemize}\compactlist
    \item Form ADV (U.S.), AIFMD disclosures (EU)
    \item Annual investor letters
    \item Offering memoranda and private placement documents
\end{itemize}

\vspace{1mm}

\textbf{Preprocessing:} Tokenization, stopword removal, domain-specific term handling

\bottomnote{Brown et al. (2020), Chen et al. (2023). Textual features capture linguistic deception cues.}
\end{frame}

% ==================== SLIDE 6: Network Features ====================
\begin{frame}{Network Features}

\begin{block}{Relationship Structures}
Network topology of fund-service-provider relationships reveals suspicious patterns and fraud contagion.
\end{block}

\vspace{2mm}

\begin{columns}[T]

\column{0.48\textwidth}
\textbf{Network Structure:}
\begin{itemize}\compactlist
    \item \textbf{Nodes:} Funds, managers, auditors, administrators, prime brokers
    \item \textbf{Edges:} Service relationships, ownership ties, board connections
    \item \textbf{Graph type:} Undirected or directed depending on relationship
\end{itemize}

\vspace{3mm}

\textbf{Feature Specifications:}

\begin{table}
\centering
\footnotesize
\begin{tabular}{@{}p{4cm}p{2.5cm}@{}}
\toprule
\textbf{Feature} & \textbf{Formula} \\
\midrule
\textcolor{mlpurple}{\textbf{Degree centrality}} & $\sum_j A_{ij}$ \\
\addlinespace
\textcolor{mlpurple}{\textbf{Betweenness centrality}} & $\sum_{s \neq v \neq t} \frac{\sigma_{st}(v)}{\sigma_{st}}$ \\
\addlinespace
\textcolor{mlpurple}{\textbf{Related-party count}} & Number of disclosed relationships \\
\bottomrule
\end{tabular}
\end{table}

\column{0.48\textwidth}
\textbf{Fraud Signals:}
\begin{itemize}\compactlist
    \item \textcolor{mlred}{\textbf{Low degree:}} Isolated funds with few relationships
    \item \textcolor{mlred}{\textbf{High clustering:}} Tight-knit groups suggest collusion
    \item \textcolor{mlred}{\textbf{Self-custody:}} Fund controls its own assets
    \item \textcolor{mlred}{\textbf{Related-party auditors:}} Conflicts of interest
    \item \textcolor{mlred}{\textbf{Offshore domiciles:}} Opaque jurisdictions
\end{itemize}

\vspace{2mm}

\textbf{Data Sources:}
\begin{itemize}\compactlist
    \item Regulatory databases
    \item Commercial data vendors
    \item Manual extraction from disclosures
\end{itemize}

\vspace{1mm}

\textbf{Challenge:} Incomplete data due to selective disclosure

\end{columns}

\bottomnote{Zhang et al. (2022), Cassar et al. (2015). Network features capture relational fraud patterns.}
\end{frame}

% ==================== SLIDE 7: Temporal Features ====================
\begin{frame}{Temporal Features}

\begin{block}{Time-Varying Patterns}
Temporal dynamics, regime changes, and calendar effects reveal behavioral anomalies over time.
\end{block}

\vspace{1mm}

\begin{table}
\centering
\small
\begin{tabular}{@{}p{4cm}p{6cm}p{2.5cm}@{}}
\toprule
\textbf{Feature} & \textbf{Description} & \textbf{Fraud Signal} \\
\midrule
\textcolor{mlpurple}{\textbf{HMM regime indicator}} & Binary state from Hidden Markov Model & Persistent ``good'' regime unrealistic \\
\addlinespace
\textcolor{mlpurple}{\textbf{Change-point score}} & Bayesian change-point detection probability & Abrupt performance shifts suspicious \\
\addlinespace
\textcolor{mlpurple}{\textbf{Calendar effects}} & Month-end, quarter-end, year-end patterns & Strategic timing of reported returns \\
\addlinespace
\textcolor{mlpurple}{\textbf{Volatility clustering}} & GARCH-based conditional volatility & Abnormal clustering patterns \\
\addlinespace
\textcolor{mlpurple}{\textbf{Trend strength}} & Moving average deviation & Unrealistic constant trends \\
\addlinespace
\textcolor{mlpurple}{\textbf{Periodicity}} & Fourier-based cycle detection & Artificial periodic patterns \\
\bottomrule
\end{tabular}
\end{table}

\vspace{2mm}

\begin{columns}[T]
\column{0.48\textwidth}
\textbf{Rolling Window Computation:}
\begin{itemize}\compactlist
    \item Enable real-time detection
    \item Maintain statistical power
    \item Typical window: 24--60 months
\end{itemize}

\column{0.48\textwidth}
\textbf{Temporal Modeling:}
\begin{itemize}\compactlist
    \item LSTMs capture sequential dependencies
    \item HMMs identify regime switches
    \item Bayesian methods detect change-points
\end{itemize}
\end{columns}

\bottomnote{Bollen \& Pool (2014), Adams \& MacKay (2007). Temporal features reveal dynamic fraud patterns.}
\end{frame}

% ==================== SLIDE 8: Feature Table Visualization ====================
\begin{frame}{Comprehensive Feature Summary}

\chartplaceholder[10cm]{Table showing all feature categories (Statistical, Benford, Textual, Network, Temporal) with 3-5 example features per category, their formulas, typical values for legitimate vs. fraudulent funds, and data source requirements. Use color coding: mlgreen for low-fraud-risk values, mlred for high-fraud-risk values.}

\bottomnote{Comprehensive feature engineering combines multiple information sources for robust fraud detection.}
\end{frame}

% ==================== SLIDE 9: Feature Selection Methods ====================
\begin{frame}{Feature Selection and Dimensionality Reduction}

\begin{block}{Managing High-Dimensional Feature Spaces}
With 50--200 features, selection techniques are essential for model performance, interpretability, and regulatory acceptance.
\end{block}

\vspace{2mm}

\begin{columns}[T]

\column{0.48\textwidth}
\textbf{Feature Selection Methods:}

\begin{enumerate}\compactlist
    \item \textcolor{mlpurple}{\textbf{LASSO Regularization}}
    \begin{itemize}\compactlist
        \item L1 penalty: $\lambda \sum_j |\beta_j|$
        \item Automatic feature selection
        \item Interpretable linear models
    \end{itemize}
    \vspace{1mm}

    \item \textcolor{mlpurple}{\textbf{Tree-Based Importance}}
    \begin{itemize}\compactlist
        \item Random Forest, XGBoost
        \item Gini importance, permutation importance
        \item Non-linear feature interactions
    \end{itemize}
    \vspace{1mm}

    \item \textcolor{mlpurple}{\textbf{Mutual Information}}
    \begin{itemize}\compactlist
        \item Information-theoretic criterion
        \item Captures non-linear dependencies
        \item Suitable for neural networks
    \end{itemize}
\end{enumerate}

\column{0.48\textwidth}
\textbf{Dimensionality Reduction:}

\begin{itemize}\compactlist
    \item \textcolor{mlgray}{\textbf{PCA:}} Less common due to loss of interpretability
    \item \textcolor{mlgray}{\textbf{Autoencoders:}} Rare in regulatory applications
    \item \textcolor{mlgreen}{\textbf{Preference:}} Feature selection over reduction
\end{itemize}

\vspace{3mm}

\textbf{Why Avoid Dimensionality Reduction?}

\begin{enumerate}\compactlist
    \item \textbf{Interpretability:} Regulators require feature-level explanations
    \item \textbf{Auditability:} Individual features must be traceable
    \item \textbf{Legal requirements:} Explainability mandates (EU AI Act)
    \item \textbf{Domain knowledge:} Selected features align with fraud theory
\end{enumerate}

\end{columns}

\vspace{2mm}

\begin{center}
\fcolorbox{mlpurple}{mllavender3}{%
\begin{minipage}{0.85\textwidth}
\centering
\textbf{Best Practice:} Combine domain expertise with data-driven selection for interpretable, high-performance models
\end{minipage}
}
\end{center}

\bottomnote{Feature selection maintains interpretability critical for regulatory applications.}
\end{frame}

% ==================== SLIDE 10: Data Source Requirements ====================
\begin{frame}{Data Source Requirements by Category}

\begin{block}{Data Availability and Quality}
Feature construction depends on access to high-quality, comprehensive data from multiple sources.
\end{block}

\vspace{1mm}

\begin{table}
\centering
\small
\begin{tabular}{@{}p{3cm}p{5cm}p{4.5cm}@{}}
\toprule
\textbf{Category} & \textbf{Data Sources} & \textbf{Quality Challenges} \\
\midrule
\textcolor{mlpurple}{\textbf{Statistical}} &
Monthly return series (24--36 months minimum) &
\textcolor{mlred}{Backfill bias}, \textcolor{mlred}{survivorship bias}, missing data \\
\addlinespace
\textcolor{mlpurple}{\textbf{Benford}} &
Returns, NAV values, fee disclosures &
Rounding effects, small sample sizes \\
\addlinespace
\textcolor{mlpurple}{\textbf{Textual}} &
Form ADV, AIFMD disclosures, annual letters, offering documents &
Unstructured format, OCR errors, \textcolor{mlred}{incomplete filings} \\
\addlinespace
\textcolor{mlpurple}{\textbf{Network}} &
Regulatory databases, commercial vendors, manual extraction &
\textcolor{mlred}{Selective disclosure}, evolving relationships, incomplete coverage \\
\addlinespace
\textcolor{mlpurple}{\textbf{Temporal}} &
Rolling return windows, market data &
Requires long histories, computational complexity \\
\bottomrule
\end{tabular}
\end{table}

\vspace{2mm}

\begin{columns}[T]
\column{0.48\textwidth}
\textbf{Commercial Databases:}
\begin{itemize}\compactlist
    \item Hedge Fund Research (HFR)
    \item Morningstar
    \item Preqin
    \item Bloomberg
\end{itemize}

\column{0.48\textwidth}
\textbf{Regulatory Sources:}
\begin{itemize}\compactlist
    \item SEC EDGAR (Form ADV)
    \item EU AIFMD registers
    \item CFTC Form PF
    \item National regulators
\end{itemize}
\end{columns}

\bottomnote{Robust fraud detection requires multi-source data integration and quality control.}
\end{frame}

\end{document}
