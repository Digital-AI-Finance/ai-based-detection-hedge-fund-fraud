\documentclass[8pt,aspectratio=169]{beamer}
\usetheme{Madrid}
\usepackage{graphicx}
\usepackage{booktabs}
\usepackage{adjustbox}
\usepackage{multicol}
\usepackage{amsmath}
\usepackage{amssymb}

% Color definitions
\definecolor{mlblue}{RGB}{0,102,204}
\definecolor{mlpurple}{RGB}{51,51,178}
\definecolor{mllavender}{RGB}{173,173,224}
\definecolor{mllavender2}{RGB}{193,193,232}
\definecolor{mllavender3}{RGB}{204,204,235}
\definecolor{mllavender4}{RGB}{214,214,239}
\definecolor{mlorange}{RGB}{255, 127, 14}
\definecolor{mlgreen}{RGB}{44, 160, 44}
\definecolor{mlred}{RGB}{214, 39, 40}
\definecolor{mlgray}{RGB}{127, 127, 127}

% Additional colors
\definecolor{lightgray}{RGB}{240, 240, 240}
\definecolor{midgray}{RGB}{180, 180, 180}

% Apply custom colors to Madrid theme
\setbeamercolor{palette primary}{bg=mllavender3,fg=mlpurple}
\setbeamercolor{palette secondary}{bg=mllavender2,fg=mlpurple}
\setbeamercolor{palette tertiary}{bg=mllavender,fg=white}
\setbeamercolor{palette quaternary}{bg=mlpurple,fg=white}

\setbeamercolor{structure}{fg=mlpurple}
\setbeamercolor{section in toc}{fg=mlpurple}
\setbeamercolor{subsection in toc}{fg=mlblue}
\setbeamercolor{title}{fg=mlpurple}
\setbeamercolor{frametitle}{fg=mlpurple,bg=mllavender3}
\setbeamercolor{block title}{bg=mllavender2,fg=mlpurple}
\setbeamercolor{block body}{bg=mllavender4,fg=black}

% Remove navigation symbols
\setbeamertemplate{navigation symbols}{}

% Clean itemize/enumerate
\setbeamertemplate{itemize items}[circle]
\setbeamertemplate{enumerate items}[default]

% Reduce margins for more content space
\setbeamersize{text margin left=5mm,text margin right=5mm}

% Command for bottom annotation
\newcommand{\bottomnote}[1]{%
\vfill
\vspace{-2mm}
\textcolor{mllavender2}{\rule{\textwidth}{0.4pt}}
\vspace{1mm}
\footnotesize
\textbf{#1}
}

% Command for compact list spacing
\newcommand{\compactlist}{%
\setlength{\itemsep}{0pt}%
\setlength{\parskip}{0pt}%
\setlength{\parsep}{0pt}%
}

% Notation macros
% Shared notation macros for AI-Based Detection of Hedge Fund Fraud
% Include this file in all Beamer slide decks via % Shared notation macros for AI-Based Detection of Hedge Fund Fraud
% Include this file in all Beamer slide decks via % Shared notation macros for AI-Based Detection of Hedge Fund Fraud
% Include this file in all Beamer slide decks via \input{notation}

% Performance metrics
\newcommand{\auc}{\ensuremath{\mathrm{AUC}}}
\newcommand{\fone}{\ensuremath{F_1}}
\newcommand{\shap}{\text{SHAP}}
\newcommand{\lime}{\text{LIME}}

% Autocorrelation
\newcommand{\rhoone}{\ensuremath{\rho_1}}

% Financial abbreviations
\newcommand{\nav}{\text{NAV}}
\newcommand{\aum}{\text{AUM}}

% Regulatory
\newcommand{\euaiact}{EU AI Act}
\newcommand{\aifmd}{AIFMD}
\newcommand{\sec}{SEC}
\newcommand{\edgar}{EDGAR}

% Key numbers from the paper (for consistency)
\newcommand{\aumdollar}{\$4.5\text{ trillion}}
\newcommand{\aucdeg}{10.6\%}
\newcommand{\numop}{10}
\newcommand{\numfraudcases}{50\text{--}100}
\newcommand{\systematicpapers}{105}


% Performance metrics
\newcommand{\auc}{\ensuremath{\mathrm{AUC}}}
\newcommand{\fone}{\ensuremath{F_1}}
\newcommand{\shap}{\text{SHAP}}
\newcommand{\lime}{\text{LIME}}

% Autocorrelation
\newcommand{\rhoone}{\ensuremath{\rho_1}}

% Financial abbreviations
\newcommand{\nav}{\text{NAV}}
\newcommand{\aum}{\text{AUM}}

% Regulatory
\newcommand{\euaiact}{EU AI Act}
\newcommand{\aifmd}{AIFMD}
\newcommand{\sec}{SEC}
\newcommand{\edgar}{EDGAR}

% Key numbers from the paper (for consistency)
\newcommand{\aumdollar}{\$4.5\text{ trillion}}
\newcommand{\aucdeg}{10.6\%}
\newcommand{\numop}{10}
\newcommand{\numfraudcases}{50\text{--}100}
\newcommand{\systematicpapers}{105}


% Performance metrics
\newcommand{\auc}{\ensuremath{\mathrm{AUC}}}
\newcommand{\fone}{\ensuremath{F_1}}
\newcommand{\shap}{\text{SHAP}}
\newcommand{\lime}{\text{LIME}}

% Autocorrelation
\newcommand{\rhoone}{\ensuremath{\rho_1}}

% Financial abbreviations
\newcommand{\nav}{\text{NAV}}
\newcommand{\aum}{\text{AUM}}

% Regulatory
\newcommand{\euaiact}{EU AI Act}
\newcommand{\aifmd}{AIFMD}
\newcommand{\sec}{SEC}
\newcommand{\edgar}{EDGAR}

% Key numbers from the paper (for consistency)
\newcommand{\aumdollar}{\$4.5\text{ trillion}}
\newcommand{\aucdeg}{10.6\%}
\newcommand{\numop}{10}
\newcommand{\numfraudcases}{50\text{--}100}
\newcommand{\systematicpapers}{105}


\title{Quiz: Adversarial Robustness and Regulatory Readiness}
\subtitle{Section 05 -- Digital-AI-Finance}
\author{Joerg Osterrieder}
\institute{Zurich University of Applied Sciences (ZHAW)}
\date{2025}

\begin{document}

\begin{frame}
\titlepage
\end{frame}

\begin{frame}{Question 1: Attack Vectors}
How many adversarial attack vectors are identified?

\vspace{0.5cm}
\begin{enumerate}[a)]
\item 2
\item 3
\item 4
\item 5
\end{enumerate}

\vspace{0.5cm}
\pause
\begin{block}{Answer}
\textbf{c) 4}

Four principal attack vectors are identified: data poisoning, evasion attacks, model extraction, and strategic timing and regime exploitation.
\end{block}
\bottomnote{Source: Section 5.1}
\end{frame}

\begin{frame}{Question 2: Mean AUC Degradation}
What is the mean AUC degradation under adversarial attack?

\vspace{0.5cm}
\begin{enumerate}[a)]
\item 5.3\%
\item 8.2\%
\item 10.6\%
\item 15.4\%
\end{enumerate}

\vspace{0.5cm}
\pause
\begin{block}{Answer}
\textbf{c) 10.6\%}

Recent work on adversarial robustness in financial ML reports a mean \auc{} degradation of 10.6\% across surveyed detection systems under adversarial attack.
\end{block}
\bottomnote{Source: Section 5.1}
\end{frame}

\begin{frame}{Question 3: Adversarial Training Recovery}
How much AUC does adversarial training recover?

\vspace{0.5cm}
\begin{enumerate}[a)]
\item 30--40\%
\item 45--55\%
\item 60--70\%
\item 80--90\%
\end{enumerate}

\vspace{0.5cm}
\pause
\begin{block}{Answer}
\textbf{c) 60--70\%}

Robust optimization applied to financial fraud detection models can recover 60--70\% of the \auc{} lost to adversarial attacks, reducing attack success rates from approximately 35\% to 5\%.
\end{block}
\bottomnote{Source: Section 5.2}
\end{frame}

\begin{frame}{Question 4: EU AI Act Transparency}
Which EU AI Act article addresses transparency?

\vspace{0.5cm}
\begin{enumerate}[a)]
\item Article 9
\item Article 13
\item Article 14
\item Article 52
\end{enumerate}

\vspace{0.5cm}
\pause
\begin{block}{Answer}
\textbf{b) Article 13}

Article 13 of the EU AI Act mandates transparency: high-risk AI systems must be designed to enable users to interpret the system's output and use it appropriately.
\end{block}
\bottomnote{Source: Section 5.3}
\end{frame}

\begin{frame}{Question 5: EU AI Act Oversight}
Which EU AI Act article addresses human oversight?

\vspace{0.5cm}
\begin{enumerate}[a)]
\item Article 9
\item Article 13
\item Article 14
\item Article 52
\end{enumerate}

\vspace{0.5cm}
\pause
\begin{block}{Answer}
\textbf{c) Article 14}

Article 14 of the EU AI Act requires human oversight: high-risk systems must allow effective oversight by natural persons, including the ability to override the system's output.
\end{block}
\bottomnote{Source: Section 5.3}
\end{frame}

\begin{frame}{Question 6: Institutional Preparedness}
What percentage of institutions lack adversarial resilience policies?

\vspace{0.5cm}
\begin{enumerate}[a)]
\item 45\%
\item 58\%
\item 68\%
\item 78\%
\end{enumerate}

\vspace{0.5cm}
\pause
\begin{block}{Answer}
\textbf{d) 78\%}

A survey of financial institutions found that 78\% lacked formal adversarial resilience policies for their ML-based detection systems, suggesting a wide gap between threat landscape and preparedness.
\end{block}
\bottomnote{Source: Section 5.1}
\end{frame}

\begin{frame}{Question 7: Data Poisoning Impact}
What is the data poisoning degradation range?

\vspace{0.5cm}
\begin{enumerate}[a)]
\item 1--3\%
\item 5--12\%
\item 15--25\%
\item 30--40\%
\end{enumerate}

\vspace{0.5cm}
\pause
\begin{block}{Answer}
\textbf{b) 5--12\%}

Data poisoning attacks can degrade model performance by 5--12\% even when the fraction of poisoned samples is small, particularly concerning given class imbalance amplifies the impact of corrupted labels.
\end{block}
\bottomnote{Source: Section 5.1}
\end{frame}

\end{document}
