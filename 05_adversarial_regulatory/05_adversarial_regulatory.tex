% ============================================================
%  Slide Deck 5 -- Adversarial Robustness, Regulatory Readiness, and Ethics
%  AI-Based Detection of Hedge Fund Fraud
% ============================================================
\documentclass[8pt,aspectratio=169]{beamer}
\usetheme{Madrid}
\usepackage{graphicx}
\usepackage{booktabs}
\usepackage{adjustbox}
\usepackage{multicol}
\usepackage{amsmath}
\usepackage{amssymb}

% ---- Color definitions ----
\definecolor{mlblue}{RGB}{0,102,204}
\definecolor{mlpurple}{RGB}{51,51,178}
\definecolor{mllavender}{RGB}{173,173,224}
\definecolor{mllavender2}{RGB}{193,193,232}
\definecolor{mllavender3}{RGB}{204,204,235}
\definecolor{mllavender4}{RGB}{214,214,239}
\definecolor{mlorange}{RGB}{255,127,14}
\definecolor{mlgreen}{RGB}{44,160,44}
\definecolor{mlred}{RGB}{214,39,40}
\definecolor{mlgray}{RGB}{127,127,127}
\definecolor{lightgray}{RGB}{240,240,240}
\definecolor{midgray}{RGB}{180,180,180}

% ---- Apply custom colors to Madrid theme ----
\setbeamercolor{palette primary}{bg=mllavender3,fg=mlpurple}
\setbeamercolor{palette secondary}{bg=mllavender2,fg=mlpurple}
\setbeamercolor{palette tertiary}{bg=mllavender,fg=white}
\setbeamercolor{palette quaternary}{bg=mlpurple,fg=white}
\setbeamercolor{structure}{fg=mlpurple}
\setbeamercolor{section in toc}{fg=mlpurple}
\setbeamercolor{subsection in toc}{fg=mlblue}
\setbeamercolor{title}{fg=mlpurple}
\setbeamercolor{frametitle}{fg=mlpurple,bg=mllavender3}
\setbeamercolor{block title}{bg=mllavender2,fg=mlpurple}
\setbeamercolor{block body}{bg=mllavender4,fg=black}

% ---- Navigation / itemize ----
\setbeamertemplate{navigation symbols}{}
\setbeamertemplate{itemize items}[circle]
\setbeamertemplate{enumerate items}[default]
\setbeamersize{text margin left=5mm,text margin right=5mm}

% ---- Custom commands ----
\newcommand{\bottomnote}[1]{%
\vfill
\vspace{-2mm}
\textcolor{mllavender2}{\rule{\textwidth}{0.4pt}}
\vspace{1mm}
\footnotesize
\textbf{#1}
}

\newcommand{\compactlist}{%
\setlength{\itemsep}{0pt}%
\setlength{\parskip}{0pt}%
\setlength{\parsep}{0pt}%
}

\newcommand{\chartplaceholder}[2][5cm]{%
\begin{center}
\begin{adjustbox}{max width=0.95\textwidth, max height=#1}
\framebox[\textwidth][c]{%
\rule{0pt}{#1}%
\textcolor{midgray}{[#2]}%
}
\end{adjustbox}
\end{center}
}

% ---- Notation ----
% Shared notation macros for AI-Based Detection of Hedge Fund Fraud
% Include this file in all Beamer slide decks via % Shared notation macros for AI-Based Detection of Hedge Fund Fraud
% Include this file in all Beamer slide decks via % Shared notation macros for AI-Based Detection of Hedge Fund Fraud
% Include this file in all Beamer slide decks via \input{notation}

% Performance metrics
\newcommand{\auc}{\ensuremath{\mathrm{AUC}}}
\newcommand{\fone}{\ensuremath{F_1}}
\newcommand{\shap}{\text{SHAP}}
\newcommand{\lime}{\text{LIME}}

% Autocorrelation
\newcommand{\rhoone}{\ensuremath{\rho_1}}

% Financial abbreviations
\newcommand{\nav}{\text{NAV}}
\newcommand{\aum}{\text{AUM}}

% Regulatory
\newcommand{\euaiact}{EU AI Act}
\newcommand{\aifmd}{AIFMD}
\newcommand{\sec}{SEC}
\newcommand{\edgar}{EDGAR}

% Key numbers from the paper (for consistency)
\newcommand{\aumdollar}{\$4.5\text{ trillion}}
\newcommand{\aucdeg}{10.6\%}
\newcommand{\numop}{10}
\newcommand{\numfraudcases}{50\text{--}100}
\newcommand{\systematicpapers}{105}


% Performance metrics
\newcommand{\auc}{\ensuremath{\mathrm{AUC}}}
\newcommand{\fone}{\ensuremath{F_1}}
\newcommand{\shap}{\text{SHAP}}
\newcommand{\lime}{\text{LIME}}

% Autocorrelation
\newcommand{\rhoone}{\ensuremath{\rho_1}}

% Financial abbreviations
\newcommand{\nav}{\text{NAV}}
\newcommand{\aum}{\text{AUM}}

% Regulatory
\newcommand{\euaiact}{EU AI Act}
\newcommand{\aifmd}{AIFMD}
\newcommand{\sec}{SEC}
\newcommand{\edgar}{EDGAR}

% Key numbers from the paper (for consistency)
\newcommand{\aumdollar}{\$4.5\text{ trillion}}
\newcommand{\aucdeg}{10.6\%}
\newcommand{\numop}{10}
\newcommand{\numfraudcases}{50\text{--}100}
\newcommand{\systematicpapers}{105}


% Performance metrics
\newcommand{\auc}{\ensuremath{\mathrm{AUC}}}
\newcommand{\fone}{\ensuremath{F_1}}
\newcommand{\shap}{\text{SHAP}}
\newcommand{\lime}{\text{LIME}}

% Autocorrelation
\newcommand{\rhoone}{\ensuremath{\rho_1}}

% Financial abbreviations
\newcommand{\nav}{\text{NAV}}
\newcommand{\aum}{\text{AUM}}

% Regulatory
\newcommand{\euaiact}{EU AI Act}
\newcommand{\aifmd}{AIFMD}
\newcommand{\sec}{SEC}
\newcommand{\edgar}{EDGAR}

% Key numbers from the paper (for consistency)
\newcommand{\aumdollar}{\$4.5\text{ trillion}}
\newcommand{\aucdeg}{10.6\%}
\newcommand{\numop}{10}
\newcommand{\numfraudcases}{50\text{--}100}
\newcommand{\systematicpapers}{105}


% ---- Title metadata ----
\title{AI-Based Detection of Hedge Fund Fraud}
\subtitle{Section 5 -- Adversarial Robustness, Regulatory Readiness, and Ethics (C2)}
\author{Joerg Osterrieder}
\institute{Zurich University of Applied Sciences (ZHAW)}
\date{2025}

% ============================================================
\begin{document}

% ----------------------------------------------------------
% SLIDE 1 -- Title
% ----------------------------------------------------------
\begin{frame}
\titlepage
\end{frame}

% ----------------------------------------------------------
% SLIDE 2 -- Outline
% ----------------------------------------------------------
\begin{frame}{Outline}
\begin{enumerate}\compactlist
\item The Adversary Profile
\item Attack Vectors (Data Poisoning, Evasion, Model Extraction, Strategic Timing)
\item Attack Vectors Chart
\item Mean \auc{} Degradation: \aucdeg{}
\item Defense: Adversarial Training
\item Defense: Ensemble Diversity
\item Defense: Input Validation and Meta-Detection
\item Defense: Certified and Randomized Methods
\item EU AI Act Requirements (Art.\ 13, 14, 9)
\item SEC Regulatory Expectations
\item The Explainability--Performance Trade-off
\item Readiness Heatmap: 5 Methods $\times$ 5 Dimensions
\item Ethics: False Positives, Selection Bias, Fairness
\item Governance Recommendations
\item Summary
\end{enumerate}
\end{frame}

% ----------------------------------------------------------
% SLIDE 3 -- The Adversary Profile
% ----------------------------------------------------------
\begin{frame}{The Adversary Profile}
\begin{columns}[T]
\column{0.55\textwidth}
\textbf{Not Script Kiddies -- PhD-Level Quants}
\begin{itemize}\compactlist
\item Adversaries are \textbf{quantitatively trained professionals} (PhDs in math, physics, CS)
\item Deep knowledge of statistical methods and the same academic literature informing detection systems
\item Access to sophisticated modeling tools, computational resources, and expert consultants
\item Strong financial incentives to remain undetected ($\Rightarrow$ billions at stake)
\end{itemize}

\column{0.42\textwidth}
\begin{block}{Key Implication}
Standard adversarial ML threat models (pixel perturbations, $\ell_p$-norm budgets) are \textbf{insufficient}. The hedge fund adversary can:
\begin{itemize}\compactlist
\item Simulate detection models
\item Engineer statistically plausible returns
\item Exploit temporal and structural gaps
\end{itemize}
\end{block}
\vspace{2mm}
\textcolor{mlred}{78\% of financial institutions lack formal adversarial resilience policies for ML-based detection.}
\end{columns}
\bottomnote{Source: Goodfellow et al.\ (2015); Biggio \& Roli (2018); paper Section 5.1}
\end{frame}

% ----------------------------------------------------------
% SLIDE 4 -- Attack Vector 1: Data Poisoning
% ----------------------------------------------------------
\begin{frame}{Attack Vector 1: Data Poisoning}
\begin{block}{Mechanism}
Fraudulent managers report \textbf{fabricated returns} to commercial databases (HFR, Lipper TASS), effectively poisoning the training data of supervised detection models.
\end{block}
\vspace{2mm}
\begin{columns}[T]
\column{0.52\textwidth}
\textbf{Why It Works}
\begin{itemize}\compactlist
\item Hedge fund reporting is \textbf{voluntary and largely unverified}
\item A manager can engineer return series that:
  \begin{itemize}\compactlist
  \item Satisfy Benford's law
  \item Exhibit appropriate serial correlation
  \item Maintain plausible distributional properties
  \end{itemize}
\item Small fraction of poisoned samples $\Rightarrow$ disproportionate effect
\end{itemize}

\column{0.45\textwidth}
\textbf{Impact}
\begin{itemize}\compactlist
\item \textcolor{mlred}{\textbf{5--12\% model performance degradation}} even with small poisoning fraction (Goldblum et al., 2023)
\item Particularly dangerous because:
  \begin{itemize}\compactlist
  \item Base rate of fraud is already low
  \item Class imbalance \textbf{amplifies} the impact of corrupted labels
  \end{itemize}
\item Undermines the foundational assumption of clean training data
\end{itemize}
\end{columns}
\bottomnote{Source: Goldblum et al.\ (2023); paper Section 5.1}
\end{frame}

% ----------------------------------------------------------
% SLIDE 5 -- Attack Vector 2: Evasion
% ----------------------------------------------------------
\begin{frame}{Attack Vector 2: Evasion Attacks}
\begin{block}{Mechanism}
Structuring reported returns to \textbf{avoid triggering detection thresholds} -- the most natural adversarial attack in the hedge fund domain.
\end{block}
\vspace{2mm}
\begin{columns}[T]
\column{0.52\textwidth}
\textbf{Forms of Evasion}
\begin{itemize}\compactlist
\item \textbf{Return smoothing} (Getmansky et al., 2004): suppresses volatility signals and serial correlation
\item Optimizing reported returns against a known or estimated \textbf{detection boundary}
\item Minor perturbations bounded by financial plausibility $\Rightarrow$ substantially alter predictions
\end{itemize}

\column{0.45\textwidth}
\textbf{Quantified Impact}
\begin{itemize}\compactlist
\item FGSM and PGD attacks degrade \auc{} by \textcolor{mlred}{\textbf{8--15\%}} (Cartella et al., 2021)
\item Mean \auc{} degradation across surveyed systems: \textcolor{mlred}{\textbf{\aucdeg{}}}
\item Even minor plausibility-bounded perturbations:
  \begin{itemize}\compactlist
  \item Elevate calibration error
  \item Increase expected portfolio loss by $\sim$5\%
  \end{itemize}
\end{itemize}
\end{columns}
\bottomnote{Source: Cartella et al.\ (2021); Goodfellow et al.\ (2015); Madry et al.\ (2018); paper Section 5.1}
\end{frame}

% ----------------------------------------------------------
% SLIDE 6 -- Attack Vector 3: Model Extraction
% ----------------------------------------------------------
\begin{frame}{Attack Vector 3: Model Extraction}
\begin{block}{Mechanism}
Reverse-engineering a regulator's detection model by \textbf{observing enforcement patterns} over time.
\end{block}
\vspace{3mm}
\begin{itemize}\compactlist
\item A sophisticated adversary can infer which features and thresholds trigger scrutiny by analyzing:
  \begin{itemize}\compactlist
  \item Which funds are investigated
  \item Which enforcement actions are brought
  \item Which anomalies are flagged in examination letters
  \end{itemize}
\item Construct a \textbf{surrogate model} of the detection system
\item Optimize reported behavior to remain below the decision boundary
\end{itemize}
\vspace{3mm}
\begin{block}{Facilitating Factor}
SEC enforcement actions and examination priority announcements are \textbf{public} -- inadvertently revealing features and thresholds that regulators prioritize.
\end{block}
\bottomnote{Source: Paper Section 5.1}
\end{frame}

% ----------------------------------------------------------
% SLIDE 7 -- Attack Vector 4: Strategic Timing
% ----------------------------------------------------------
\begin{frame}{Attack Vector 4: Strategic Timing and Regime Exploitation}
\begin{columns}[T]
\column{0.50\textwidth}
\textbf{Market Stress Exploitation}
\begin{itemize}\compactlist
\item Time fraudulent activity to coincide with \textbf{periods of market stress}
\item During crises, legitimate fund returns exhibit unusual distributional properties
\item Fabricated returns are \textbf{masked within broader market noise}
\item Detection models struggle to distinguish fraud from genuine market dislocation
\end{itemize}

\column{0.47\textwidth}
\textbf{Gradual Introduction}
\begin{itemize}\compactlist
\item Introduce fraudulent reporting \textbf{incrementally}
\item Detection models trained on historical data treat evolving fraud as \textbf{legitimate regime change}
\item Exploits the assumption of stationarity in most supervised models
\item By the time the pattern becomes detectable, years of damage may have occurred
\end{itemize}
\end{columns}
\vspace{3mm}
\textcolor{mlred}{$\Rightarrow$ Most detection models do not account for these temporal dynamics -- a critical vulnerability.}
\bottomnote{Source: Paper Section 5.1}
\end{frame}

% ----------------------------------------------------------
% SLIDE 8 -- Attack Vectors Chart
% ----------------------------------------------------------
\begin{frame}{Attack Vectors: Summary}

\chartplaceholder[5cm]{Chart: 01\_attack\_vectors -- Bar chart comparing four attack vectors (Data Poisoning, Evasion, Model Extraction, Strategic Timing) across impact severity and detection difficulty, with annotations for key metrics (5--12\% degradation, 8--15\% AUC loss)}

\vspace{2mm}
\begin{itemize}\compactlist
\item All four vectors exploit structural features of the hedge fund ecosystem
\item Evasion and data poisoning have the most \textbf{empirically quantified} impact
\item Model extraction and strategic timing are harder to measure but equally dangerous
\end{itemize}
\bottomnote{Source: Paper Section 5.1}
\end{frame}

% ----------------------------------------------------------
% SLIDE 9 -- Mean AUC Degradation
% ----------------------------------------------------------
\begin{frame}{Mean \auc{} Degradation: \aucdeg{}}
\begin{columns}[T]
\column{0.55\textwidth}
\begin{block}{Aggregate Finding}
Across surveyed detection systems, adversarial perturbations cause a \textbf{mean \auc{} degradation of \aucdeg{}}, with some techniques exhibiting degradation \textbf{exceeding 25\%} under informed adversaries.
\end{block}
\vspace{2mm}
\textbf{Breakdown}
\begin{itemize}\compactlist
\item Data poisoning: \textbf{5--12\%} degradation
\item Evasion (FGSM/PGD): \textbf{8--15\%} degradation
\item Plausibility-bounded perturbations: $\sim$\textbf{5\%} portfolio loss increase
\item Attack success rate (pre-defense): $\sim$\textbf{35\%}
\end{itemize}

\column{0.42\textwidth}
\textbf{Why This Understates the Risk}
\begin{itemize}\compactlist
\item Literature tests \textit{generic} perturbations, not financially tailored attacks
\item Real adversaries use \textbf{domain-specific constraints}
\item Actual degradation against PhD-level quants likely \textbf{higher}
\item No published study tests robustness against hedge-fund-specific adversaries
\end{itemize}
\end{columns}
\vspace{2mm}
\textcolor{mlred}{$\Rightarrow$ Detection performance on historical data is necessary but \textbf{not sufficient} for deployment.}
\bottomnote{Source: Cartella et al.\ (2021); Chen et al.\ (2024); paper Section 5.1}
\end{frame}

% ----------------------------------------------------------
% SLIDE 10 -- Defense: Adversarial Training
% ----------------------------------------------------------
\begin{frame}{Defense 1: Adversarial Training}
\begin{block}{Approach}
Augment training data with adversarial examples, forcing the model to learn decision boundaries \textbf{robust to perturbations} (min-max optimization: Madry et al., 2018).
\end{block}
\vspace{2mm}
\begin{columns}[T]
\column{0.50\textwidth}
\textbf{Effectiveness}
\begin{itemize}\compactlist
\item Recovers \textcolor{mlgreen}{\textbf{60--70\%}} of \auc{} lost to adversarial attacks (Chen et al., 2024)
\item Reduces attack success rate from $\sim$35\% to $\sim$\textbf{5\%}
\item Most direct and well-studied defense mechanism
\end{itemize}

\column{0.47\textwidth}
\textbf{Limitations}
\begin{itemize}\compactlist
\item Computationally expensive: \textcolor{mlred}{\textbf{5--10$\times$}} standard training cost
\item Which perturbation model to use?
  \begin{itemize}\compactlist
  \item $\ell_\infty$-bounded perturbations dominate CV literature
  \item May not capture \textbf{financially constrained} perturbations
  \end{itemize}
\item Need domain-specific perturbation budgets
\end{itemize}
\end{columns}
\bottomnote{Source: Madry et al.\ (2018); Chen et al.\ (2024); paper Section 5.2}
\end{frame}

% ----------------------------------------------------------
% SLIDE 11 -- Defense: Ensemble Diversity
% ----------------------------------------------------------
\begin{frame}{Defense 2: Ensemble Diversity}
\begin{block}{Implicit Adversarial Robustness}
Ensembles aggregating predictions from \textbf{multiple heterogeneous base learners} provide a natural form of adversarial robustness.
\end{block}
\vspace{3mm}
\begin{itemize}\compactlist
\item An adversary optimizing evasion against one model component is \textbf{unlikely to simultaneously fool all components}
\item Most effective when ensemble incorporates \textbf{diverse model families}:
  \begin{itemize}\compactlist
  \item Tree-based models (XGBoost, LightGBM)
  \item Neural networks (LSTM, autoencoders)
  \item Statistical anomaly detectors (isolation forests, one-class SVM)
  \end{itemize}
\item Diversity metric over ensemble components \textbf{correlates positively} with robustness (Patel et al., 2025)
\item Yields robustness benefits \textbf{beyond} the accuracy gains typically reported for ensembles
\end{itemize}
\vspace{2mm}
\textcolor{mlblue}{$\Rightarrow$ Combining gradient-boosted trees with neural network classifiers is both a performance and a robustness strategy.}
\bottomnote{Source: Patel et al.\ (2025); paper Section 5.2}
\end{frame}

% ----------------------------------------------------------
% SLIDE 12 -- Defense: Input Validation
% ----------------------------------------------------------
\begin{frame}{Defense 3: Input Validation and Meta-Level Anomaly Detection}
\begin{block}{Layered Defense}
Deploy a \textbf{separate anomaly detection system} to screen model inputs for signs of adversarial manipulation -- treating adversarial detection as a distinct task from fraud detection.
\end{block}
\vspace{3mm}
\begin{columns}[T]
\column{0.50\textwidth}
\textbf{What the Meta-Detector Flags}
\begin{itemize}\compactlist
\item Suspiciously precise conformity to Benford's law
\item Implausibly low serial correlation for the stated strategy type
\item Distributional characteristics inconsistent with reported asset class exposures
\item Statistical properties ``too clean'' to be genuine
\end{itemize}

\column{0.47\textwidth}
\textbf{Advantages}
\begin{itemize}\compactlist
\item Does not require modifying the classification model itself
\item Enables \textbf{specialized models} for each task
\item Can be updated independently as new attack patterns emerge
\item Complements rather than replaces adversarial training
\end{itemize}
\end{columns}
\bottomnote{Source: Paper Section 5.2}
\end{frame}

% ----------------------------------------------------------
% SLIDE 13 -- Defense: Certified and Randomized
% ----------------------------------------------------------
\begin{frame}{Defense 4: Certified and Randomized Defenses}
\begin{block}{Approach}
Certified defense methods (e.g., \textbf{randomized smoothing}) provide \textbf{provable robustness guarantees} within a specified perturbation radius.
\end{block}
\vspace{3mm}
\begin{columns}[T]
\column{0.50\textwidth}
\textbf{Theoretical Appeal}
\begin{itemize}\compactlist
\item Formal guarantee: bounded input change $\Rightarrow$ bounded output change
\item Eliminates need to enumerate all possible attacks
\item Provides \textbf{worst-case} robustness assurance
\end{itemize}

\column{0.47\textwidth}
\textbf{Practical Challenges}
\begin{itemize}\compactlist
\item Perturbation model must be defined over \textbf{financially meaningful dimensions} (returns, risk metrics, factor exposures)
\item Certification radius must align with plausible adversarial strategies, not arbitrary $\ell_p$ norms
\item \textcolor{mlred}{No published work} has adapted certified defenses to hedge fund fraud detection
\end{itemize}
\end{columns}
\vspace{3mm}
\textcolor{mlblue}{$\Rightarrow$ An important \textbf{open area} for future research (see OP8).}
\bottomnote{Source: Paper Section 5.2}
\end{frame}

% ----------------------------------------------------------
% SLIDE 14 -- EU AI Act
% ----------------------------------------------------------
\begin{frame}{EU AI Act: Requirements for Fraud Detection Systems}
\begin{block}{Classification}
Under Annex III of the EU AI Act (Regulation 2024/1689), AI systems for financial fraud detection are classified as \textbf{high-risk AI systems}, subject to the Act's most stringent requirements.
\end{block}
\vspace{2mm}
\begin{columns}[T]
\column{0.32\textwidth}
\begin{block}{Art.\ 13: Transparency}
\begin{itemize}\compactlist
\item Operations must be \textbf{sufficiently transparent}
\item Users must interpret output and use it appropriately
\item $\Rightarrow$ Favors interpretable models or robust post-hoc explanations
\end{itemize}
\end{block}

\column{0.32\textwidth}
\begin{block}{Art.\ 14: Human Oversight}
\begin{itemize}\compactlist
\item Effective oversight by \textbf{natural persons}
\item Ability to override output or intervene
\item $\Rightarrow$ Codifies \textbf{human-in-the-loop} principle
\end{itemize}
\end{block}

\column{0.32\textwidth}
\begin{block}{Art.\ 9: Risk Management}
\begin{itemize}\compactlist
\item Comprehensive risk management system
\item Must address \textbf{adversarial vulnerabilities}
\item $\Rightarrow$ Links adversarial robustness to \textbf{legal compliance}
\end{itemize}
\end{block}
\end{columns}
\vspace{2mm}
\textcolor{mlred}{Entered into force August 2024, phased implementation through 2027.}
\bottomnote{Source: EU AI Act (2024/1689); paper Section 5.3.1}
\end{frame}

% ----------------------------------------------------------
% SLIDE 15 -- SEC Expectations
% ----------------------------------------------------------
\begin{frame}{SEC Regulatory Expectations}
\begin{columns}[T]
\column{0.52\textwidth}
\textbf{Current Landscape}
\begin{itemize}\compactlist
\item No comprehensive AI legislation comparable to EU AI Act
\item However, SEC signals \textbf{increasing regulatory attention}:
  \begin{itemize}\compactlist
  \item Division of Examinations: AI and emerging technology as \textbf{priority area}
  \item Staff guidance: fiduciary obligations for algorithmic tools
  \end{itemize}
\item DERA itself employs ML models for market surveillance and enforcement targeting
\item Creates \textbf{implicit benchmarks} for industry practice
\end{itemize}

\column{0.45\textwidth}
\textbf{AI Washing Enforcement}
\begin{itemize}\compactlist
\item Recent enforcement actions against firms that \textbf{exaggerated or fabricated AI use} in investment processes
\item Signals SEC views \textbf{AI governance} as within enforcement purview
\item Industry should anticipate:
  \begin{itemize}\compactlist
  \item More specific expectations for AI-based compliance systems
  \item Governance requirements for ML-based surveillance
  \end{itemize}
\end{itemize}
\end{columns}
\vspace{2mm}
\textcolor{mlblue}{$\Rightarrow$ Even without legislation, SEC expectations will shape deployment standards.}
\bottomnote{Source: SEC Examination Priorities (2023); paper Section 5.3.2}
\end{frame}

% ----------------------------------------------------------
% SLIDE 16 -- Explainability--Performance Trade-off
% ----------------------------------------------------------
\begin{frame}{The Explainability--Performance Trade-off}
\begin{columns}[T]
\column{0.50\textwidth}
\textbf{The Fundamental Tension}
\begin{itemize}\compactlist
\item Most accurate methods (deep ensembles, transformers, GNNs) are the \textbf{most opaque}
\item Inherently interpretable models (logistic regression, decision trees, rules) offer transparency but \textbf{lower detection rates}
\item Regulatory requirements favor transparency
\item Detection requirements favor complexity
\end{itemize}

\column{0.47\textwidth}
\textbf{Post-Hoc Explainability Methods}
\begin{itemize}\compactlist
\item \textbf{\shap{}} (Lundberg \& Lee, 2017): theoretically grounded feature attribution scores
\item \textbf{\lime{}} (Ribeiro et al., 2016): local surrogate models
\item Can generate: ``flagged due to low serial correlation, high Sharpe vs.\ peers, irregular distributions''
\item \textcolor{mlred}{Limitations}: added cost, may not faithfully represent true decision process, can be unstable
\end{itemize}
\end{columns}
\vspace{2mm}
\begin{block}{Open Question}
No regulatory consensus on what constitutes ``sufficient'' explainability for financial fraud detection AI. Compliance requirements may tighten retroactively.
\end{block}
\bottomnote{Source: Rudin (2019); Lundberg \& Lee (2017); Ribeiro et al.\ (2016); paper Section 5.3.3}
\end{frame}

% ----------------------------------------------------------
% SLIDE 17 -- Explainability Chart
% ----------------------------------------------------------
\begin{frame}{Explainability vs.\ Performance Spectrum}

\chartplaceholder[5cm]{Chart: 03\_explainability\_tradeoff -- Scatter plot positioning method families (linear/logistic, tree ensembles, neural networks, hybrid architectures) along axes of detection performance (y) vs.\ explainability (x), with regulatory compliance threshold annotated}

\vspace{2mm}
\begin{itemize}\compactlist
\item Tree-based ensembles currently occupy the \textbf{most favorable position}: strong performance with moderate explainability via \shap{}
\item Hybrid architectures (tree classification + neural feature extraction) offer promising trade-offs
\end{itemize}
\bottomnote{Source: Paper Sections 5.3.3 and 5.4}
\end{frame}

% ----------------------------------------------------------
% SLIDE 18 -- Readiness Heatmap
% ----------------------------------------------------------
\begin{frame}{Readiness Assessment: 5 Methods $\times$ 5 Dimensions}

{\footnotesize
\begin{center}
\begin{tabular}{l c c c c c}
\toprule
\textbf{Method Family} & \textbf{Adversarial} & \textbf{Intrinsic} & \textbf{Post-Hoc} & \textbf{Regulatory} & \textbf{Deployment} \\
 & \textbf{Robustness} & \textbf{Explain.} & \textbf{Explain.} & \textbf{Readiness} & \textbf{Maturity} \\
\midrule
Linear / Logistic & \textcolor{mlred}{Low} & \textcolor{mlgreen}{High} & \textcolor{mlgreen}{High} & \textcolor{mlgreen}{High} & \textcolor{mlgreen}{High} \\
Tree-Based Ensembles & \textcolor{mlorange}{Moderate} & \textcolor{mlorange}{Moderate} & \textcolor{mlgreen}{High} & \textcolor{mlgreen}{High} & \textcolor{mlgreen}{High} \\
Deep Learning & \textcolor{mlred}{Low} & \textcolor{mlred}{Low} & \textcolor{mlorange}{Moderate} & \textcolor{mlred}{Low} & \textcolor{mlorange}{Moderate} \\
Hybrid / Ensemble & \textcolor{mlorange}{Mod--High} & \textcolor{mlorange}{Moderate} & \textcolor{mlorange}{Mod--High} & \textcolor{mlorange}{Moderate} & \textcolor{mlorange}{Moderate} \\
Anomaly Detection & \textcolor{mlorange}{Moderate} & \textcolor{mlorange}{Variable} & \textcolor{mlorange}{Variable} & \textcolor{mlorange}{Moderate} & \textcolor{mlorange}{Moderate} \\
\bottomrule
\end{tabular}
\end{center}
}

\vspace{2mm}
\begin{itemize}\compactlist
\item \textbf{Tree-based ensembles} (XGBoost, LightGBM, RF) occupy the most favorable overall position
\item \textbf{Deep learning} offers highest potential accuracy but ranks lowest on explainability and regulatory readiness
\item \textbf{Hybrid architectures} provide pragmatic path: interpretable classification + neural feature extraction
\end{itemize}
\bottomnote{Source: Paper Section 5.4}
\end{frame}

% ----------------------------------------------------------
% SLIDE 19 -- Readiness Chart
% ----------------------------------------------------------
\begin{frame}{Readiness Heatmap: Visual}

\chartplaceholder[5cm]{Chart: 02\_readiness\_heatmap -- Heatmap of 5 method families (rows) $\times$ 5 readiness dimensions (columns), color-coded from green (high readiness) through yellow (moderate) to red (low readiness)}

\vspace{2mm}
\begin{itemize}\compactlist
\item No single method family dominates across \textbf{all} criteria
\item Selection depends on organizational priorities: accuracy vs.\ compliance vs.\ robustness
\end{itemize}
\bottomnote{Source: Paper Section 5.4}
\end{frame}

% ----------------------------------------------------------
% SLIDE 20 -- Ethics: False Positives and Selection Bias
% ----------------------------------------------------------
\begin{frame}{Ethics: False Positives and Selection Bias}
\begin{columns}[T]
\column{0.50\textwidth}
\begin{block}{Harm from False Positives}
\begin{itemize}\compactlist
\item Consequences are \textbf{substantial and asymmetric}:
  \begin{itemize}\compactlist
  \item Severe reputational damage
  \item Investor redemptions
  \item Counterparty relationship termination
  \item Regulatory scrutiny cascade
  \end{itemize}
\item Can \textbf{destroy a legitimate business} even if allegation is unfounded
\item Market-level effects possible (contagion from public investigation)
\item Categorically different from false positives in spam filtering
\end{itemize}
\end{block}

\column{0.47\textwidth}
\begin{block}{Selection Bias from Enforcement Data}
\begin{itemize}\compactlist
\item Supervised models trained on past \sec{} actions learn patterns of \textbf{historically prosecuted} fraud
\item Prosecuted cases $\neq$ random sample of all fraud
\item Historical concentration on certain:
  \begin{itemize}\compactlist
  \item Fund types and strategies
  \item Geographies
  \item Visible fraud schemes
  \end{itemize}
\item Models may \textbf{perpetuate and amplify} selective scrutiny
\item Under-detect in overlooked categories
\end{itemize}
\end{block}
\end{columns}
\bottomnote{Source: Paper Section 5.5}
\end{frame}

% ----------------------------------------------------------
% SLIDE 21 -- Ethics: Fairness, Dual-Use, Transparency Paradox
% ----------------------------------------------------------
\begin{frame}{Ethics: Fairness, Dual-Use, and the Transparency Paradox}
\begin{columns}[T]
\column{0.32\textwidth}
\begin{block}{Fairness Across Fund Characteristics}
\begin{itemize}\compactlist
\item Do models disproportionately flag:
  \begin{itemize}\compactlist
  \item Small funds?
  \item Emerging market funds?
  \item Certain demographic groups?
  \end{itemize}
\item Almost \textbf{no empirical attention} in literature
\item Need fairness metrics for financial context
\end{itemize}
\end{block}

\column{0.32\textwidth}
\begin{block}{Dual-Use Nature}
\begin{itemize}\compactlist
\item Same AI techniques used by regulators can be \textbf{repurposed by fraudsters}
\item Published models provide a \textbf{roadmap for evasion}
\item Not hypothetical: hedge fund quants consume and operationalize published research
\end{itemize}
\end{block}

\column{0.32\textwidth}
\begin{block}{Transparency Paradox}
\begin{itemize}\compactlist
\item Regulatory demands for disclosure \textbf{conflict with} operational security
\item Every piece of disclosed information is exploitable by adversaries
\item Current regulations do not adequately address this tension
\end{itemize}
\end{block}
\end{columns}
\vspace{2mm}
\textcolor{mlred}{$\Rightarrow$ Balancing transparency with adversarial security requires nuanced governance frameworks.}
\bottomnote{Source: Doshi-Velez \& Kim (2017); paper Section 5.5}
\end{frame}

% ----------------------------------------------------------
% SLIDE 22 -- Governance Recommendations
% ----------------------------------------------------------
\begin{frame}{Governance Recommendations}
\begin{block}{Recommended Governance Framework}
Organizations deploying AI-based fraud detection should adopt:
\end{block}
\vspace{2mm}
\begin{enumerate}\compactlist
\item \textcolor{mlblue}{\textbf{Regular bias audits}} across fund characteristics and demographic dimensions
\item \textcolor{mlblue}{\textbf{Diverse and representative training data}} that corrects for historical enforcement biases where possible
\item \textcolor{mlblue}{\textbf{Human-in-the-loop decision-making}}: AI systems inform but do \textbf{not replace} human judgment in enforcement decisions
\item \textcolor{mlblue}{\textbf{Transparent model governance}}: documented procedures for model development, validation, monitoring, and retirement
\end{enumerate}
\vspace{3mm}
\begin{itemize}\compactlist
\item Aligns with emerging best practices in responsible AI (Arrieta et al., 2020; Molnar, 2020)
\item Aligns with EU AI Act human oversight requirements (Art.\ 14)
\item Adversarial testing should be \textbf{standard practice}, not optional
\end{itemize}
\bottomnote{Source: Arrieta et al.\ (2020); Molnar (2020); paper Section 5.5}
\end{frame}

% ----------------------------------------------------------
% SLIDE 23 -- Summary
% ----------------------------------------------------------
\begin{frame}{Summary: Adversarial Robustness, Regulatory Readiness, and Ethics}
\begin{enumerate}\compactlist
\item \textbf{Adversary profile}: PhD-level quants with deep statistical knowledge -- not generic attackers
\item \textbf{Four attack vectors}: data poisoning (5--12\%), evasion (8--15\% \auc{}), model extraction, strategic timing
\item \textbf{Mean \auc{} degradation}: \aucdeg{} across surveyed systems -- likely understates real-world risk
\item \textbf{Defenses}: adversarial training (recovers 60--70\%), ensemble diversity, input validation, certified methods (open area)
\item \textbf{EU AI Act}: fraud detection classified as high-risk; mandates transparency, human oversight, risk management
\item \textbf{SEC}: no comprehensive legislation yet, but increasing attention and implicit expectations
\item \textbf{Explainability--performance trade-off}: tree-based ensembles best positioned; no regulatory consensus on ``sufficient'' explainability
\item \textbf{Readiness}: no single method dominates all 5 dimensions; hybrid architectures most pragmatic
\item \textbf{Ethics}: false positive harm is severe, selection bias is systematic, dual-use and transparency paradox require governance
\item \textbf{Governance}: bias audits, diverse training data, human-in-the-loop, transparent model governance
\end{enumerate}
\bottomnote{Source: Paper Section 5 (Contribution C2)}
\end{frame}

\end{document}
