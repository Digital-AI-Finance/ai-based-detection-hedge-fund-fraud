% ==================== APPENDIX A: SYSTEMATIC SEARCH PROTOCOL ====================
\section{Systematic Search Protocol}
\label{app:search-protocol}

This appendix documents the systematic search methodology employed to identify relevant literature on AI-based detection of hedge fund fraud. The search framework was informed by the SALSA methodology \citep{grant2009salsa}---Search, AppraisaL, Synthesis, Analysis---adapted from systematic reviews in clinical research to accommodate the interdisciplinary nature of AI/ML applications in financial fraud detection.

\subsection*{Search Framework and Databases}

The systematic search was conducted across five major academic databases to ensure comprehensive coverage of both computer science and finance literature: Scopus, Web of Science, IEEE Xplore (for AI/ML methodology), SSRN (for finance working papers), and Google Scholar (for supplementary coverage and citation tracking). The search period covered publications from 2000 to 2025, with particular emphasis on the period 2015--2025 when deep learning and modern AI techniques became prominent in financial applications.

\subsection*{Query Strings and Search Strategy}

The primary search query combined three conceptual blocks using Boolean operators:
\begin{itemize}
    \item \textbf{Financial domain:} (``hedge fund'' OR ``investment fund'' OR ``alternative investment'')
    \item \textbf{Fraud-related terms:} (``fraud'' OR ``manipulation'' OR ``anomaly'')
    \item \textbf{AI/ML methods:} (``machine learning'' OR ``artificial intelligence'' OR ``deep learning'' OR ``neural network'')
\end{itemize}

A secondary, broader query was employed to capture related research that might use different terminology: (``fund'' OR ``portfolio'') AND (``fraud detection'' OR ``anomaly detection'') AND (``classification'' OR ``prediction''). This secondary search helped identify relevant studies that focused on mutual funds or general portfolio fraud rather than exclusively hedge funds.

\subsection*{Inclusion and Exclusion Criteria}

Studies were included if they met the following criteria: (1) directly addressed fraud, manipulation, or anomaly detection in investment funds or alternative investments; (2) employed or substantively discussed AI/ML methods for detection, prediction, or classification; (3) were peer-reviewed publications or widely cited preprints with at least 20 citations; and (4) were published in English.

Exclusion criteria eliminated: (1) pure methodology papers without financial application or validation; (2) studies focused exclusively on credit card, payment, or transaction-level fraud without fund-level components; and (3) duplicate publications reporting the same study across multiple venues.

\subsection*{Screening Process and Results}

The initial database searches yielded approximately 500 potentially relevant publications. Title and abstract screening reduced this set to 120 papers warranting full-text review. After detailed examination, 80 papers met all inclusion criteria and form the core of this survey. An additional 25 papers on broader topics (financial regulation, general fraud detection, or foundational ML methods) were included for contextual background, bringing the total reference count to 105.

Two independent reviewers conducted the screening process, with disagreements resolved through discussion and consultation with a third reviewer. The reference lists of highly cited papers were manually reviewed using a snowballing technique to identify additional relevant studies not captured by the database searches. This systematic approach ensures that the survey provides comprehensive coverage of the AI-based hedge fund fraud detection literature while maintaining rigorous inclusion standards.
