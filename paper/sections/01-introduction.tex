% ==================== SECTION 1: INTRODUCTION ====================
\section{Introduction}
\label{sec:introduction}

\subsection{The Scale of Hedge Fund Fraud}
\label{sec:intro-scale}

The global hedge fund industry manages assets exceeding \$4.5 trillion as of 2025, having grown substantially from roughly \$2 trillion at the onset of the 2008 financial crisis. This expansion has been accompanied by an equally striking growth in the complexity and diversity of investment strategies, ranging from quantitative statistical arbitrage to activist equity positions and illiquid credit. Yet unlike mutual funds, which operate under the transparency and reporting requirements of the Investment Company Act of 1940, hedge funds have historically benefited from broad exemptions that permit limited disclosure, voluntary performance reporting, and minimal portfolio-level transparency \citep{stulz2007hedge}. These structural features---while enabling the strategic flexibility that attracts institutional capital---simultaneously create an environment in which fraud can persist undetected for years or even decades.

The financial toll of hedge fund fraud is enormous. The collapse of Bernard L.\ Madoff Investment Securities in December 2008 revealed a Ponzi scheme with estimated losses of \$65 billion in stated account value, making it the largest financial fraud in history \citep{markopolos2010noone}. The Bayou Group, a Connecticut-based hedge fund, concealed roughly \$450 million in trading losses through fabricated financial statements and a sham auditing firm between 2003 and 2005. More recently, the implosion of Archegos Capital Management in March 2021 generated over \$10 billion in counterparty losses across major prime brokers, exposing failures in risk monitoring and concentrated position reporting that regulators had not anticipated. These are not isolated incidents. The U.S.\ Securities and Exchange Commission (SEC) brings dozens of enforcement actions against hedge fund managers and private fund advisers each year, with violations spanning return misrepresentation, asset misappropriation, insider trading, and valuation manipulation.

Several characteristics render hedge funds uniquely vulnerable to fraudulent activity. First, hedge funds frequently invest in illiquid or hard-to-value assets---including distressed debt, private equity co-investments, and bespoke derivatives---for which independent pricing is difficult or impossible to obtain \citep{getmansky2004econometric}. This opacity in valuation creates opportunities for managers to inflate reported net asset values (NAVs), smooth returns to conceal volatility, or fabricate performance altogether. Second, hedge fund reporting to commercial databases such as Hedge Fund Research (HFR), Lipper TASS, and Morningstar is entirely voluntary, introducing well-documented survivorship, backfill, and self-selection biases that complicate statistical analysis \citep{fung2009measurement, agarwal2011hedge}. Third, lock-up periods and redemption gates restrict investor liquidity, delaying the discovery of fraud by preventing investors from withdrawing capital when suspicions arise. Fourth, the limited partnership structures typical of hedge funds concentrate decision-making authority in a small group of general partners, often with minimal independent oversight. Taken together, these features create what \citet{stulz2007hedge} characterize as an agency problem of unusual severity: managers possess both the incentive and the means to misrepresent fund performance, while investors and regulators lack the information and access needed for effective monitoring.

\subsection{Why AI? The Limitations of Traditional Detection}
\label{sec:intro-why-ai}

The mismatch between regulatory capacity and industry scale is stark. The SEC employs approximately 4,600 staff to oversee thousands of registered investment advisers, broker-dealers, and fund complexes, with the Division of Examinations conducting only a fraction of possible inspections in any given year. Human auditors, even when experienced, face fundamental capacity constraints: a single examiner reviewing a hedge fund's monthly return series, trading records, and valuation documentation can assess at most a handful of funds per quarter. This throughput bottleneck means that most hedge funds receive regulatory scrutiny only infrequently, creating long windows during which fraudulent schemes can operate undiscovered.

Beyond capacity, human judgment is subject to well-documented cognitive limitations. The case of Harry Markopolos is instructive. Beginning in 2000, Markopolos repeatedly submitted detailed analyses to the SEC arguing that Madoff's reported returns were statistically implausible, yet the agency failed to act for nearly a decade \citep{markopolos2010noone}. This failure reflected not only institutional shortcomings but also the difficulty of distinguishing genuine skill from fabrication when evaluating complex, opaque strategies. Hindsight bias, confirmation bias, and anchoring effects further impair the ability of human analysts to identify fraud signals that deviate from established templates.

Traditional statistical methods for fraud detection have provided a valuable foundation but remain insufficient on their own. Benford's law analysis---which tests whether the leading digits of reported returns conform to the expected logarithmic distribution---can identify certain forms of data fabrication but is easily defeated by a knowledgeable fraudster who engineers returns to satisfy digit-frequency tests. Serial correlation analysis, which examines suspicious smoothness in reported return series, has been applied with some success to detect NAV manipulation in hedge funds \citep{bollen2012suspicious, getmansky2004econometric}, but it captures only one dimension of a potentially multi-faceted fraud. Similarly, forensic ratio analysis and outlier detection based on univariate distributional properties can flag individual anomalies without capturing the complex, multi-dimensional patterns that characterize sophisticated schemes. \citet{dimmock2012predicting} demonstrated that regulatory disclosure data can predict future fraud, yet their logistic regression approach, while interpretable, operates on a limited feature space and does not scale to the volume or variety of data now available.

Artificial intelligence (AI) and machine learning (ML) methods offer four fundamental advantages that address these limitations. First, \textit{scalability}: ML algorithms can process thousands of fund return series, regulatory filings, and alternative data sources simultaneously, enabling surveillance at a scale that human analysts cannot achieve. Second, \textit{pattern recognition}: methods ranging from ensemble classifiers to deep neural networks can detect subtle, nonlinear, and multi-dimensional anomalies that elude univariate statistical tests. For example, a random forest trained on dozens of return-based features---including higher moments, serial correlation coefficients, and distributional shape statistics---can identify suspicious combinations of characteristics that no single test would flag \citep{bollen2012suspicious}. Third, \textit{real-time monitoring}: once trained and deployed, ML models can evaluate incoming data continuously, enabling early warning systems that alert regulators and investors to emerging risks before losses compound. Fourth, \textit{multi-modal data integration}: modern AI architectures can fuse structured data (return series, regulatory filings, financial ratios) with unstructured data (news articles, social media sentiment, legal documents) and relational data (counterparty networks, manager affiliation graphs), constructing a richer and more holistic picture of fund behavior than any single data modality permits.

Despite this promise, the application of AI to hedge fund fraud detection remains fragmented, methodologically heterogeneous, and largely disconnected from the operational realities of regulatory enforcement. Existing studies span multiple disciplines---computer science, finance, accounting, and law---employ divergent datasets, evaluation metrics, and fraud definitions, and rarely address the adversarial dynamics inherent in financial fraud, where perpetrators actively adapt their behavior to evade detection. This fragmentation motivates the present survey.

\subsection{Survey Scope and Contributions}
\label{sec:intro-contributions}

This paper presents a systematic, qualitative survey of AI-based approaches to hedge fund fraud detection. We synthesize the scattered literature into a coherent analytical framework, identify critical gaps, and propose a concrete research agenda. To the best of our knowledge, no existing survey addresses AI-based fraud detection with a specific focus on the hedge fund context, its unique data challenges, and its distinctive regulatory environment.

Several prior surveys have examined the broader intersection of AI and financial fraud. \citet{ngai2011application} provided an early taxonomy of data mining techniques applied to financial fraud, covering credit card fraud, insurance fraud, and securities fraud but without distinguishing hedge funds from other financial institutions. \citet{abdallah2016fraud} reviewed fraud detection systems across multiple domains and emphasized the importance of class imbalance, yet their treatment of investment fraud remained cursory. \citet{west2016intelligent} surveyed intelligent financial fraud detection, focusing primarily on credit card and payment fraud with limited attention to asset management. \citet{pourhabibi2020fraud} examined fraud detection using process mining and network analysis, contributing valuable methodological perspectives but not addressing the specific data ecosystem of hedge funds. \citet{bao2020detecting} focused on detecting financial statement fraud using deep learning, with an accounting rather than investment management orientation. \citet{hilal2022financial} offered a comprehensive survey of financial fraud detection methods, covering a broad taxonomy of techniques but treating hedge fund fraud only peripherally. More recently, \citet{ahmed2024survey} provided a wide-ranging survey of AI for financial crime detection, and several reviews have examined graph neural networks (GNNs) for fraud detection in financial networks, yet none of these works systematically maps AI detection capabilities to the specific fraud typologies, data structures, and regulatory constraints that characterize the hedge fund industry. \Cref{tab:survey-comparison} summarizes the scope of these existing surveys and highlights the gaps that the present work addresses.

\begin{table}[t]
\centering
\caption{Comparison of this survey with prior related surveys. Checkmarks (\checkmark) indicate that a topic is substantively addressed; dashes (--) indicate peripheral or absent coverage.}
\label{tab:survey-comparison}
\small
\begin{tabularx}{\textwidth}{lcccccc}
\toprule
\textbf{Survey} & \textbf{Hedge Fund} & \textbf{AI/ML} & \textbf{Fraud} & \textbf{Adversarial} & \textbf{Regulatory} & \textbf{Research} \\
 & \textbf{Focus} & \textbf{Methods} & \textbf{Taxonomy} & \textbf{Robustness} & \textbf{Readiness} & \textbf{Agenda} \\
\midrule
\citet{ngai2011application} & -- & \checkmark & -- & -- & -- & -- \\
\citet{abdallah2016fraud} & -- & \checkmark & -- & -- & -- & -- \\
\citet{west2016intelligent} & -- & \checkmark & -- & -- & -- & -- \\
\citet{pourhabibi2020fraud} & -- & \checkmark & -- & -- & -- & -- \\
\citet{bao2020detecting} & -- & \checkmark & -- & -- & -- & -- \\
\citet{hilal2022financial} & -- & \checkmark & -- & -- & -- & \checkmark \\
\citet{ahmed2024survey} & -- & \checkmark & -- & -- & -- & \checkmark \\
\textbf{This survey} & \checkmark & \checkmark & \checkmark & \checkmark & \checkmark & \checkmark \\
\bottomrule
\end{tabularx}
\end{table}

We make three principal contributions:

\begin{description}
    \item[C1: Detection Pipeline Taxonomy.] We propose a unified five-stage framework---spanning data ingestion, feature engineering, model selection, explainability, and deployment---that systematically maps hedge fund fraud types to appropriate AI detection methods. This taxonomy provides researchers and practitioners with a structured lens for understanding which methods apply to which fraud scenarios and where methodological gaps remain. No existing survey provides this hedge-fund-specific mapping.

    \item[C2: Adversarial and Regulatory Readiness Assessment.] We conduct a systematic evaluation of how robust current AI detection methods are to adversarial manipulation by sophisticated hedge fund managers, and we assess whether these methods satisfy emerging regulatory requirements, including the European Union Artificial Intelligence Act (EU AI Act) and SEC guidance on the use of predictive analytics. This assessment bridges the gap between the technical ML literature and the practical demands of regulators and compliance professionals. No prior survey has evaluated AI-based fraud detection through this dual lens.

    \item[C3: Actionable Research Roadmap.] We identify ten concrete open research problems, each differentiated by the specific characteristics of the hedge fund context. For each problem, we suggest methodological approaches, outline evaluation protocols, and discuss feasibility considerations. This roadmap is designed to guide both academic researchers seeking impactful problems and industry practitioners seeking evidence-based solutions.
\end{description}

It is important to note what this survey does \textit{not} attempt. We do not conduct a quantitative meta-analysis of detection performance across studies, as the heterogeneity of datasets, fraud definitions, evaluation protocols, and reporting standards across the existing literature precludes meaningful statistical aggregation. Instead, we adopt a qualitative synthesis approach, critically analyzing the methodological strengths, limitations, and contextual applicability of each body of work. This approach is appropriate given the current state of the field, where standardization of benchmarks and evaluation procedures remains an open challenge that we address explicitly in our research agenda.

\subsection{Paper Organization}
\label{sec:intro-organization}

The remainder of this paper is organized as follows. \Cref{sec:background} establishes the necessary background, presenting a taxonomy of hedge fund fraud types, describing the data ecosystem available for detection research, and summarizing the regulatory context within which detection systems must operate. \Cref{sec:pipeline} introduces our detection pipeline framework (Contribution C1), detailing the five stages through which raw data are transformed into actionable fraud assessments. \Cref{sec:literature} provides a comprehensive qualitative review of AI and ML methods that have been applied---or proposed for application---to hedge fund fraud detection, organized by methodological family and mapped onto our pipeline taxonomy. \Cref{sec:adversarial} addresses adversarial robustness, regulatory readiness, and ethical considerations (Contribution C2), evaluating the extent to which current methods withstand strategic manipulation and satisfy legal requirements. \Cref{sec:research-agenda} presents our research agenda (Contribution C3), articulating ten open problems with suggested approaches and evaluation criteria. Finally, \Cref{sec:conclusion} concludes with a synthesis of key findings and implications for researchers, regulators, and practitioners.
