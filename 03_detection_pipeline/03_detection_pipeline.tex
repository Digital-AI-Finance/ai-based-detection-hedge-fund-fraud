% ============================================================
%  Slide Deck 3 -- A Unified Detection Pipeline Framework (C1)
%  AI-Based Detection of Hedge Fund Fraud
% ============================================================
\documentclass[8pt,aspectratio=169]{beamer}
\usetheme{Madrid}
\usepackage{graphicx}
\usepackage{booktabs}
\usepackage{adjustbox}
\usepackage{multicol}
\usepackage{amsmath}
\usepackage{amssymb}

% ---- Color definitions ----
\definecolor{mlblue}{RGB}{0,102,204}
\definecolor{mlpurple}{RGB}{51,51,178}
\definecolor{mllavender}{RGB}{173,173,224}
\definecolor{mllavender2}{RGB}{193,193,232}
\definecolor{mllavender3}{RGB}{204,204,235}
\definecolor{mllavender4}{RGB}{214,214,239}
\definecolor{mlorange}{RGB}{255, 127, 14}
\definecolor{mlgreen}{RGB}{44, 160, 44}
\definecolor{mlred}{RGB}{214, 39, 40}
\definecolor{mlgray}{RGB}{127, 127, 127}
\definecolor{lightgray}{RGB}{240, 240, 240}
\definecolor{midgray}{RGB}{180, 180, 180}

% ---- Apply custom colors to Madrid theme ----
\setbeamercolor{palette primary}{bg=mllavender3,fg=mlpurple}
\setbeamercolor{palette secondary}{bg=mllavender2,fg=mlpurple}
\setbeamercolor{palette tertiary}{bg=mllavender,fg=white}
\setbeamercolor{palette quaternary}{bg=mlpurple,fg=white}
\setbeamercolor{structure}{fg=mlpurple}
\setbeamercolor{section in toc}{fg=mlpurple}
\setbeamercolor{subsection in toc}{fg=mlblue}
\setbeamercolor{title}{fg=mlpurple}
\setbeamercolor{frametitle}{fg=mlpurple,bg=mllavender3}
\setbeamercolor{block title}{bg=mllavender2,fg=mlpurple}
\setbeamercolor{block body}{bg=mllavender4,fg=black}

% ---- Navigation / itemize ----
\setbeamertemplate{navigation symbols}{}
\setbeamertemplate{itemize items}[circle]
\setbeamertemplate{enumerate items}[default]
\setbeamersize{text margin left=5mm,text margin right=5mm}

% ---- Custom commands ----
\newcommand{\bottomnote}[1]{%
\vfill
\vspace{-2mm}
\textcolor{mllavender2}{\rule{\textwidth}{0.4pt}}
\vspace{1mm}
\footnotesize
\textbf{#1}
}

\newcommand{\compactlist}{%
\setlength{\itemsep}{0pt}%
\setlength{\parskip}{0pt}%
\setlength{\parsep}{0pt}%
}

\newcommand{\chartplaceholder}[2][5cm]{%
\begin{center}
\begin{adjustbox}{max width=0.95\textwidth, max height=#1}
\framebox[\textwidth][c]{%
\rule{0pt}{#1}%
\textcolor{midgray}{[#2]}%
}
\end{adjustbox}
\end{center}
}

% ---- Notation ----
% Shared notation macros for AI-Based Detection of Hedge Fund Fraud
% Include this file in all Beamer slide decks via % Shared notation macros for AI-Based Detection of Hedge Fund Fraud
% Include this file in all Beamer slide decks via % Shared notation macros for AI-Based Detection of Hedge Fund Fraud
% Include this file in all Beamer slide decks via \input{notation}

% Performance metrics
\newcommand{\auc}{\ensuremath{\mathrm{AUC}}}
\newcommand{\fone}{\ensuremath{F_1}}
\newcommand{\shap}{\text{SHAP}}
\newcommand{\lime}{\text{LIME}}

% Autocorrelation
\newcommand{\rhoone}{\ensuremath{\rho_1}}

% Financial abbreviations
\newcommand{\nav}{\text{NAV}}
\newcommand{\aum}{\text{AUM}}

% Regulatory
\newcommand{\euaiact}{EU AI Act}
\newcommand{\aifmd}{AIFMD}
\newcommand{\sec}{SEC}
\newcommand{\edgar}{EDGAR}

% Key numbers from the paper (for consistency)
\newcommand{\aumdollar}{\$4.5\text{ trillion}}
\newcommand{\aucdeg}{10.6\%}
\newcommand{\numop}{10}
\newcommand{\numfraudcases}{50\text{--}100}
\newcommand{\systematicpapers}{105}


% Performance metrics
\newcommand{\auc}{\ensuremath{\mathrm{AUC}}}
\newcommand{\fone}{\ensuremath{F_1}}
\newcommand{\shap}{\text{SHAP}}
\newcommand{\lime}{\text{LIME}}

% Autocorrelation
\newcommand{\rhoone}{\ensuremath{\rho_1}}

% Financial abbreviations
\newcommand{\nav}{\text{NAV}}
\newcommand{\aum}{\text{AUM}}

% Regulatory
\newcommand{\euaiact}{EU AI Act}
\newcommand{\aifmd}{AIFMD}
\newcommand{\sec}{SEC}
\newcommand{\edgar}{EDGAR}

% Key numbers from the paper (for consistency)
\newcommand{\aumdollar}{\$4.5\text{ trillion}}
\newcommand{\aucdeg}{10.6\%}
\newcommand{\numop}{10}
\newcommand{\numfraudcases}{50\text{--}100}
\newcommand{\systematicpapers}{105}


% Performance metrics
\newcommand{\auc}{\ensuremath{\mathrm{AUC}}}
\newcommand{\fone}{\ensuremath{F_1}}
\newcommand{\shap}{\text{SHAP}}
\newcommand{\lime}{\text{LIME}}

% Autocorrelation
\newcommand{\rhoone}{\ensuremath{\rho_1}}

% Financial abbreviations
\newcommand{\nav}{\text{NAV}}
\newcommand{\aum}{\text{AUM}}

% Regulatory
\newcommand{\euaiact}{EU AI Act}
\newcommand{\aifmd}{AIFMD}
\newcommand{\sec}{SEC}
\newcommand{\edgar}{EDGAR}

% Key numbers from the paper (for consistency)
\newcommand{\aumdollar}{\$4.5\text{ trillion}}
\newcommand{\aucdeg}{10.6\%}
\newcommand{\numop}{10}
\newcommand{\numfraudcases}{50\text{--}100}
\newcommand{\systematicpapers}{105}


% ---- Title metadata ----
\title{A Unified Detection Pipeline Framework (C1)}
\subtitle{Section 3 -- AI-Based Detection of Hedge Fund Fraud}
\author{Joerg Osterrieder}
\institute{Zurich University of Applied Sciences (ZHAW)}
\date{2025}

% ============================================================
\begin{document}

% ----------------------------------------------------------
% SLIDE 1 -- Title
% ----------------------------------------------------------
\begin{frame}
\titlepage
\end{frame}

% ----------------------------------------------------------
% SLIDE 2 -- Outline
% ----------------------------------------------------------
\begin{frame}{Outline}
\begin{enumerate}\compactlist
\item Pipeline Overview: Five Stages
\item Stage 1: Data Ingestion and Integration
  \begin{itemize}\compactlist
  \item Temporal alignment, data quality, entity resolution, fusion
  \end{itemize}
\item Stage 2: Feature Engineering (5 families)
  \begin{itemize}\compactlist
  \item Statistical, Benford's law, textual, network, temporal
  \end{itemize}
\item Stage 3: Model Selection and Training (6 families)
  \begin{itemize}\compactlist
  \item Classical ML, deep learning, anomaly detection, NLP, GNN, generative
  \end{itemize}
\item Stage 4: Explainability and Interpretation
\item Stage 5: Deployment and Monitoring
\item Summary
\end{enumerate}
\end{frame}

% ----------------------------------------------------------
% SLIDE 3 -- Pipeline Overview
% ----------------------------------------------------------
\begin{frame}{Pipeline Overview: Five Stages}
\begin{columns}[T]
\column{0.45\textwidth}
\begin{enumerate}\compactlist
\item \textbf{Data Ingestion \& Integration}\\
  Multi-source collection, temporal alignment, entity resolution
\item \textbf{Feature Engineering}\\
  5 feature families: statistical, Benford, textual, network, temporal
\item \textbf{Model Selection \& Training}\\
  6 method families matched to fraud types
\item \textbf{Explainability \& Interpretation}\\
  Regulatory transparency (\euaiact{} Art.\ 13)
\item \textbf{Deployment \& Monitoring}\\
  Batch/real-time, drift detection, human-in-the-loop
\end{enumerate}

\column{0.52\textwidth}
\textbf{Key Design Principles}
\begin{itemize}\compactlist
\item Pipeline is \textbf{not strictly unidirectional}
\item Feedback loop: deployment $\to$ feature engineering, model training
\item Incorporates investigator judgments, adapts to evolving fraud patterns
\item No prior work assembles these components into a coherent framework \textit{tailored to hedge funds}
\end{itemize}
\vspace{2mm}
\textbf{Three Purposes}
\begin{itemize}\compactlist
\item Researchers: structured lens for positioning contributions
\item Practitioners: engineering blueprint for operational systems
\item Gap analysis: reveals methodological gaps $\to$ research agenda
\end{itemize}
\end{columns}
\bottomnote{Source: Paper Section 3.1 -- Contribution C1}
\end{frame}

% ----------------------------------------------------------
% SLIDE 4 -- Stage 1: Data Ingestion Overview
% ----------------------------------------------------------
\begin{frame}{Stage 1: Data Ingestion and Integration}
\begin{itemize}\compactlist
\item Assemble a coherent analytical dataset from sources differing in \textbf{structure, frequency, reliability, provenance}
\item Four layers: returns (monthly, numerical), filings (quarterly/annual, text), alternative (continuous, heterogeneous), synthetic (on-demand)
\item Four sub-problems must be solved:
\end{itemize}
\vspace{3mm}
\begin{columns}[T]
\column{0.24\textwidth}
\begin{block}{\small Temporal Alignment}
\small Different cadences:\\returns (monthly), filings (quarterly), alt data (continuous)
\end{block}

\column{0.24\textwidth}
\begin{block}{\small Data Quality}
\small Survivorship (+242 bp), backfill (+442 bp), selection biases
\end{block}

\column{0.24\textwidth}
\begin{block}{\small Entity Resolution}
\small Same fund $\to$ different IDs across databases (TASS, HFR, CRD, LEI)
\end{block}

\column{0.24\textwidth}
\begin{block}{\small Multi-Source Fusion}
\small Early vs.\ late fusion architectures
\end{block}
\end{columns}
\bottomnote{Source: Paper Section 3.2}
\end{frame}

% ----------------------------------------------------------
% SLIDE 5 -- Temporal Alignment
% ----------------------------------------------------------
\begin{frame}{Stage 1: Temporal Alignment Challenge}
\begin{columns}[T]
\column{0.50\textwidth}
\textbf{Reporting Cadences}
\begin{itemize}\compactlist
\item Return data: monthly, 30--60 day lag
\item Form ADV: annual + ad hoc amendments
\item Form 13F: quarterly, 45-day delay
\item News / social media: continuous, irregular
\end{itemize}
\vspace{2mm}
\textbf{Alignment Strategies}
\begin{itemize}\compactlist
\item Aggregate to quarter-end $\Rightarrow$ discards intra-quarter dynamics
\item Multi-resolution architectures: process each stream at native frequency before fusion
\end{itemize}

\column{0.47\textwidth}
\textbf{Why It Matters}
\begin{itemize}\compactlist
\item End-of-quarter return spikes (Bollen \& Pool, 2012) = potential manipulation signals
\item Choice of alignment strategy determines which temporal patterns remain visible
\item Misalignment can introduce spurious correlations or mask genuine signals
\end{itemize}
\vspace{2mm}
\textcolor{mlblue}{\textbf{Practical recommendation:} hybrid architecture that preserves native frequency per source, fuses at the model input stage}
\end{columns}
\bottomnote{Source: Bollen \& Pool (2012); paper Section 3.2}
\end{frame}

% ----------------------------------------------------------
% SLIDE 6 -- Data Quality and Bias
% ----------------------------------------------------------
\begin{frame}{Stage 1: Data Quality and Bias Correction}
\begin{itemize}\compactlist
\item Biases propagate through the \textbf{entire pipeline} if uncorrected
\end{itemize}
\vspace{2mm}
\begin{columns}[T]
\column{0.48\textwidth}
\textbf{Bias Correction Procedures}
\begin{itemize}\compactlist
\item \textbf{Backfill}: restrict analysis to post-reporting-inception returns only
\item \textbf{Survivorship}: require databases with graveyard (defunct) fund records
\item \textbf{Selection}: model the reporting decision process itself as an informative signal
\end{itemize}

\column{0.48\textwidth}
\textbf{Fraud-Specific Asymmetry}
\begin{itemize}\compactlist
\item Detected frauds $\to$ graveyard section
\item \textcolor{mlred}{Undetected frauds remain in live data}
\item ``Clean'' training class is contaminated
\item Models trained on biased data underestimate base rate of fraud
\item May miss signatures of currently active schemes
\end{itemize}
\end{columns}
\bottomnote{Source: Fung \& Hsieh (2009); Agarwal et al.\ (2011); paper Section 3.2}
\end{frame}

% ----------------------------------------------------------
% SLIDE 7 -- Entity Resolution
% ----------------------------------------------------------
\begin{frame}{Stage 1: Entity Resolution}
\begin{columns}[T]
\column{0.50\textwidth}
\textbf{The Problem}
\begin{itemize}\compactlist
\item Same fund $\Rightarrow$ different identifiers across sources:
  \begin{itemize}\compactlist
  \item TASS fund ID $\neq$ HFR ID $\neq$ SEC CRD number $\neq$ LEI
  \end{itemize}
\item Managers may \textbf{deliberately obscure} connections to prior failed/sanctioned funds
\item Inability to link is itself a \textbf{fraud-relevant signal}
\end{itemize}
\vspace{2mm}
\textbf{Methods}
\begin{itemize}\compactlist
\item Approximate string matching (fund/manager names)
\item Shared-attribute clustering (addresses, auditors, administrators, prime brokers)
\item Graph-based entity resolution (propagate identity evidence through shared relationships)
\end{itemize}

\column{0.47\textwidth}
\textbf{Post-Dodd-Frank Improvement}
\begin{itemize}\compactlist
\item Form ADV filing since 2010 provides \textbf{stable CRD numbers} for US-registered advisers
\item CRD serves as linkage key across databases
\item Substantially improved entity resolution for US funds
\end{itemize}
\vspace{2mm}
\textbf{Remaining Gaps}
\begin{itemize}\compactlist
\item Non-US funds lack stable identifiers
\item Offshore structures (Cayman Islands, BVI) deliberately fragment entity trails
\item Graph-based methods offer robustness but at higher computational cost
\end{itemize}
\end{columns}
\bottomnote{Source: Brown et al.\ (2008); paper Section 3.2}
\end{frame}

% ----------------------------------------------------------
% SLIDE 8 -- Stage 2: Feature Engineering Overview
% ----------------------------------------------------------
\begin{frame}{Stage 2: Feature Engineering -- Five Families}
\begin{center}
\small
\begin{tabular}{lp{3.5cm}p{3.5cm}p{3.5cm}}
\toprule
\textbf{Family} & \textbf{Key Features} & \textbf{Target Fraud Types} & \textbf{Data Source} \\
\midrule
Statistical & $\rhoone$, Sharpe, skewness, kurtosis, max drawdown, Hurst $H$ & Performance fabrication, NAV manipulation & Return series \\
Benford's law & First-digit $\chi^2$, second-digit, summation test, KS statistic & Data fabrication & Returns, NAVs \\
Textual & Fog index, FinBERT sentiment, boilerplate deviation, topic drift & Strategy misrep., regulatory fraud & Form ADV, letters \\
Network & Auditor risk, manager history, co-investment centrality, clustering & Allocation fraud, serial offenders & Filings, databases \\
Temporal & HMM regimes, change-points, calendar effects, momentum/reversal & All types (dynamic dimension) & Return series \\
\bottomrule
\end{tabular}
\end{center}
\bottomnote{Source: Paper Section 3.3}
\end{frame}

% ----------------------------------------------------------
% SLIDE 9 -- Statistical Features
% ----------------------------------------------------------
\begin{frame}{Statistical Features}
\begin{columns}[T]
\column{0.48\textwidth}
\textbf{Serial Correlation}
\begin{itemize}\compactlist
\item $\rhoone = \mathrm{Corr}(r_t, r_{t-1})$: proxy for return smoothing
\item Typical $\rhoone = 0.3$--$0.5$ for illiquid positions
\item Higher-order: $\rho_2, \rho_3$, Ljung-Box $Q$-statistic
\item Abnormally high for funds claiming liquid assets $\Rightarrow$ NAV manipulation
\end{itemize}
\vspace{2mm}
\textbf{Distributional Discontinuity}
\begin{itemize}\compactlist
\item Bollen-Pool ``kink'' at zero: excess small positives, deficit of small negatives
\item Quantified via kernel density estimation or histogram structural break
\end{itemize}

\column{0.48\textwidth}
\textbf{Higher Moments \& Risk-Adjusted}
\begin{itemize}\compactlist
\item Sharpe ratio $S = \bar{r}/\sigma_r$: implausibly high $\Rightarrow$ fabrication
\item Skewness, excess kurtosis: fabricated returns $\approx$ zero skew, low kurtosis
\item Maximum drawdown: difficult to fake over long horizons
\end{itemize}
\vspace{2mm}
\textbf{Long-Memory Detection}
\begin{itemize}\compactlist
\item Hurst exponent $H$ (R/S analysis, DFA)
\item $H \gg 0.5$: persistent serial dependence $\Rightarrow$ smoothing
\item Combined with GARCH residual analysis for volatility clustering
\end{itemize}
\end{columns}
\bottomnote{Source: Getmansky et al.\ (2004); Lo (2001); paper Section 3.3.1}
\end{frame}

% ----------------------------------------------------------
% SLIDE 10 -- Benford's Law Features
% ----------------------------------------------------------
\begin{frame}{Benford's Law Features}
\begin{columns}[T]
\column{0.48\textwidth}
\textbf{Foundation}
\begin{itemize}\compactlist
\item Leading digit $d$: $P(d) = \log_{10}(1 + 1/d)$
\item Naturally occurring data follow this logarithmic distribution
\item Human-generated / engineered numbers often fail to reproduce it
\end{itemize}
\vspace{2mm}
\textbf{Three Test Types}
\begin{itemize}\compactlist
\item \textbf{First-digit test}: $\chi^2 = \sum_{d=1}^{9}\frac{(O_d - E_d)^2}{E_d}$
\item \textbf{Second-digit test}: more sensitive to subtle manipulation (fraudsters engineer first digits but neglect second)
\item \textbf{Summation test}: detects round-number manipulation in reported amounts
\end{itemize}

\column{0.48\textwidth}
\textbf{ML Feature Space}
\begin{itemize}\compactlist
\item 9 first-digit frequencies + 10 second-digit frequencies + 2 test statistics ($\chi^2$, KS)
\item \textbf{21-dimensional feature vector} per fund
\item Enables detection of complex digit manipulation patterns:
  \begin{itemize}\compactlist
  \item E.g., first digits conform but second digits anomalously uniform
  \end{itemize}
\end{itemize}
\vspace{2mm}
\textbf{Limitations}
\begin{itemize}\compactlist
\item Low statistical power for short return histories ($< 60$ months)
\item Knowledgeable fraudster can engineer conformity
\item Most valuable as one component in multi-feature pipeline
\end{itemize}
\end{columns}
\bottomnote{Source: Benford (1938); Nigrini (2012); paper Section 3.3.2}
\end{frame}

% ----------------------------------------------------------
% SLIDE 11 -- Textual Features
% ----------------------------------------------------------
\begin{frame}{Textual Features from Regulatory Filings}
\begin{columns}[T]
\column{0.48\textwidth}
\textbf{Filing Complexity / Readability}
\begin{itemize}\compactlist
\item Gunning Fog: $0.4 \times (\text{ASL} + \text{PHW})$
\item Firms engaged in misconduct tend to produce more complex filings
\item Word count, sentence count, type-token ratio
\end{itemize}
\vspace{2mm}
\textbf{Sentiment Analysis}
\begin{itemize}\compactlist
\item FinBERT: BERT fine-tuned on financial text
\item SEC-BERT: pre-trained on EDGAR filings
\item Aggregate sentiment, sentiment volatility, proportion of hedging language
\end{itemize}

\column{0.48\textwidth}
\textbf{Boilerplate Deviation}
\begin{itemize}\compactlist
\item Cosine similarity vs.\ peer-template mean TF representation
\item Unusual deviation (vague \textit{or} suspiciously precise) $\Rightarrow$ scrutiny
\item \textbf{Temporal trajectory} more informative than single snapshot:
  \begin{itemize}\compactlist
  \item Deteriorating readability
  \item Increasing hedging language
  \item Growing divergence from prior filings
  \end{itemize}
\end{itemize}
\vspace{2mm}
\textbf{Topic Modeling}
\begin{itemize}\compactlist
\item LDA / neural topic models on strategy descriptions
\item Cross-modal consistency: text strategy vs.\ quantitative factor exposures
\end{itemize}
\end{columns}
\bottomnote{Source: Araci (2019); Loukas et al.\ (2022); paper Section 3.3.3}
\end{frame}

% ----------------------------------------------------------
% SLIDE 12 -- Network Features
% ----------------------------------------------------------
\begin{frame}{Network and Relational Features}
\begin{columns}[T]
\column{0.48\textwidth}
\textbf{Fund--Service-Provider Graphs}
\begin{itemize}\compactlist
\item Bipartite graph: funds $\leftrightarrow$ auditors, administrators, custodians
\item Small, non-Big-Four auditors over-represented among sanctioned funds
\item Auditor \textit{change} from reputable to small firm $+$ other risk signals $\Rightarrow$ weakening oversight
\item Node features: client count, historical sanctions, change frequency
\end{itemize}
\vspace{2mm}
\textbf{Manager History Networks}
\begin{itemize}\compactlist
\item ``Serial offenders'': new fund after prior failure/sanction
\item Prior sanctions = significant fraud predictor (Dimmock \& Gerken, 2012)
\end{itemize}

\column{0.48\textwidth}
\textbf{Co-Investment / Capital Flow Networks}
\begin{itemize}\compactlist
\item Funds sharing common investors or correlated capital flows
\item Centrality measures as fund-level features:
  \begin{itemize}\compactlist
  \item \textbf{Betweenness}: bridges disconnected investor communities (Ponzi signature)
  \item \textbf{Degree}: number of connections
  \item \textbf{Eigenvector}: importance of neighbors
  \item \textbf{Clustering coefficient}: local density
  \end{itemize}
\end{itemize}
\vspace{2mm}
\textcolor{mlblue}{\textbf{``Guilt by association''}: individually inconspicuous funds embedded in suspicious relational structures}
\end{columns}
\bottomnote{Source: Brown et al.\ (2008); Dimmock \& Gerken (2012); paper Section 3.3.4}
\end{frame}

% ----------------------------------------------------------
% SLIDE 13 -- Temporal Features
% ----------------------------------------------------------
\begin{frame}{Temporal Features}
\begin{columns}[T]
\column{0.48\textwidth}
\textbf{Regime Detection (HMMs)}
\begin{itemize}\compactlist
\item Hidden Markov Models: ``normal'' vs.\ ``manipulated'' regime
\item Different mean, variance, autocorrelation per state
\item Transition probabilities and regime timing as features
\item Distinguishes legitimate adaptation from suspicious behavioral shifts
\end{itemize}
\vspace{2mm}
\textbf{Change-Point Detection}
\begin{itemize}\compactlist
\item Patton \& Ramadorai (2015): structural breaks precede fund failure
\item Bayesian Online Change Point Detection (BOCPD)
\item Features: number of change points, spacing, magnitude
\item Frequent, large shifts without identifiable market events $\Rightarrow$ suspicious
\end{itemize}

\column{0.48\textwidth}
\textbf{Calendar Effects}
\begin{itemize}\compactlist
\item Consistently higher December returns
\item Abnormally positive last-trading-day-of-quarter returns
\item Signals end-of-period NAV manipulation
\item Computed as month/day-of-quarter dummy coefficients
\end{itemize}
\vspace{2mm}
\textbf{Momentum / Reversal Patterns}
\begin{itemize}\compactlist
\item Autocorrelation at multiple lags
\item Legitimate strategies: characteristic momentum-reversal from investment style
\item Fabricated returns: artificially smooth momentum \textit{without} mean-reversion imposed by fundamentals
\end{itemize}
\end{columns}
\bottomnote{Source: Patton \& Ramadorai (2015); paper Section 3.3.5}
\end{frame}

% ----------------------------------------------------------
% SLIDE 14 -- Feature Families Chart
% ----------------------------------------------------------
\begin{frame}{Feature Families: Visual Summary}

\chartplaceholder[5.5cm]{Chart: 02\_feature\_families -- Matrix or radar showing five feature families (statistical, Benford, textual, network, temporal) mapped to five fraud types with relative importance and data requirements}

\bottomnote{Source: Paper Section 3.3}
\end{frame}

% ----------------------------------------------------------
% SLIDE 15 -- Stage 3: Model Selection Overview
% ----------------------------------------------------------
\begin{frame}{Stage 3: Model Selection -- Six Families}
\begin{center}
\small
\begin{tabular}{lp{3.5cm}p{4cm}l}
\toprule
\textbf{Family} & \textbf{Key Methods} & \textbf{Strengths} & \textbf{Best For} \\
\midrule
Classical ML & Logistic regression, SVM, RF, XGBoost & Interpretable, handles mixed features & Tabular, small $n$ \\
Deep learning & LSTM, CNN, Transformer & Temporal patterns, nonlinear & Sequential data \\
Anomaly detection & Isolation Forest, LOF, autoencoder & No labels needed & Label-scarce \\
NLP / text mining & FinBERT, SEC-BERT, TF-IDF & Text signals, cross-modal & Filings \\
Graph neural networks & GCN, GAT, GraphSAGE & Relational ``guilt by association'' & Network data \\
Generative & GANs, VAEs & Anomaly detection + augmentation & Class imbalance \\
\bottomrule
\end{tabular}
\end{center}
\vspace{2mm}
\textbf{Context-specific challenges}: extreme class imbalance, small sample sizes, heterogeneous fraud types, multi-modal inputs, adversarial dynamics.
\bottomnote{Source: Paper Section 3.4}
\end{frame}

% ----------------------------------------------------------
% SLIDE 16 -- Classical ML
% ----------------------------------------------------------
\begin{frame}{Classical ML: Logistic Regression, SVM, RF, XGBoost}
\begin{columns}[T]
\column{0.48\textwidth}
\textbf{Logistic Regression}
\begin{itemize}\compactlist
\item Interpretable baseline: coefficients = log-odds contributions
\item Dimmock \& Gerken (2012) on Form ADV: \auc{} $\approx 0.65$--$0.70$
\item Limitation: assumes linear feature-response
\end{itemize}
\vspace{2mm}
\textbf{SVM}
\begin{itemize}\compactlist
\item Maximum-margin hyperplanes
\item One-class SVM: semi-supervised, no fraud labels needed
\item Good on small datasets; limited scalability and interpretability
\end{itemize}

\column{0.48\textwidth}
\textbf{Random Forests}
\begin{itemize}\compactlist
\item Hundreds of trees, bootstrap + random features
\item Handles high-dimensional, mixed-type features naturally
\item Robust to outliers, missing values
\item Feature importance via permutation / mean decrease in impurity
\end{itemize}
\vspace{2mm}
\textbf{Gradient Boosting (XGBoost, LightGBM, CatBoost)}
\begin{itemize}\compactlist
\item State of the art for tabular classification
\item Built-in class imbalance handling
\item Stacking ensembles: \fone{} $\sim$\textbf{0.88} (Hilal et al., 2022)
\item CatBoost: native categorical feature support
\end{itemize}
\end{columns}
\bottomnote{Source: Breiman (2001); Chen et al.\ (2016); Hilal et al.\ (2022); paper Section 3.4.1}
\end{frame}

% ----------------------------------------------------------
% SLIDE 17 -- Deep Learning
% ----------------------------------------------------------
\begin{frame}{Deep Learning: LSTM, CNN, Transformer}
\begin{columns}[T]
\column{0.32\textwidth}
\begin{block}{LSTM}
\begin{itemize}\compactlist
\item Gated memory cells capture long-range sequential dependencies
\item Natural fit for monthly return time series
\item Learns temporal patterns: gradual smoothing onset, regime transitions
\end{itemize}
\end{block}

\column{0.32\textwidth}
\begin{block}{CNN}
\begin{itemize}\compactlist
\item Returns $\to$ 2D representation (Gramian Angular Fields, recurrence plots)
\item Spatial pattern detection on visual representations
\item ``Smooth upward ramp'' of Ponzi scheme
\end{itemize}
\end{block}

\column{0.32\textwidth}
\begin{block}{Transformer}
\begin{itemize}\compactlist
\item Self-attention over all sequence positions
\item Handles long-range dependencies in sparse monthly series
\item Attention weights = built-in explainability
\end{itemize}
\end{block}
\end{columns}
\vspace{3mm}
\textcolor{mlred}{\textbf{Key challenge:}} hedge fund return series are extremely short (60--120 months) and $<100$ positive labels exist. Regularization, pre-training, and transfer learning are \textbf{essential} to prevent overfitting.
\bottomnote{Source: Hochreiter \& Schmidhuber (1997); Vaswani et al.\ (2017); paper Section 3.4.2}
\end{frame}

% ----------------------------------------------------------
% SLIDE 18 -- Anomaly Detection
% ----------------------------------------------------------
\begin{frame}{Anomaly Detection: Isolation Forest, LOF, Autoencoders}
\begin{columns}[T]
\column{0.48\textwidth}
\textbf{Isolation Forest}
\begin{itemize}\compactlist
\item Random recursive partitioning; anomalies isolated with fewer splits
\item Efficient, high-dimensional, no distributional assumptions
\item Detects \textit{global} anomalies (vs.\ entire population)
\end{itemize}
\vspace{2mm}
\textbf{Local Outlier Factor (LOF)}
\begin{itemize}\compactlist
\item Local density deviation vs.\ $k$-nearest neighbors
\item Detects \textit{local} anomalies (unusual within strategy peer group)
\item Critical: strategy-specific norms differ substantially
\end{itemize}

\column{0.48\textwidth}
\textbf{Deep Autoencoders}
\begin{itemize}\compactlist
\item Learn compressed representation of normal behavior
\item High reconstruction error $\Rightarrow$ anomaly
\item \auc{} $\approx$ \textbf{0.79} on hedge fund returns (Chalapathy \& Chawla, 2019)
\item Competitive with supervised methods \textit{without requiring fraud labels}
\end{itemize}
\vspace{2mm}
\textbf{DBSCAN}
\begin{itemize}\compactlist
\item Density-based clustering; noise points = potential anomalies
\item Fund not assignable to any peer group $\Rightarrow$ possible strategy misrepresentation
\end{itemize}
\end{columns}
\vspace{2mm}
\textcolor{mlblue}{Unsupervised methods are especially valuable when confirmed fraud labels are scarce and potentially biased toward historically detected types.}
\bottomnote{Source: Liu et al.\ (2008); Breunig et al.\ (2000); Chalapathy \& Chawla (2019); paper Section 3.4.3}
\end{frame}

% ----------------------------------------------------------
% SLIDE 19 -- NLP and GNN Methods
% ----------------------------------------------------------
\begin{frame}{NLP and Graph Neural Network Methods}
\begin{columns}[T]
\column{0.48\textwidth}
\textbf{NLP / Text Mining}
\begin{itemize}\compactlist
\item Evolution: bag-of-words $\to$ TF-IDF $\to$ word2vec $\to$ transformers
\item \textbf{FinBERT}: financial-domain sentiment
\item \textbf{SEC-BERT}: EDGAR-specific pre-training
\item Applications:
  \begin{itemize}\compactlist
  \item Detect vague/evasive strategy descriptions
  \item Filing inconsistencies across time
  \item Text-quant divergence (stated strategy vs.\ factor exposures)
  \end{itemize}
\item Multi-modal fusion (NLP + returns) more robust: fraudster must maintain both textual \textit{and} statistical consistency
\end{itemize}

\column{0.48\textwidth}
\textbf{Graph Neural Networks}
\begin{itemize}\compactlist
\item \textbf{GCN}: spectral graph convolutions, neighborhood aggregation
\item \textbf{GAT}: attention-weighted neighbor contributions (heterogeneous graphs)
\item \textbf{GraphSAGE}: inductive learning on unseen nodes (new fund cold-start)
\item \textbf{Temporal knowledge graphs}: time-stamped edges capture ``flight from oversight'' patterns
\end{itemize}
\vspace{2mm}
\textbf{Key Advantage}
\begin{itemize}\compactlist
\item ``Guilt by association'': individually benign funds in suspicious relational structures
\end{itemize}
\textbf{Key Limitation}
\begin{itemize}\compactlist
\item Graph construction requires entity resolution across databases
\end{itemize}
\end{columns}
\bottomnote{Source: Kipf \& Welling (2017); Velickovic et al.\ (2018); paper Sections 3.4.4--3.4.5}
\end{frame}

% ----------------------------------------------------------
% SLIDE 20 -- Generative Methods
% ----------------------------------------------------------
\begin{frame}{Generative and Synthetic Methods}
\begin{columns}[T]
\column{0.48\textwidth}
\textbf{Dual Role}
\begin{itemize}\compactlist
\item \textit{Anomaly detection}: learn normal distribution, flag deviations
\item \textit{Data augmentation}: generate synthetic fraud examples for class imbalance
\end{itemize}
\vspace{2mm}
\textbf{GAN-Based Anomaly Detection}
\begin{itemize}\compactlist
\item BiGAN, AnoGAN: generator learns normal returns
\item High reconstruction error in latent space $\Rightarrow$ anomalous
\item Adversarial training captures non-Gaussian features and temporal dependencies
\end{itemize}

\column{0.48\textwidth}
\textbf{Synthetic Data Generation}
\begin{itemize}\compactlist
\item Beyond SMOTE: conditional GANs and VAEs
\item Conditional generation essential: different fraud types $\to$ different signatures
\item Wasserstein GAN: earth-mover distance for financial time series
\end{itemize}
\vspace{2mm}
\textbf{Validation Requirements}
\begin{itemize}\compactlist
\item Generated samples must pass domain-specific plausibility checks:
  \begin{itemize}\compactlist
  \item Realistic return magnitudes
  \item Appropriate serial correlation
  \item Sensible Sharpe ratios
  \end{itemize}
\end{itemize}
\end{columns}
\bottomnote{Source: Goodfellow et al.\ (2014); Kingma \& Welling (2014); paper Section 3.4.6}
\end{frame}

% ----------------------------------------------------------
% SLIDE 21 -- Model Families Chart
% ----------------------------------------------------------
\begin{frame}{Model Families: Visual Summary}

\chartplaceholder[5.5cm]{Chart: 03\_model\_families -- Comparison of six model families showing reported performance ranges (AUC, F1), data requirements, interpretability level, and fraud-type suitability}

\bottomnote{Source: Paper Section 3.4}
\end{frame}

% ----------------------------------------------------------
% SLIDE 22 -- Stage 4: Explainability
% ----------------------------------------------------------
\begin{frame}{Stage 4: Explainability and Interpretation}
\begin{columns}[T]
\column{0.48\textwidth}
\textbf{\shap{} Values}
\begin{itemize}\compactlist
\item Decompose each prediction into per-feature contributions (Shapley values)
\item Example: ``35\% from high $\rhoone$, 25\% from non-Big-Four auditor with 2 prior sanctions, 20\% from deteriorating readability, 20\% from peripheral network position''
\item Maps directly to SEC investigative categories
\end{itemize}
\vspace{2mm}
\textbf{\lime{}}
\begin{itemize}\compactlist
\item Local interpretable model around each prediction
\item Model-agnostic: works with deep nets, GNNs
\item Tailored case-by-case explanations
\end{itemize}

\column{0.48\textwidth}
\textbf{Attention Visualization}
\begin{itemize}\compactlist
\item Transformer: which historical periods are most relevant
\item GAT: which relational connections drove the score
\item Intrinsic explainability (no post-hoc layer needed)
\end{itemize}
\vspace{2mm}
\textbf{Regulatory Requirement}
\begin{itemize}\compactlist
\item \euaiact{} Art.\ 13: ``sufficiently transparent to enable deployers to interpret''
\item Opaque probability scores $\to$ \textcolor{mlred}{insufficient}
\item Structured explanation $\to$ \textcolor{mlgreen}{actionable investigation plan}
\item Three independent lines of inquiry per alert: return analysis, operational due diligence, filing review
\end{itemize}
\end{columns}
\bottomnote{Source: Lundberg \& Lee (2017); Ribeiro et al.\ (2016); EU AI Act Art.\ 13; paper Section 3.5}
\end{frame}

% ----------------------------------------------------------
% SLIDE 23 -- Stage 5: Deployment
% ----------------------------------------------------------
\begin{frame}{Stage 5: Deployment and Monitoring}
\begin{columns}[T]
\column{0.48\textwidth}
\textbf{Processing Cadence}
\begin{itemize}\compactlist
\item Hedge fund surveillance is \textbf{batch-oriented} (monthly/quarterly data)
\item Hybrid: batch scoring of full fund universe $+$ event-triggered inter-batch reassessment
\end{itemize}
\vspace{2mm}
\textbf{Concept Drift Detection}
\begin{itemize}\compactlist
\item Models degrade: legitimate strategies evolve; fraudsters adapt
\item ADWIN: variable-length window, distribution comparison
\item DDM: monitors error rate vs.\ historical baseline
\item Hybrid retraining: scheduled $+$ drift-triggered emergency updates
\end{itemize}

\column{0.48\textwidth}
\textbf{Human-in-the-Loop}
\begin{itemize}\compactlist
\item Designed for \textbf{collaboration, not replacement}
\item Investigator feedback loop $\to$ feature engineering and model training
\item Active learning: query cases most likely to improve decision boundary
\item Reduces alert fatigue from false positives
\item EU AI Act Art.\ 14: human oversight requirement satisfied
\end{itemize}
\vspace{2mm}
\textbf{Alert Prioritization}
\begin{itemize}\compactlist
\item Composite score: fraud probability $\times$ AUM exposure $\times$ novelty $\times$ tractability
\item Transforms raw flags into manageable investigation queue
\end{itemize}
\end{columns}
\bottomnote{Source: Pang et al.\ (2021); EU AI Act Art.\ 14; paper Section 3.6}
\end{frame}

% ----------------------------------------------------------
% SLIDE 24 -- Pipeline Architecture Chart
% ----------------------------------------------------------
\begin{frame}{Pipeline Architecture: Visual Summary}

\chartplaceholder[5.5cm]{Chart: 01\_pipeline\_architecture -- Full five-stage pipeline diagram with forward data flow, feedback loops (investigator feedback, drift-triggered retraining), and fraud-type annotations per stage}

\bottomnote{Source: Paper Figure 1; Section 3}
\end{frame}

% ----------------------------------------------------------
% SLIDE 25 -- Summary
% ----------------------------------------------------------
\begin{frame}{Summary: The Detection Pipeline (C1)}
\begin{enumerate}\compactlist
\item The \textbf{five-stage pipeline} (ingestion $\to$ features $\to$ models $\to$ explainability $\to$ deployment) provides the first hedge-fund-specific end-to-end framework
\item \textbf{Data ingestion} must solve temporal alignment, bias correction, entity resolution, and multi-source fusion
\item \textbf{Five feature families} (statistical, Benford, textual, network, temporal) capture complementary fraud signals across different data modalities
\item \textbf{Six model families} are matched to specific fraud types and data characteristics; tree-based ensembles currently dominate tabular detection (\fone{} $\sim 0.88$)
\item \textbf{Explainability} is not optional: EU AI Act mandates transparency for high-risk AI; \shap{}/\lime{}/attention provide structured explanations
\item \textbf{Deployment} requires batch+event processing, drift detection, human-in-the-loop feedback, and GRC integration
\item The pipeline is \textbf{not unidirectional}: feedback from deployment enables continuous adaptation
\end{enumerate}
\bottomnote{Source: Paper Section 3 -- Contribution C1}
\end{frame}

\end{document}
