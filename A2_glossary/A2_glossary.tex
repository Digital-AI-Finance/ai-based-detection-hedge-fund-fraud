\documentclass[8pt,aspectratio=169]{beamer}
\usetheme{Madrid}
\usepackage{graphicx}
\usepackage{booktabs}
\usepackage{adjustbox}
\usepackage{multicol}
\usepackage{amsmath}
\usepackage{amssymb}

\definecolor{mlblue}{RGB}{0,102,204}
\definecolor{mlpurple}{RGB}{51,51,178}
\definecolor{mllavender}{RGB}{173,173,224}
\definecolor{mllavender2}{RGB}{193,193,232}
\definecolor{mllavender3}{RGB}{204,204,235}
\definecolor{mllavender4}{RGB}{214,214,239}
\definecolor{mlorange}{RGB}{255,127,14}
\definecolor{mlgreen}{RGB}{44,160,44}
\definecolor{mlred}{RGB}{214,39,40}
\definecolor{mlgray}{RGB}{127,127,127}
\definecolor{lightgray}{RGB}{240,240,240}
\definecolor{midgray}{RGB}{180,180,180}

\setbeamercolor{palette primary}{bg=mllavender3,fg=mlpurple}
\setbeamercolor{palette secondary}{bg=mllavender2,fg=mlpurple}
\setbeamercolor{palette tertiary}{bg=mllavender,fg=white}
\setbeamercolor{palette quaternary}{bg=mlpurple,fg=white}
\setbeamercolor{structure}{fg=mlpurple}
\setbeamercolor{title}{fg=mlpurple}
\setbeamercolor{frametitle}{fg=mlpurple,bg=mllavender3}
\setbeamercolor{block title}{bg=mllavender2,fg=mlpurple}
\setbeamercolor{block body}{bg=mllavender4,fg=black}

\setbeamertemplate{navigation symbols}{}
\setbeamertemplate{itemize items}[circle]
\setbeamertemplate{enumerate items}[default]
\setbeamersize{text margin left=5mm,text margin right=5mm}

\newcommand{\bottomnote}[1]{\vfill\vspace{-2mm}\textcolor{mllavender2}{\rule{\textwidth}{0.4pt}}\vspace{1mm}\footnotesize\textbf{#1}}
\newcommand{\compactlist}{\setlength{\itemsep}{0pt}\setlength{\parskip}{0pt}\setlength{\parsep}{0pt}}
\newcommand{\chartplaceholder}[2][5cm]{\begin{center}\begin{adjustbox}{max width=0.95\textwidth, max height=#1}\framebox[\textwidth][c]{\rule{0pt}{#1}\textcolor{midgray}{[#2]}}\end{adjustbox}\end{center}}

% Shared notation macros for AI-Based Detection of Hedge Fund Fraud
% Include this file in all Beamer slide decks via % Shared notation macros for AI-Based Detection of Hedge Fund Fraud
% Include this file in all Beamer slide decks via % Shared notation macros for AI-Based Detection of Hedge Fund Fraud
% Include this file in all Beamer slide decks via \input{notation}

% Performance metrics
\newcommand{\auc}{\ensuremath{\mathrm{AUC}}}
\newcommand{\fone}{\ensuremath{F_1}}
\newcommand{\shap}{\text{SHAP}}
\newcommand{\lime}{\text{LIME}}

% Autocorrelation
\newcommand{\rhoone}{\ensuremath{\rho_1}}

% Financial abbreviations
\newcommand{\nav}{\text{NAV}}
\newcommand{\aum}{\text{AUM}}

% Regulatory
\newcommand{\euaiact}{EU AI Act}
\newcommand{\aifmd}{AIFMD}
\newcommand{\sec}{SEC}
\newcommand{\edgar}{EDGAR}

% Key numbers from the paper (for consistency)
\newcommand{\aumdollar}{\$4.5\text{ trillion}}
\newcommand{\aucdeg}{10.6\%}
\newcommand{\numop}{10}
\newcommand{\numfraudcases}{50\text{--}100}
\newcommand{\systematicpapers}{105}


% Performance metrics
\newcommand{\auc}{\ensuremath{\mathrm{AUC}}}
\newcommand{\fone}{\ensuremath{F_1}}
\newcommand{\shap}{\text{SHAP}}
\newcommand{\lime}{\text{LIME}}

% Autocorrelation
\newcommand{\rhoone}{\ensuremath{\rho_1}}

% Financial abbreviations
\newcommand{\nav}{\text{NAV}}
\newcommand{\aum}{\text{AUM}}

% Regulatory
\newcommand{\euaiact}{EU AI Act}
\newcommand{\aifmd}{AIFMD}
\newcommand{\sec}{SEC}
\newcommand{\edgar}{EDGAR}

% Key numbers from the paper (for consistency)
\newcommand{\aumdollar}{\$4.5\text{ trillion}}
\newcommand{\aucdeg}{10.6\%}
\newcommand{\numop}{10}
\newcommand{\numfraudcases}{50\text{--}100}
\newcommand{\systematicpapers}{105}


% Performance metrics
\newcommand{\auc}{\ensuremath{\mathrm{AUC}}}
\newcommand{\fone}{\ensuremath{F_1}}
\newcommand{\shap}{\text{SHAP}}
\newcommand{\lime}{\text{LIME}}

% Autocorrelation
\newcommand{\rhoone}{\ensuremath{\rho_1}}

% Financial abbreviations
\newcommand{\nav}{\text{NAV}}
\newcommand{\aum}{\text{AUM}}

% Regulatory
\newcommand{\euaiact}{EU AI Act}
\newcommand{\aifmd}{AIFMD}
\newcommand{\sec}{SEC}
\newcommand{\edgar}{EDGAR}

% Key numbers from the paper (for consistency)
\newcommand{\aumdollar}{\$4.5\text{ trillion}}
\newcommand{\aucdeg}{10.6\%}
\newcommand{\numop}{10}
\newcommand{\numfraudcases}{50\text{--}100}
\newcommand{\systematicpapers}{105}


\title{Appendix C: Glossary of Terms}
\subtitle{AI-Based Detection of Hedge Fund Fraud}
\author{Comprehensive Terminology Reference}
\date{}

\begin{document}

% ==================== SLIDE 1: Title ====================
\begin{frame}
\titlepage
\end{frame}

% ==================== SLIDE 2: ML/AI Terms (Part 1) ====================
\begin{frame}{Machine Learning and AI Terms (1/2)}

\begin{columns}[T]

\column{0.48\textwidth}

\textbf{\textcolor{mlpurple}{AUC (Area Under the Curve)}}
\begin{itemize}\compactlist
    \item Performance metric for classification models
    \item Area under the ROC curve
    \item Range: 0--1 (0.5 = random, 1.0 = perfect)
\end{itemize}

\vspace{2mm}

\textbf{\textcolor{mlpurple}{Autoencoder}}
\begin{itemize}\compactlist
    \item Neural network learning compressed representations
    \item Trains to reconstruct its input
    \item Used for anomaly detection via reconstruction error
\end{itemize}

\vspace{2mm}

\textbf{\textcolor{mlpurple}{BERT (Bidirectional Encoder Representations from Transformers)}}
\begin{itemize}\compactlist
    \item Transformer-based language model
    \item Generates contextualized word embeddings
    \item FinBERT: variant fine-tuned on financial text
\end{itemize}

\vspace{2mm}

\textbf{\textcolor{mlpurple}{Class Imbalance}}
\begin{itemize}\compactlist
    \item One class (fraud) substantially less frequent
    \item Requires specialized sampling or algorithms
    \item Critical challenge in fraud detection
\end{itemize}

\column{0.48\textwidth}

\textbf{\textcolor{mlpurple}{Concept Drift}}
\begin{itemize}\compactlist
    \item Statistical properties of target change over time
    \item Requires model retraining or adaptive learning
    \item Common in fraud detection due to adversarial behavior
\end{itemize}

\vspace{2mm}

\textbf{\textcolor{mlpurple}{DBSCAN (Density-Based Spatial Clustering)}}
\begin{itemize}\compactlist
    \item Unsupervised clustering algorithm
    \item Groups densely packed points
    \item Identifies outliers as potential anomalies
\end{itemize}

\vspace{2mm}

\textbf{\textcolor{mlpurple}{Ensemble Methods}}
\begin{itemize}\compactlist
    \item Combine multiple base models (e.g., decision trees)
    \item Improve predictive performance and robustness
    \item Examples: Random Forest, XGBoost
\end{itemize}

\vspace{2mm}

\textbf{\textcolor{mlpurple}{F1 Score}}
\begin{itemize}\compactlist
    \item Harmonic mean of precision and recall
    \item Formula: $F_1 = 2 \cdot \frac{\text{precision} \cdot \text{recall}}{\text{precision} + \text{recall}}$
    \item Particularly useful for imbalanced datasets
\end{itemize}

\end{columns}

\bottomnote{Key ML/AI concepts for understanding fraud detection methodologies.}
\end{frame}

% ==================== SLIDE 3: ML/AI Terms (Part 2) ====================
\begin{frame}{Machine Learning and AI Terms (2/2)}

\begin{columns}[T]

\column{0.48\textwidth}

\textbf{\textcolor{mlpurple}{GNN (Graph Neural Network)}}
\begin{itemize}\compactlist
    \item Deep learning for graph-structured data
    \item Captures node features and topology
    \item Subtypes: GCN, GAT
\end{itemize}

\vspace{2mm}

\textbf{\textcolor{mlpurple}{GCN (Graph Convolutional Network)}}
\begin{itemize}\compactlist
    \item Neural network operating on graphs
    \item Aggregates features from neighboring nodes
    \item Convolutional operations on graph structure
\end{itemize}

\vspace{2mm}

\textbf{\textcolor{mlpurple}{GAT (Graph Attention Network)}}
\begin{itemize}\compactlist
    \item GNN with attention mechanism
    \item Learns importance weights for neighbors
    \item Selective information aggregation
\end{itemize}

\vspace{2mm}

\textbf{\textcolor{mlpurple}{Gradient Boosting}}
\begin{itemize}\compactlist
    \item Ensemble method building models sequentially
    \item Each model corrects errors of previous ones
    \item Popular: XGBoost, LightGBM
\end{itemize}

\vspace{2mm}

\textbf{\textcolor{mlpurple}{Isolation Forest}}
\begin{itemize}\compactlist
    \item Anomaly detection algorithm
    \item Identifies outliers via ease of isolation
    \item Uses randomly constructed decision trees
\end{itemize}

\column{0.48\textwidth}

\textbf{\textcolor{mlpurple}{LSTM (Long Short-Term Memory)}}
\begin{itemize}\compactlist
    \item Recurrent neural network architecture
    \item Gating mechanisms for long-range dependencies
    \item Used for time series analysis
\end{itemize}

\vspace{2mm}

\textbf{\textcolor{mlpurple}{Random Forest}}
\begin{itemize}\compactlist
    \item Ensemble of multiple decision trees
    \item Outputs mode (classification) or mean (regression)
    \item Robust to overfitting
\end{itemize}

\vspace{2mm}

\textbf{\textcolor{mlpurple}{XGBoost (eXtreme Gradient Boosting)}}
\begin{itemize}\compactlist
    \item Optimized gradient boosting implementation
    \item Regularization, parallel processing
    \item Widely used in fraud detection competitions
\end{itemize}

\vspace{2mm}

\textbf{\textcolor{mlpurple}{SHAP (SHapley Additive exPlanations)}}
\begin{itemize}\compactlist
    \item Explainability framework from game theory
    \item Assigns feature importance for predictions
    \item Model-agnostic approach
\end{itemize}

\vspace{2mm}

\textbf{\textcolor{mlpurple}{LIME (Local Interpretable Model-Agnostic Explanations)}}
\begin{itemize}\compactlist
    \item Local approximation of complex models
    \item Uses interpretable linear models
    \item Explains individual predictions
\end{itemize}

\end{columns}

\bottomnote{Advanced ML/AI techniques for fraud detection and model explainability.}
\end{frame}

% ==================== SLIDE 4: Financial and Regulatory Terms ====================
\begin{frame}{Financial and Regulatory Terms}

\begin{columns}[T]

\column{0.48\textwidth}

\textbf{\textcolor{mlpurple}{Hedge Fund}}
\begin{itemize}\compactlist
    \item Pooled investment vehicle with diverse strategies
    \item Long/short equity, global macro, event-driven
    \item Limited regulatory oversight and investor restrictions
\end{itemize}

\vspace{2mm}

\textbf{\textcolor{mlpurple}{NAV (Net Asset Value)}}
\begin{itemize}\compactlist
    \item Total assets minus liabilities
    \item Typically calculated per share
    \item Primary metric reported to investors
\end{itemize}

\vspace{2mm}

\textbf{\textcolor{mlpurple}{Ponzi Scheme}}
\begin{itemize}\compactlist
    \item Fraudulent operation paying returns from new capital
    \item Not from legitimate profits
    \item Collapses when new investments slow
\end{itemize}

\vspace{2mm}

\textbf{\textcolor{mlpurple}{Serial Correlation}}
\begin{itemize}\compactlist
    \item Correlation at different time lags
    \item High serial correlation may indicate return smoothing
    \item Fraud detection red flag
\end{itemize}

\vspace{2mm}

\textbf{\textcolor{mlpurple}{Survivorship Bias}}
\begin{itemize}\compactlist
    \item Failed/closed funds excluded from databases
    \item Leads to overestimation of average returns
    \item Critical data quality issue
\end{itemize}

\column{0.48\textwidth}

\textbf{\textcolor{mlpurple}{Backfill Bias}}
\begin{itemize}\compactlist
    \item Artificial inflation of historical performance
    \item Funds report past returns after track record
    \item Selective reporting problem
\end{itemize}

\vspace{2mm}

\textbf{\textcolor{mlpurple}{Benford's Law}}
\begin{itemize}\compactlist
    \item Leading digits follow logarithmic distribution
    \item ``1'' appears as first digit ~30\% of time
    \item Deviations suggest manipulation
\end{itemize}

\vspace{2mm}

\textbf{\textcolor{mlpurple}{AIFMD (Alternative Investment Fund Managers Directive)}}
\begin{itemize}\compactlist
    \item European Union regulation
    \item Registration, disclosure, oversight requirements
    \item Applies to hedge funds and private equity
\end{itemize}

\vspace{2mm}

\textbf{\textcolor{mlpurple}{Dodd-Frank Act}}
\begin{itemize}\compactlist
    \item U.S. financial reform legislation (2010)
    \item Hedge fund registration with SEC
    \item Periodic disclosure of positions and risk metrics
\end{itemize}

\vspace{2mm}

\textbf{\textcolor{mlpurple}{Form ADV}}
\begin{itemize}\compactlist
    \item SEC registration form for investment advisers
    \item Business practices, fees, conflicts, discipline
    \item Primary disclosure document in U.S.
\end{itemize}

\end{columns}

\bottomnote{Essential financial and regulatory concepts for hedge fund fraud detection.}
\end{frame}

% ==================== SLIDE 5: Adversarial ML and Explainability ====================
\begin{frame}{Adversarial ML and Explainability Terms}

\begin{columns}[T]

\column{0.48\textwidth}

\begin{block}{Adversarial Machine Learning}
Techniques for robust models in adversarial environments where fraudsters actively evade detection.
\end{block}

\vspace{2mm}

\textbf{\textcolor{mlpurple}{Adversarial Training}}
\begin{itemize}\compactlist
    \item Train on genuine and perturbed examples
    \item Improves robustness against attacks
    \item Critical for fraud detection systems
\end{itemize}

\vspace{2mm}

\textbf{\textcolor{mlpurple}{Data Poisoning}}
\begin{itemize}\compactlist
    \item Inject malicious data into training set
    \item Corrupts model learning
    \item Supply chain attack on ML systems
\end{itemize}

\vspace{2mm}

\textbf{\textcolor{mlpurple}{Evasion Attack}}
\begin{itemize}\compactlist
    \item Fraudsters modify behavior to avoid detection
    \item Exploit knowledge of detection model
    \item Primary threat in deployed systems
\end{itemize}

\vspace{2mm}

\textbf{\textcolor{mlpurple}{FGSM (Fast Gradient Sign Method)}}
\begin{itemize}\compactlist
    \item Adversarial example generation technique
    \item Uses gradient information
    \item Tests model robustness
\end{itemize}

\column{0.48\textwidth}

\begin{block}{Explainability and Interpretability}
Techniques to understand and explain model predictions, critical for regulatory compliance.
\end{block}

\vspace{2mm}

\textbf{\textcolor{mlpurple}{SHAP (Revisited)}}
\begin{itemize}\compactlist
    \item Game-theoretic approach to feature importance
    \item Consistent, locally accurate explanations
    \item Regulatory-friendly interpretability
\end{itemize}

\vspace{2mm}

\textbf{\textcolor{mlpurple}{LIME (Revisited)}}
\begin{itemize}\compactlist
    \item Local linear approximations
    \item Explains individual predictions
    \item Model-agnostic approach
\end{itemize}

\vspace{2mm}

\textbf{\textcolor{mlpurple}{SupTech (Supervisory Technology)}}
\begin{itemize}\compactlist
    \item Technology-based regulatory solutions
    \item Data collection, risk assessment, surveillance
    \item Enables scalable regulatory oversight
\end{itemize}

\vspace{2mm}

\textbf{\textcolor{mlpurple}{EU AI Act}}
\begin{itemize}\compactlist
    \item Regulation of high-risk AI systems
    \item Transparency and explainability requirements
    \item Applies to fraud detection in finance
\end{itemize}

\end{columns}

\bottomnote{Adversarial robustness and explainability are critical for deployed fraud detection systems.}
\end{frame}

% ==================== SLIDE 6: References and Resources ====================
\begin{frame}{References and Further Resources}

\begin{block}{Key Reference Sources}
This glossary synthesizes terminology from academic literature, regulatory documents, and industry practice.
\end{block}

\vspace{2mm}

\begin{columns}[T]

\column{0.48\textwidth}

\textbf{Academic Literature:}
\begin{itemize}\compactlist
    \item Getmansky et al. (2004) - Serial correlation
    \item Brown et al. (2009) - Statistical features
    \item Bollen \& Pool (2009) - Discontinuity at zero
    \item Amiram et al. (2015) - Benford's law
    \item Zhang et al. (2022) - Graph neural networks
    \item Chen et al. (2023) - FinBERT applications
\end{itemize}

\vspace{3mm}

\textbf{Regulatory Sources:}
\begin{itemize}\compactlist
    \item SEC (Securities and Exchange Commission)
    \item ESMA (European Securities and Markets Authority)
    \item FINRA (Financial Industry Regulatory Authority)
    \item FSA (Financial Services Authority, UK)
\end{itemize}

\column{0.48\textwidth}

\textbf{Technical Resources:}
\begin{itemize}\compactlist
    \item Goodfellow et al. (2016) - Deep Learning textbook
    \item Bishop (2006) - Pattern Recognition and ML
    \item Murphy (2022) - Probabilistic Machine Learning
    \item Lundberg \& Lee (2017) - SHAP paper
    \item Ribeiro et al. (2016) - LIME paper
\end{itemize}

\vspace{3mm}

\textbf{Industry Standards:}
\begin{itemize}\compactlist
    \item Alternative Investment Management Association (AIMA)
    \item CFA Institute standards
    \item International Organization of Securities Commissions (IOSCO)
\end{itemize}

\vspace{3mm}

\textbf{Additional Context:}
\begin{itemize}\compactlist
    \item Full reference list in main paper
    \item Appendices A \& B provide methodological details
    \item See main body for applications and case studies
\end{itemize}

\end{columns}

\bottomnote{Comprehensive terminology enables clear communication across AI, finance, and regulatory domains.}
\end{frame}

\end{document}
