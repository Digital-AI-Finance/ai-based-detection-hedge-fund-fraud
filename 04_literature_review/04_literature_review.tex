% ============================================================
%  Slide Deck 4 -- Review of AI-Based Detection Methods
%  AI-Based Detection of Hedge Fund Fraud
% ============================================================
\documentclass[8pt,aspectratio=169]{beamer}
\usetheme{Madrid}
\usepackage{graphicx}
\usepackage{booktabs}
\usepackage{adjustbox}
\usepackage{multicol}
\usepackage{amsmath}
\usepackage{amssymb}

% ---- Color definitions ----
\definecolor{mlblue}{RGB}{0,102,204}
\definecolor{mlpurple}{RGB}{51,51,178}
\definecolor{mllavender}{RGB}{173,173,224}
\definecolor{mllavender2}{RGB}{193,193,232}
\definecolor{mllavender3}{RGB}{204,204,235}
\definecolor{mllavender4}{RGB}{214,214,239}
\definecolor{mlorange}{RGB}{255, 127, 14}
\definecolor{mlgreen}{RGB}{44, 160, 44}
\definecolor{mlred}{RGB}{214, 39, 40}
\definecolor{mlgray}{RGB}{127, 127, 127}
\definecolor{lightgray}{RGB}{240, 240, 240}
\definecolor{midgray}{RGB}{180, 180, 180}

% ---- Apply custom colors to Madrid theme ----
\setbeamercolor{palette primary}{bg=mllavender3,fg=mlpurple}
\setbeamercolor{palette secondary}{bg=mllavender2,fg=mlpurple}
\setbeamercolor{palette tertiary}{bg=mllavender,fg=white}
\setbeamercolor{palette quaternary}{bg=mlpurple,fg=white}
\setbeamercolor{structure}{fg=mlpurple}
\setbeamercolor{section in toc}{fg=mlpurple}
\setbeamercolor{subsection in toc}{fg=mlblue}
\setbeamercolor{title}{fg=mlpurple}
\setbeamercolor{frametitle}{fg=mlpurple,bg=mllavender3}
\setbeamercolor{block title}{bg=mllavender2,fg=mlpurple}
\setbeamercolor{block body}{bg=mllavender4,fg=black}

% ---- Navigation / itemize ----
\setbeamertemplate{navigation symbols}{}
\setbeamertemplate{itemize items}[circle]
\setbeamertemplate{enumerate items}[default]
\setbeamersize{text margin left=5mm,text margin right=5mm}

% ---- Custom commands ----
\newcommand{\bottomnote}[1]{%
\vfill
\vspace{-2mm}
\textcolor{mllavender2}{\rule{\textwidth}{0.4pt}}
\vspace{1mm}
\footnotesize
\textbf{#1}
}

\newcommand{\compactlist}{%
\setlength{\itemsep}{0pt}%
\setlength{\parskip}{0pt}%
\setlength{\parsep}{0pt}%
}

\newcommand{\chartplaceholder}[2][5cm]{%
\begin{center}
\begin{adjustbox}{max width=0.95\textwidth, max height=#1}
\framebox[\textwidth][c]{%
\rule{0pt}{#1}%
\textcolor{midgray}{[#2]}%
}
\end{adjustbox}
\end{center}
}

% ---- Notation ----
% Shared notation macros for AI-Based Detection of Hedge Fund Fraud
% Include this file in all Beamer slide decks via % Shared notation macros for AI-Based Detection of Hedge Fund Fraud
% Include this file in all Beamer slide decks via % Shared notation macros for AI-Based Detection of Hedge Fund Fraud
% Include this file in all Beamer slide decks via \input{notation}

% Performance metrics
\newcommand{\auc}{\ensuremath{\mathrm{AUC}}}
\newcommand{\fone}{\ensuremath{F_1}}
\newcommand{\shap}{\text{SHAP}}
\newcommand{\lime}{\text{LIME}}

% Autocorrelation
\newcommand{\rhoone}{\ensuremath{\rho_1}}

% Financial abbreviations
\newcommand{\nav}{\text{NAV}}
\newcommand{\aum}{\text{AUM}}

% Regulatory
\newcommand{\euaiact}{EU AI Act}
\newcommand{\aifmd}{AIFMD}
\newcommand{\sec}{SEC}
\newcommand{\edgar}{EDGAR}

% Key numbers from the paper (for consistency)
\newcommand{\aumdollar}{\$4.5\text{ trillion}}
\newcommand{\aucdeg}{10.6\%}
\newcommand{\numop}{10}
\newcommand{\numfraudcases}{50\text{--}100}
\newcommand{\systematicpapers}{105}


% Performance metrics
\newcommand{\auc}{\ensuremath{\mathrm{AUC}}}
\newcommand{\fone}{\ensuremath{F_1}}
\newcommand{\shap}{\text{SHAP}}
\newcommand{\lime}{\text{LIME}}

% Autocorrelation
\newcommand{\rhoone}{\ensuremath{\rho_1}}

% Financial abbreviations
\newcommand{\nav}{\text{NAV}}
\newcommand{\aum}{\text{AUM}}

% Regulatory
\newcommand{\euaiact}{EU AI Act}
\newcommand{\aifmd}{AIFMD}
\newcommand{\sec}{SEC}
\newcommand{\edgar}{EDGAR}

% Key numbers from the paper (for consistency)
\newcommand{\aumdollar}{\$4.5\text{ trillion}}
\newcommand{\aucdeg}{10.6\%}
\newcommand{\numop}{10}
\newcommand{\numfraudcases}{50\text{--}100}
\newcommand{\systematicpapers}{105}


% Performance metrics
\newcommand{\auc}{\ensuremath{\mathrm{AUC}}}
\newcommand{\fone}{\ensuremath{F_1}}
\newcommand{\shap}{\text{SHAP}}
\newcommand{\lime}{\text{LIME}}

% Autocorrelation
\newcommand{\rhoone}{\ensuremath{\rho_1}}

% Financial abbreviations
\newcommand{\nav}{\text{NAV}}
\newcommand{\aum}{\text{AUM}}

% Regulatory
\newcommand{\euaiact}{EU AI Act}
\newcommand{\aifmd}{AIFMD}
\newcommand{\sec}{SEC}
\newcommand{\edgar}{EDGAR}

% Key numbers from the paper (for consistency)
\newcommand{\aumdollar}{\$4.5\text{ trillion}}
\newcommand{\aucdeg}{10.6\%}
\newcommand{\numop}{10}
\newcommand{\numfraudcases}{50\text{--}100}
\newcommand{\systematicpapers}{105}


% ---- Title metadata ----
\title{Review of AI-Based Detection Methods}
\subtitle{Section 4 -- AI-Based Detection of Hedge Fund Fraud}
\author{Joerg Osterrieder}
\institute{Zurich University of Applied Sciences (ZHAW)}
\date{2025}

% ============================================================
\begin{document}

% ----------------------------------------------------------
% SLIDE 1 -- Title
% ----------------------------------------------------------
\begin{frame}
\titlepage
\end{frame}

% ----------------------------------------------------------
% SLIDE 2 -- Outline
% ----------------------------------------------------------
\begin{frame}{Outline}
\begin{enumerate}\compactlist
\item Classical Statistical Approaches
\item Tree-Based and Ensemble Methods
\item Why Trees Dominate Hedge Fund Detection
\item Deep Learning Approaches
\item Deep Learning Limitations
\item NLP for Financial Filings
\item Graph Neural Networks for Fund Networks
\item Semi-Supervised and Self-Supervised Methods
\item Synthetic Data and Data Augmentation
\item Critical Assessment of the Literature
\item Summary: Field at Early Stage
\end{enumerate}
\end{frame}

% ----------------------------------------------------------
% SLIDE 3 -- Classical Statistical
% ----------------------------------------------------------
\begin{frame}{Classical Statistical and Rule-Based Approaches}
\begin{columns}[T]
\column{0.48\textwidth}
\textbf{Benford's Law (Nigrini, 2012)}
\begin{itemize}\compactlist
\item Applied retrospectively to Madoff: anomalies in \textbf{9/10 tests}
\item Fabricators are poor intuitive generators of logarithmic digit distributions
\item Requires large samples; limited power for $<120$--$240$ monthly observations
\item \textcolor{mlred}{Aware fraudsters can engineer conformity}
\end{itemize}
\vspace{2mm}
\textbf{Serial Correlation (Getmansky et al., 2004)}
\begin{itemize}\compactlist
\item MA($k$) model: smoothing component from managed pricing
\item 30--40\% of TASS funds show significant positive $\rhoone$
\item Smoothing parameters now standard ML features
\end{itemize}

\column{0.48\textwidth}
\textbf{Bollen \& Pool (2012) -- Distributional ``Kink''}
\begin{itemize}\compactlist
\item Discontinuity at zero: excess small positives, deficit of small negatives
\item Correctly identified $\sim$\textbf{50\%} of funds subsequently facing SEC enforcement
\end{itemize}
\vspace{2mm}
\textbf{Operational Risk / Filing-Based}
\begin{itemize}\compactlist
\item Brown et al.\ (2008): \textbf{omega-score} from Form ADV data -- governance and organizational risk
\item Dimmock \& Gerken (2012): logistic regression on SEC filing data -- past violations, ownership, custody $\Rightarrow$ \auc{} $\approx 0.65$--$0.70$
\end{itemize}
\vspace{2mm}
\textcolor{mlblue}{\textbf{Collective limitation:} each method detects one specific signature of one fraud type. High false positive rates when deployed independently.}
\end{columns}
\bottomnote{Source: Paper Section 4.1}
\end{frame}

% ----------------------------------------------------------
% SLIDE 4 -- Tree-Based / Ensemble
% ----------------------------------------------------------
\begin{frame}{Tree-Based and Ensemble Methods}
\begin{columns}[T]
\column{0.48\textwidth}
\textbf{Random Forests}
\begin{itemize}\compactlist
\item Aggregate hundreds of independently grown trees
\item Robust to overfitting, tolerant of missing data
\item Handle mixed feature types (numerical + categorical) without standardization
\end{itemize}
\vspace{2mm}
\textbf{Gradient Boosting}
\begin{itemize}\compactlist
\item XGBoost, LightGBM, CatBoost
\item Sequential tree construction correcting residuals
\item Built-in class imbalance handling: weighting, stratified subsampling, cost-sensitive learning
\end{itemize}

\column{0.48\textwidth}
\textbf{Key Results}
\begin{itemize}\compactlist
\item \textbf{Bao et al.\ (2020)}: RUSBoost on 28,000+ firm-years linked to AAERs
  \begin{itemize}\compactlist
  \item \auc{} = \textbf{0.725}, outperforming logistic regression
  \item Accounting fraud, but methodology transferable
  \end{itemize}
\item \textbf{Stacking ensembles} (Hilal et al., 2022):
  \begin{itemize}\compactlist
  \item XGBoost + LightGBM + CatBoost via meta-learner
  \item \fone{} $\approx$ \textbf{0.88} -- highest among individual method families
  \end{itemize}
\end{itemize}
\vspace{2mm}
\textbf{Principal Limitation}
\begin{itemize}\compactlist
\item Relies on supervised training: only \numfraudcases{} confirmed hedge fund fraud cases
\item Positive class is \textit{heterogeneous}: Ponzi $\neq$ NAV manipulation $\neq$ style drift
\end{itemize}
\end{columns}
\bottomnote{Source: Bao et al.\ (2020); Hilal et al.\ (2022); paper Section 4.2}
\end{frame}

% ----------------------------------------------------------
% SLIDE 5 -- Why Trees Dominate
% ----------------------------------------------------------
\begin{frame}{Why Trees Dominate Hedge Fund Detection}
\begin{columns}[T]
\column{0.48\textwidth}
\textbf{\shap{} Compatibility}
\begin{itemize}\compactlist
\item Exact \shap{} computation for tree-based models (Lundberg \& Lee, 2017)
\item Per-feature attribution: ``flagged because $\rhoone$ is 2.3$\sigma$ above peer median, auditor has 2 prior sanctions, readability deteriorating''
\item Maps directly to SEC investigative categories
\item Satisfies EU AI Act transparency requirements
\end{itemize}
\vspace{2mm}
\textbf{Class Imbalance Handling}
\begin{itemize}\compactlist
\item Native mechanisms: sample weighting, cost-sensitive learning
\item RUSBoost: random undersampling + AdaBoost
\end{itemize}

\column{0.48\textwidth}
\textbf{Mixed Feature Types}
\begin{itemize}\compactlist
\item Numerical (return statistics) + categorical (strategy class, auditor ID, jurisdiction)
\item No feature standardization required
\item CatBoost: native categorical support
\end{itemize}
\vspace{2mm}
\textbf{Robustness}
\begin{itemize}\compactlist
\item Tolerant of outliers and missing values
\item Stable performance across varying hyperparameters (vs.\ deep learning sensitivity)
\item Can ingest full concatenated 5-family feature vector without dimensionality reduction
\end{itemize}
\vspace{2mm}
\textcolor{mlblue}{Trees provide the best trade-off between \textbf{performance, interpretability, and practical robustness} for current hedge fund data conditions.}
\end{columns}
\bottomnote{Source: Lundberg \& Lee (2017); paper Section 4.2}
\end{frame}

% ----------------------------------------------------------
% SLIDE 6 -- Deep Learning
% ----------------------------------------------------------
\begin{frame}{Deep Learning Approaches}
\begin{columns}[T]
\column{0.48\textwidth}
\textbf{LSTM Networks}
\begin{itemize}\compactlist
\item Gated memory cells: long-range sequential dependencies
\item Detect gradual shifts in serial correlation, regime-dependent anomalies
\item Fraud escalates over months/years: sequential nature matches
\end{itemize}
\vspace{2mm}
\textbf{CNN via Gramian Angular Fields}
\begin{itemize}\compactlist
\item Encode returns as 2D images (Gramian, recurrence plots)
\item Spatial pattern recognition on visual representations
\item Hybrid CNN-LSTM: local features + temporal aggregation
\end{itemize}

\column{0.48\textwidth}
\textbf{Transformers}
\begin{itemize}\compactlist
\item Self-attention across arbitrary sequence positions
\item Long-range dependencies without vanishing gradients
\item Link suspicious patterns across years of sparse monthly data
\item Attention weights $\Rightarrow$ intrinsic explainability
\end{itemize}
\vspace{2mm}
\textbf{Autoencoders}
\begin{itemize}\compactlist
\item Most directly applicable under label scarcity
\item Trained on normal behavior; high reconstruction error = anomaly
\item \auc{} $\approx$ \textbf{0.79} on hedge fund returns (Chalapathy \& Chawla, 2019)
\item No fraud labels needed; sidesteps data poisoning vulnerability
\end{itemize}
\end{columns}
\bottomnote{Source: Hochreiter \& Schmidhuber (1997); Chalapathy \& Chawla (2019); paper Section 4.3}
\end{frame}

% ----------------------------------------------------------
% SLIDE 7 -- Deep Learning Limitations
% ----------------------------------------------------------
\begin{frame}{Deep Learning Limitations in Hedge Fund Context}
\begin{columns}[T]
\column{0.48\textwidth}
\textbf{Data Scarcity}
\begin{itemize}\compactlist
\item Modern architectures need thousands to millions of examples
\item Hedge fund universe: $\sim$10,000--15,000 funds
\item At most 120--240 monthly observations per fund
\item \textcolor{mlred}{Fewer than 100 positive (fraud) examples}
\end{itemize}
\vspace{2mm}
\textbf{Overfitting Risk}
\begin{itemize}\compactlist
\item Models with millions of parameters trained on $<100$ positives
\item May memorize specific fraud fingerprints rather than learn generalizable patterns
\item Common practice: single train-test split (inadequate)
\end{itemize}

\column{0.48\textwidth}
\textbf{Opacity / Explainability}
\begin{itemize}\compactlist
\item Predictions resist human interpretation
\item EU AI Act mandates transparency for high-risk AI
\item Post-hoc methods (\shap{}, \lime{}) add complexity but do not fully resolve
\end{itemize}
\vspace{2mm}
\textbf{Hyperparameter Sensitivity}
\begin{itemize}\compactlist
\item Small changes in architecture, learning rate, regularization $\Rightarrow$ qualitatively different results
\item Undermines reproducibility and reliability of performance claims
\end{itemize}
\vspace{3mm}
\textcolor{mlblue}{\textbf{Verdict:} deep learning has theoretical appeal but faces severe practical constraints in the current data regime. Best used in combination with tree-based methods or as anomaly detectors (autoencoders).}
\end{columns}
\bottomnote{Source: Paper Section 4.3}
\end{frame}

% ----------------------------------------------------------
% SLIDE 8 -- NLP
% ----------------------------------------------------------
\begin{frame}{NLP for Financial Filings}
\begin{columns}[T]
\column{0.48\textwidth}
\textbf{Domain-Specific Lexicons}
\begin{itemize}\compactlist
\item Loughran \& McDonald (2011): general sentiment lexicons perform poorly on financial text
\item ``Liability,'' ``tax,'' ``depreciation'' = negative in general, neutral in finance
\item Financial-domain dictionary: more accurate sentiment classification
\end{itemize}
\vspace{2mm}
\textbf{Transformer Models}
\begin{itemize}\compactlist
\item \textbf{FinBERT}: 87\% accuracy on financial sentiment
\item \textbf{SEC-BERT}: pre-trained on EDGAR filings; improved NER and document classification for regulatory text
\end{itemize}

\column{0.48\textwidth}
\textbf{Detection Applications}
\begin{itemize}\compactlist
\item Vague/evasive language in strategy descriptions
\item Changes in filing complexity over time $\to$ associated with subsequent regulatory action
\item Boilerplate deviation: unusual departure from peer templates
\item Cross-modal: text strategy vs.\ quantitative factor exposures
\end{itemize}
\vspace{2mm}
\textbf{Multi-Modal Fusion}
\begin{itemize}\compactlist
\item NLP + quantitative returns: \textbf{+3--5\% \auc{}} (Ahmed et al., 2024)
\item Logic: text claims conservative equity L/S but returns load on leveraged distressed credit $\Rightarrow$ stronger signal
\end{itemize}
\vspace{2mm}
\textbf{Limitation}: filings are heavily boilerplate; low signal-to-noise ratio; fraudsters use compliance counsel to match expected patterns
\end{columns}
\bottomnote{Source: Loughran \& McDonald (2011); Araci (2019); Ahmed et al.\ (2024); paper Section 4.4}
\end{frame}

% ----------------------------------------------------------
% SLIDE 9 -- GNN
% ----------------------------------------------------------
\begin{frame}{Graph Neural Networks for Fund Networks}
\begin{columns}[T]
\column{0.48\textwidth}
\textbf{Architectures}
\begin{itemize}\compactlist
\item \textbf{GCN} (Kipf \& Welling, 2017): spectral convolutions, neighborhood aggregation
\item \textbf{GAT} (Velickovic et al., 2018): attention-weighted neighbor contributions
\item \textbf{GraphSAGE} (Hamilton et al., 2017): inductive learning -- score new funds without retraining (cold-start)
\item \textbf{Temporal knowledge graphs}: time-stamped edges for evolving relationships
\end{itemize}
\vspace{2mm}
\textbf{Results from Adjacent Domains}
\begin{itemize}\compactlist
\item Wang et al.\ (2019): semi-supervised GNN on transaction networks -- \auc{} = \textbf{0.87}
\item Liu et al.\ (2021): ``camouflage'' problem -- fraud detection even when immediate neighborhood appears benign
\end{itemize}

\column{0.48\textwidth}
\textbf{Hedge Fund Application}
\begin{itemize}\compactlist
\item Service provider network: auditor, administrator, custodian, prime broker
\item Small/unregistered auditors, lack of independent administrators $\Rightarrow$ elevated risk
\item Dynamic signals: sudden auditor change, administrator linked to multiple sanctioned funds, manager ``phoenix'' pattern
\end{itemize}
\vspace{2mm}
\textbf{Strengths}
\begin{itemize}\compactlist
\item Captures relational info inaccessible to tabular methods
\item ``Guilt by association,'' network centrality, structural equivalence
\end{itemize}
\vspace{2mm}
\textbf{Limitations}
\begin{itemize}\compactlist
\item Graph construction requires entity resolution (largely unaddressed)
\item Incomplete relationship data
\item Computational cost scales with graph size
\end{itemize}
\end{columns}
\bottomnote{Source: Wang et al.\ (2019); Liu et al.\ (2021); paper Section 4.5}
\end{frame}

% ----------------------------------------------------------
% SLIDE 10 -- Semi-Supervised
% ----------------------------------------------------------
\begin{frame}{Semi-Supervised and Self-Supervised Methods}
\begin{columns}[T]
\column{0.48\textwidth}
\textbf{Label Propagation / Self-Training}
\begin{itemize}\compactlist
\item Extend sparse labels through the data manifold
\item Effective when $<5\%$ of data carry labels (characteristic of hedge fund context)
\item Self-training: iteratively label most confident predictions, retrain
\end{itemize}
\vspace{2mm}
\textbf{Contrastive Learning}
\begin{itemize}\compactlist
\item Learn representations maximizing agreement between augmented views of same fund
\item Minimize agreement between different funds
\item Separate normal from anomalous without explicit fraud labels
\item Downstream classification with very few labeled examples
\end{itemize}

\column{0.48\textwidth}
\textbf{Self-Supervised Pre-Training}
\begin{itemize}\compactlist
\item Objectives on unlabeled returns:
  \begin{itemize}\compactlist
  \item Masked return prediction
  \item Temporal order prediction
  \item Next-period forecasting
  \end{itemize}
\item Creates general-purpose fund behavior representations
\item Fine-tune for fraud with minimal labels (pre-train $\to$ fine-tune paradigm)
\end{itemize}
\vspace{2mm}
\textbf{Transfer Learning}
\begin{itemize}\compactlist
\item Adapt models from banking/insurance/accounting fraud
\item Degree of cross-domain transferability = open empirical question
\end{itemize}
\vspace{2mm}
\textbf{Limitation}: sensitive to distributional shifts; labeled examples may not represent full fraud population
\end{columns}
\bottomnote{Source: Pang et al.\ (2021); paper Section 4.6}
\end{frame}

% ----------------------------------------------------------
% SLIDE 11 -- Synthetic Data
% ----------------------------------------------------------
\begin{frame}{Synthetic Data and Data Augmentation}
\begin{columns}[T]
\column{0.48\textwidth}
\textbf{SMOTE and Variants}
\begin{itemize}\compactlist
\item Most widely used: interpolation between existing positives in feature space
\item \textcolor{mlred}{Problem}: interpolating between a Ponzi scheme and a valuation fraud generates implausible patterns
\item Borderline-SMOTE, ADASYN: concentrate near decision boundary (partial fix)
\end{itemize}
\vspace{2mm}
\textbf{Conditional GANs}
\begin{itemize}\compactlist
\item Condition on fraud type or attributes
\item Capture complex dependencies (returns $\times$ filings $\times$ operations)
\item More realistic than interpolation
\end{itemize}

\column{0.48\textwidth}
\textbf{VAEs}
\begin{itemize}\compactlist
\item Better-calibrated uncertainty estimates
\item Advantageous when confidence of generated examples matters
\end{itemize}
\vspace{2mm}
\textbf{Validation Circularity}
\begin{itemize}\compactlist
\item Generator learns to resemble \textit{known} fraud
\item If known examples are not representative $\Rightarrow$ synthetic data perpetuates biases
\item Ensuring realism \textit{without} assuming we know what fraud looks like: fundamental challenge
\end{itemize}
\vspace{2mm}
\textbf{Synthetic Benchmarks}
\begin{itemize}\compactlist
\item Calibrated simulation (Fiore et al., 2019)
\item Avoids confidentiality constraints
\item Hedge-fund-specific benchmark = open priority (OP1)
\end{itemize}
\end{columns}
\bottomnote{Source: Chawla et al.\ (2002); Fiore et al.\ (2019); paper Section 4.7}
\end{frame}

% ----------------------------------------------------------
% SLIDE 12 -- Method Families Chart
% ----------------------------------------------------------
\begin{frame}{Method Families: Comparative Overview}

\chartplaceholder[5.5cm]{Chart: 01\_method\_families -- Comparative visualization of method families: classical, tree-based, deep learning, NLP, GNN, semi-supervised, synthetic -- showing performance ranges, data requirements, interpretability, and maturity in hedge fund context}

\bottomnote{Source: Paper Section 4}
\end{frame}

% ----------------------------------------------------------
% SLIDE 13 -- Critical Assessment: Reproducibility
% ----------------------------------------------------------
\begin{frame}{Critical Assessment: Reproducibility Crisis}
\begin{columns}[T]
\column{0.48\textwidth}
\textbf{Proprietary Data}
\begin{itemize}\compactlist
\item Majority of studies use \textbf{proprietary datasets}: licensed databases, internal regulatory records, bespoke compilations
\item Results cannot be independently verified
\item Performance metrics must be \textbf{taken on trust}
\item Contrasts sharply with credit card fraud detection: public benchmarks enable rigorous comparison
\end{itemize}

\column{0.48\textwidth}
\textbf{No Standard Benchmark}
\begin{itemize}\compactlist
\item Each study assembles its own data, defines its own fraud labels
\item Reports results on non-overlapping fund populations
\item \textcolor{mlred}{Cross-study comparison is effectively impossible}
\item An \auc{} of 0.79 in one study cannot be compared to 0.72 in another when:
  \begin{itemize}\compactlist
  \item Datasets differ
  \item Label definitions differ
  \item Feature sets differ
  \item Evaluation protocols differ
  \end{itemize}
\end{itemize}
\end{columns}
\bottomnote{Source: Paper Section 4.8}
\end{frame}

% ----------------------------------------------------------
% SLIDE 14 -- Critical Assessment: Benchmarks, Domain
% ----------------------------------------------------------
\begin{frame}{Critical Assessment: Benchmark Gap and Domain Specificity}
\begin{columns}[T]
\column{0.48\textwidth}
\textbf{Domain Specificity}
\begin{itemize}\compactlist
\item Most impressive claims (\fone{} $> 0.85$, \auc{} $> 0.90$) originate from \textbf{adjacent domains}:
  \begin{itemize}\compactlist
  \item Credit card fraud
  \item Payment fraud
  \item Banking fraud
  \end{itemize}
\item These domains have abundant labeled data and well-characterized fraud
\item \textcolor{mlred}{Transferability to hedge funds is uncertain:}
  \begin{itemize}\compactlist
  \item Sparse data
  \item Heterogeneous fraud types
  \item Sophisticated adversaries
  \end{itemize}
\end{itemize}

\column{0.48\textwidth}
\textbf{Class Imbalance Handling}
\begin{itemize}\compactlist
\item Treatment varies enormously across studies
\item Often inadequately reported:
  \begin{itemize}\compactlist
  \item SMOTE applied without impact evaluation
  \item Ad hoc cost ratios for cost-sensitive learning
  \item Some report only \textbf{accuracy} -- meaningless under severe imbalance (classifying all as non-fraud $>$ 97\%)
  \end{itemize}
\item No standardized protocols for handling \textit{or} reporting class imbalance
\end{itemize}
\vspace{2mm}
\textcolor{mlblue}{\textbf{Studies reporting high performance on general financial fraud benchmarks likely overestimate effectiveness for hedge fund surveillance.}}
\end{columns}
\bottomnote{Source: Bolton \& Hand (2002); Phua et al.\ (2010); paper Section 4.8}
\end{frame}

% ----------------------------------------------------------
% SLIDE 15 -- Critical Assessment: Evaluation, Publication
% ----------------------------------------------------------
\begin{frame}{Critical Assessment: Evaluation Protocols and Publication Bias}
\begin{columns}[T]
\column{0.48\textwidth}
\textbf{Temporal Evaluation Gap}
\begin{itemize}\compactlist
\item Proper: train on period $t$, evaluate on period $t+1$ (temporal split)
\item \textcolor{mlred}{Many studies use random cross-validation}
\item Allows future information to leak into training set
\item \textbf{Inflates reported performance}
\item In a domain with concept drift, temporal evaluation is a \textit{necessity}, not a refinement
\end{itemize}

\column{0.48\textwidth}
\textbf{Publication Bias}
\begin{itemize}\compactlist
\item Positive results preferentially published
\item Studies reporting high detection rates more likely to reach peer-reviewed venues
\item True performance landscape is \textbf{less optimistic} than published literature suggests
\item Potential remedies:
  \begin{itemize}\compactlist
  \item Registered reports
  \item Pre-registered analysis plans
  \item Commitment to publish regardless of outcome
  \item Not yet standard practice in this field
  \end{itemize}
\end{itemize}
\end{columns}
\bottomnote{Source: Paper Section 4.8}
\end{frame}

% ----------------------------------------------------------
% SLIDE 16 -- Critical Assessment: Overfitting
% ----------------------------------------------------------
\begin{frame}{Critical Assessment: Overfitting Risk}
\begin{columns}[T]
\column{0.55\textwidth}
\textbf{The Small-Sample Problem}
\begin{itemize}\compactlist
\item Only $\sim$\numfraudcases{} labeled fraud cases in the historical record
\item Even moderate-complexity models risk \textbf{memorizing} idiosyncratic characteristics of specific schemes
\item A model achieving high performance on 20 held-out fraud cases may simply have learned fingerprints of specific Ponzi schemes, valuation frauds, style misrepresentations in its training set
\item \textcolor{mlred}{Without capacity to detect novel fraud types}
\end{itemize}

\column{0.42\textwidth}
\textbf{Amplifying Factors}
\begin{itemize}\compactlist
\item Common practice: single train-test split (rather than multiple independent evaluations)
\item Heterogeneous positive class: Ponzi $\neq$ NAV manipulation $\neq$ insider trading
\item Pooled fraud labels $\Rightarrow$ compromise decision boundary
\item May fail to detect any individual fraud type with high sensitivity
\end{itemize}
\vspace{3mm}
\textbf{Mitigation Needed}
\begin{itemize}\compactlist
\item Fraud-type-specific evaluation
\item Multiple temporal splits
\item Synthetic augmentation with validation
\item Ensemble of type-specific detectors
\end{itemize}
\end{columns}
\bottomnote{Source: Paper Section 4.8}
\end{frame}

% ----------------------------------------------------------
% SLIDE 17 -- Literature Gaps Chart
% ----------------------------------------------------------
\begin{frame}{Literature Gaps: Visual Summary}

\chartplaceholder[5.5cm]{Chart: 02\_literature\_gaps -- Heatmap or gap analysis showing method families vs.\ critical challenges (reproducibility, benchmark, domain specificity, class imbalance, evaluation protocol, publication bias, overfitting) with gap severity indicators}

\bottomnote{Source: Paper Section 4.8}
\end{frame}

% ----------------------------------------------------------
% SLIDE 18 -- Summary
% ----------------------------------------------------------
\begin{frame}{Summary: Field at Early Stage of Scientific Maturity}
\begin{enumerate}\compactlist
\item \textbf{Classical methods} (Benford, serial correlation, omega-score) provide foundational features but each captures only one fraud dimension
\item \textbf{Tree-based ensembles} dominate: best performance (\fone{} $\sim 0.88$), \shap{}-compatible, robust to class imbalance
\item \textbf{Deep learning} offers theoretical appeal (temporal patterns, nonlinear representations) but faces severe data scarcity ($<100$ positives) and overfitting risk
\item \textbf{NLP} (FinBERT 87\%, SEC-BERT) enables text-quant cross-modal detection; multi-modal fusion adds \textbf{+3--5\% \auc{}}
\item \textbf{GNNs} capture relational ``guilt by association'' (\auc{} 0.87 in adjacent domains) but hedge fund application remains empirically underexplored
\item \textbf{Semi-supervised / self-supervised} methods directly address label scarcity through contrastive learning and pre-train/fine-tune paradigms
\item \textbf{Synthetic data} (GANs, VAEs) address class imbalance but face validation circularity
\item \textcolor{mlred}{\textbf{Critical gaps:}} no standard benchmark, reproducibility crisis, evaluation protocol inconsistencies, publication bias, and high overfitting risk with \numfraudcases{} cases
\end{enumerate}
\bottomnote{Source: Paper Section 4}
\end{frame}

\end{document}
