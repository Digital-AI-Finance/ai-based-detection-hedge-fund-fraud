\documentclass[8pt,aspectratio=169]{beamer}
\usetheme{Madrid}
\usepackage{graphicx}
\usepackage{booktabs}
\usepackage{adjustbox}
\usepackage{multicol}
\usepackage{amsmath}
\usepackage{amssymb}

% Color definitions
\definecolor{mlblue}{RGB}{0,102,204}
\definecolor{mlpurple}{RGB}{51,51,178}
\definecolor{mllavender}{RGB}{173,173,224}
\definecolor{mllavender2}{RGB}{193,193,232}
\definecolor{mllavender3}{RGB}{204,204,235}
\definecolor{mllavender4}{RGB}{214,214,239}
\definecolor{mlorange}{RGB}{255, 127, 14}
\definecolor{mlgreen}{RGB}{44, 160, 44}
\definecolor{mlred}{RGB}{214, 39, 40}
\definecolor{mlgray}{RGB}{127, 127, 127}

% Additional colors
\definecolor{lightgray}{RGB}{240, 240, 240}
\definecolor{midgray}{RGB}{180, 180, 180}

% Apply custom colors to Madrid theme
\setbeamercolor{palette primary}{bg=mllavender3,fg=mlpurple}
\setbeamercolor{palette secondary}{bg=mllavender2,fg=mlpurple}
\setbeamercolor{palette tertiary}{bg=mllavender,fg=white}
\setbeamercolor{palette quaternary}{bg=mlpurple,fg=white}

\setbeamercolor{structure}{fg=mlpurple}
\setbeamercolor{section in toc}{fg=mlpurple}
\setbeamercolor{subsection in toc}{fg=mlblue}
\setbeamercolor{title}{fg=mlpurple}
\setbeamercolor{frametitle}{fg=mlpurple,bg=mllavender3}
\setbeamercolor{block title}{bg=mllavender2,fg=mlpurple}
\setbeamercolor{block body}{bg=mllavender4,fg=black}

% Remove navigation symbols
\setbeamertemplate{navigation symbols}{}

% Clean itemize/enumerate
\setbeamertemplate{itemize items}[circle]
\setbeamertemplate{enumerate items}[default]

% Reduce margins for more content space
\setbeamersize{text margin left=5mm,text margin right=5mm}

% Command for bottom annotation
\newcommand{\bottomnote}[1]{%
\vfill
\vspace{-2mm}
\textcolor{mllavender2}{\rule{\textwidth}{0.4pt}}
\vspace{1mm}
\footnotesize
\textbf{#1}
}

% Command for compact list spacing
\newcommand{\compactlist}{%
\setlength{\itemsep}{0pt}%
\setlength{\parskip}{0pt}%
\setlength{\parsep}{0pt}%
}

% Notation macros
% Shared notation macros for AI-Based Detection of Hedge Fund Fraud
% Include this file in all Beamer slide decks via % Shared notation macros for AI-Based Detection of Hedge Fund Fraud
% Include this file in all Beamer slide decks via % Shared notation macros for AI-Based Detection of Hedge Fund Fraud
% Include this file in all Beamer slide decks via \input{notation}

% Performance metrics
\newcommand{\auc}{\ensuremath{\mathrm{AUC}}}
\newcommand{\fone}{\ensuremath{F_1}}
\newcommand{\shap}{\text{SHAP}}
\newcommand{\lime}{\text{LIME}}

% Autocorrelation
\newcommand{\rhoone}{\ensuremath{\rho_1}}

% Financial abbreviations
\newcommand{\nav}{\text{NAV}}
\newcommand{\aum}{\text{AUM}}

% Regulatory
\newcommand{\euaiact}{EU AI Act}
\newcommand{\aifmd}{AIFMD}
\newcommand{\sec}{SEC}
\newcommand{\edgar}{EDGAR}

% Key numbers from the paper (for consistency)
\newcommand{\aumdollar}{\$4.5\text{ trillion}}
\newcommand{\aucdeg}{10.6\%}
\newcommand{\numop}{10}
\newcommand{\numfraudcases}{50\text{--}100}
\newcommand{\systematicpapers}{105}


% Performance metrics
\newcommand{\auc}{\ensuremath{\mathrm{AUC}}}
\newcommand{\fone}{\ensuremath{F_1}}
\newcommand{\shap}{\text{SHAP}}
\newcommand{\lime}{\text{LIME}}

% Autocorrelation
\newcommand{\rhoone}{\ensuremath{\rho_1}}

% Financial abbreviations
\newcommand{\nav}{\text{NAV}}
\newcommand{\aum}{\text{AUM}}

% Regulatory
\newcommand{\euaiact}{EU AI Act}
\newcommand{\aifmd}{AIFMD}
\newcommand{\sec}{SEC}
\newcommand{\edgar}{EDGAR}

% Key numbers from the paper (for consistency)
\newcommand{\aumdollar}{\$4.5\text{ trillion}}
\newcommand{\aucdeg}{10.6\%}
\newcommand{\numop}{10}
\newcommand{\numfraudcases}{50\text{--}100}
\newcommand{\systematicpapers}{105}


% Performance metrics
\newcommand{\auc}{\ensuremath{\mathrm{AUC}}}
\newcommand{\fone}{\ensuremath{F_1}}
\newcommand{\shap}{\text{SHAP}}
\newcommand{\lime}{\text{LIME}}

% Autocorrelation
\newcommand{\rhoone}{\ensuremath{\rho_1}}

% Financial abbreviations
\newcommand{\nav}{\text{NAV}}
\newcommand{\aum}{\text{AUM}}

% Regulatory
\newcommand{\euaiact}{EU AI Act}
\newcommand{\aifmd}{AIFMD}
\newcommand{\sec}{SEC}
\newcommand{\edgar}{EDGAR}

% Key numbers from the paper (for consistency)
\newcommand{\aumdollar}{\$4.5\text{ trillion}}
\newcommand{\aucdeg}{10.6\%}
\newcommand{\numop}{10}
\newcommand{\numfraudcases}{50\text{--}100}
\newcommand{\systematicpapers}{105}


\title{Quiz: Research Agenda}
\subtitle{AI-Based Detection of Hedge Fund Fraud}
\author{Joerg Osterrieder}
\institute{Zurich University of Applied Sciences (ZHAW)}
\date{2025}

\begin{document}

\begin{frame}
\titlepage
\end{frame}

\begin{frame}{Question 1}
How many open problems are identified in the research agenda?
\begin{itemize}\compactlist
\item[a)] 5
\item[b)] 8
\item[c)] 10
\item[d)] 12
\end{itemize}

\pause
\begin{block}{Answer}
\textbf{c)} 10\\[2mm]
The research agenda identifies exactly 10 open problems (OP1--OP10), organized across three categories: Data challenges, Methodological advances, and Deployment considerations.
\end{block}
\bottomnote{Section 6: Research Agenda}
\end{frame}

\begin{frame}{Question 2}
What are the three categories of open problems?
\begin{itemize}\compactlist
\item[a)] Input/Process/Output
\item[b)] Data/Methodological/Deployment
\item[c)] Theory/Practice/Policy
\item[d)] Design/Build/Test
\end{itemize}

\pause
\begin{block}{Answer}
\textbf{b)} Data/Methodological/Deployment\\[2mm]
The 10 open problems are organized into: (1) Data challenges (OP1--OP3), (2) Methodological advances (OP4--OP8), and (3) Deployment considerations (OP9--OP10).
\end{block}
\bottomnote{Section 6: Research Agenda}
\end{frame}

\begin{frame}{Question 3}
Which two problems are identified as critical preconditions for the field?
\begin{itemize}\compactlist
\item[a)] OP2 and OP5
\item[b)] OP3 and OP6
\item[c)] OP1 and OP4
\item[d)] OP8 and OP9
\end{itemize}

\pause
\begin{block}{Answer}
\textbf{c)} OP1 and OP4\\[2mm]
OP1 (Benchmark datasets) and OP4 (Interpretability standards) are critical preconditions. Without accessible data and explainable models, the field cannot progress toward production deployment.
\end{block}
\bottomnote{Section 6: Research Agenda}
\end{frame}

\begin{frame}{Question 4}
What is OP1's two-track approach for creating benchmark datasets?
\begin{itemize}\compactlist
\item[a)] Survey + interview
\item[b)] Synthetic data + differential privacy
\item[c)] Open source + proprietary
\item[d)] Academic + industry
\end{itemize}

\pause
\begin{block}{Answer}
\textbf{b)} Synthetic data + differential privacy\\[2mm]
OP1 proposes generating synthetic fraud datasets while applying differential privacy techniques to protect sensitive information from real cases, enabling research without exposing confidential data.
\end{block}
\bottomnote{Section 6.1: Data Challenges}
\end{frame}

\begin{frame}{Question 5}
What method does OP2 suggest for cross-jurisdictional data integration?
\begin{itemize}\compactlist
\item[a)] Federated learning
\item[b)] Blockchain
\item[c)] Cloud computing
\item[d)] API integration
\end{itemize}

\pause
\begin{block}{Answer}
\textbf{a)} Federated learning\\[2mm]
OP2 proposes federated learning frameworks that enable model training across multiple jurisdictions without sharing raw data, respecting privacy regulations while leveraging global patterns.
\end{block}
\bottomnote{Section 6.1: Data Challenges}
\end{frame}

\begin{frame}{Question 6}
How many labeled fraud cases typically exist in the literature?
\begin{itemize}\compactlist
\item[a)] 10--20
\item[b)] 25--50
\item[c)] 50--100
\item[d)] 200--500
\end{itemize}

\pause
\begin{block}{Answer}
\textbf{c)} 50--100\\[2mm]
The paper identifies that only 50--100 labeled fraud cases exist, creating a severe small-sample problem. OP3 addresses semi-supervised and active learning to work with limited labels.
\end{block}
\bottomnote{Section 6.1: Data Challenges}
\end{frame}

\begin{frame}{Question 7}
What is the name of OP5?
\begin{itemize}\compactlist
\item[a)] Warm-start training
\item[b)] Cold-start detection
\item[c)] Zero-shot classification
\item[d)] Bootstrap analysis
\end{itemize}

\pause
\begin{block}{Answer}
\textbf{b)} Cold-start detection\\[2mm]
OP5 focuses on detecting fraud in funds with limited history (12--24 months of returns), where traditional statistical tests lack power. This requires methods that work with short time series.
\end{block}
\bottomnote{Section 6.2: Methodological Advances}
\end{frame}

\begin{frame}{Question 8}
Which two problems have the lowest feasibility scores?
\begin{itemize}\compactlist
\item[a)] OP1 and OP3
\item[b)] OP4 and OP7
\item[c)] OP5 and OP6
\item[d)] OP2 and OP10
\end{itemize}

\pause
\begin{block}{Answer}
\textbf{d)} OP2 and OP10\\[2mm]
OP2 (Cross-jurisdictional data) and OP10 (Regulatory alignment) have lowest feasibility due to legal barriers, international coordination challenges, and complex compliance requirements.
\end{block}
\bottomnote{Section 6: Priority Matrix}
\end{frame}

\end{document}
