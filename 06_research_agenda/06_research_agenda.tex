% ============================================================
%  Slide Deck 6 -- Research Agenda: 10 Open Problems
%  AI-Based Detection of Hedge Fund Fraud
% ============================================================
\documentclass[8pt,aspectratio=169]{beamer}
\usetheme{Madrid}
\usepackage{graphicx}
\usepackage{booktabs}
\usepackage{adjustbox}
\usepackage{multicol}
\usepackage{amsmath}
\usepackage{amssymb}

% ---- Color definitions ----
\definecolor{mlblue}{RGB}{0,102,204}
\definecolor{mlpurple}{RGB}{51,51,178}
\definecolor{mllavender}{RGB}{173,173,224}
\definecolor{mllavender2}{RGB}{193,193,232}
\definecolor{mllavender3}{RGB}{204,204,235}
\definecolor{mllavender4}{RGB}{214,214,239}
\definecolor{mlorange}{RGB}{255,127,14}
\definecolor{mlgreen}{RGB}{44,160,44}
\definecolor{mlred}{RGB}{214,39,40}
\definecolor{mlgray}{RGB}{127,127,127}
\definecolor{lightgray}{RGB}{240,240,240}
\definecolor{midgray}{RGB}{180,180,180}

% ---- Apply custom colors to Madrid theme ----
\setbeamercolor{palette primary}{bg=mllavender3,fg=mlpurple}
\setbeamercolor{palette secondary}{bg=mllavender2,fg=mlpurple}
\setbeamercolor{palette tertiary}{bg=mllavender,fg=white}
\setbeamercolor{palette quaternary}{bg=mlpurple,fg=white}
\setbeamercolor{structure}{fg=mlpurple}
\setbeamercolor{section in toc}{fg=mlpurple}
\setbeamercolor{subsection in toc}{fg=mlblue}
\setbeamercolor{title}{fg=mlpurple}
\setbeamercolor{frametitle}{fg=mlpurple,bg=mllavender3}
\setbeamercolor{block title}{bg=mllavender2,fg=mlpurple}
\setbeamercolor{block body}{bg=mllavender4,fg=black}

% ---- Navigation / itemize ----
\setbeamertemplate{navigation symbols}{}
\setbeamertemplate{itemize items}[circle]
\setbeamertemplate{enumerate items}[default]
\setbeamersize{text margin left=5mm,text margin right=5mm}

% ---- Custom commands ----
\newcommand{\bottomnote}[1]{%
\vfill
\vspace{-2mm}
\textcolor{mllavender2}{\rule{\textwidth}{0.4pt}}
\vspace{1mm}
\footnotesize
\textbf{#1}
}

\newcommand{\compactlist}{%
\setlength{\itemsep}{0pt}%
\setlength{\parskip}{0pt}%
\setlength{\parsep}{0pt}%
}

\newcommand{\chartplaceholder}[2][5cm]{%
\begin{center}
\begin{adjustbox}{max width=0.95\textwidth, max height=#1}
\framebox[\textwidth][c]{%
\rule{0pt}{#1}%
\textcolor{midgray}{[#2]}%
}
\end{adjustbox}
\end{center}
}

% ---- Notation ----
% Shared notation macros for AI-Based Detection of Hedge Fund Fraud
% Include this file in all Beamer slide decks via % Shared notation macros for AI-Based Detection of Hedge Fund Fraud
% Include this file in all Beamer slide decks via % Shared notation macros for AI-Based Detection of Hedge Fund Fraud
% Include this file in all Beamer slide decks via \input{notation}

% Performance metrics
\newcommand{\auc}{\ensuremath{\mathrm{AUC}}}
\newcommand{\fone}{\ensuremath{F_1}}
\newcommand{\shap}{\text{SHAP}}
\newcommand{\lime}{\text{LIME}}

% Autocorrelation
\newcommand{\rhoone}{\ensuremath{\rho_1}}

% Financial abbreviations
\newcommand{\nav}{\text{NAV}}
\newcommand{\aum}{\text{AUM}}

% Regulatory
\newcommand{\euaiact}{EU AI Act}
\newcommand{\aifmd}{AIFMD}
\newcommand{\sec}{SEC}
\newcommand{\edgar}{EDGAR}

% Key numbers from the paper (for consistency)
\newcommand{\aumdollar}{\$4.5\text{ trillion}}
\newcommand{\aucdeg}{10.6\%}
\newcommand{\numop}{10}
\newcommand{\numfraudcases}{50\text{--}100}
\newcommand{\systematicpapers}{105}


% Performance metrics
\newcommand{\auc}{\ensuremath{\mathrm{AUC}}}
\newcommand{\fone}{\ensuremath{F_1}}
\newcommand{\shap}{\text{SHAP}}
\newcommand{\lime}{\text{LIME}}

% Autocorrelation
\newcommand{\rhoone}{\ensuremath{\rho_1}}

% Financial abbreviations
\newcommand{\nav}{\text{NAV}}
\newcommand{\aum}{\text{AUM}}

% Regulatory
\newcommand{\euaiact}{EU AI Act}
\newcommand{\aifmd}{AIFMD}
\newcommand{\sec}{SEC}
\newcommand{\edgar}{EDGAR}

% Key numbers from the paper (for consistency)
\newcommand{\aumdollar}{\$4.5\text{ trillion}}
\newcommand{\aucdeg}{10.6\%}
\newcommand{\numop}{10}
\newcommand{\numfraudcases}{50\text{--}100}
\newcommand{\systematicpapers}{105}


% Performance metrics
\newcommand{\auc}{\ensuremath{\mathrm{AUC}}}
\newcommand{\fone}{\ensuremath{F_1}}
\newcommand{\shap}{\text{SHAP}}
\newcommand{\lime}{\text{LIME}}

% Autocorrelation
\newcommand{\rhoone}{\ensuremath{\rho_1}}

% Financial abbreviations
\newcommand{\nav}{\text{NAV}}
\newcommand{\aum}{\text{AUM}}

% Regulatory
\newcommand{\euaiact}{EU AI Act}
\newcommand{\aifmd}{AIFMD}
\newcommand{\sec}{SEC}
\newcommand{\edgar}{EDGAR}

% Key numbers from the paper (for consistency)
\newcommand{\aumdollar}{\$4.5\text{ trillion}}
\newcommand{\aucdeg}{10.6\%}
\newcommand{\numop}{10}
\newcommand{\numfraudcases}{50\text{--}100}
\newcommand{\systematicpapers}{105}


% ---- Title metadata ----
\title{AI-Based Detection of Hedge Fund Fraud}
\subtitle{Section 6 -- Research Agenda: 10 Open Problems (C3)}
\author{Joerg Osterrieder}
\institute{Zurich University of Applied Sciences (ZHAW)}
\date{2025}

% ============================================================
\begin{document}

% ----------------------------------------------------------
% SLIDE 1 -- Title
% ----------------------------------------------------------
\begin{frame}
\titlepage
\end{frame}

% ----------------------------------------------------------
% SLIDE 2 -- Outline
% ----------------------------------------------------------
\begin{frame}{Outline}
\begin{enumerate}\compactlist
\item Data Challenges Overview
\item OP1: Benchmark Dataset Creation
\item OP2: Cross-Jurisdictional Data Integration
\item OP3: Real-Time Alternative Data Pipelines
\item Methodological Challenges Overview
\item OP4: Extreme Class Imbalance at Small Scale
\item OP5: Cold-Start Detection for New Funds
\item OP6: Temporal Concept Drift and Adaptive Models
\item OP7: Multi-Modal Fusion Architectures
\item Deployment Challenges Overview
\item OP8: Adversarial Robustness Guarantees
\item OP9: Explainability Without Performance Loss
\item OP10: Human-AI Collaboration
\item Priority Matrix and Dependencies
\item Path Forward: Collaboration Required
\item Summary
\end{enumerate}
\end{frame}

% ----------------------------------------------------------
% SLIDE 3 -- Data Challenges Overview
% ----------------------------------------------------------
\begin{frame}{Data Challenges: OP1--OP3}
\begin{block}{The Fundamental Constraint}
Progress in AI-based hedge fund fraud detection is \textbf{fundamentally constrained by data availability}. Unlike credit card fraud, which benefits from large-scale public benchmarks, the hedge fund domain lacks standardized datasets, suffers from jurisdictional fragmentation, and relies on reporting cycles with \textbf{months-long detection lags}.
\end{block}
\vspace{3mm}
\begin{columns}[T]
\column{0.32\textwidth}
\begin{block}{OP1: Benchmarks}
No public benchmark dataset exists. Each study assembles proprietary data with different labels.
\end{block}

\column{0.32\textwidth}
\begin{block}{OP2: Cross-Jurisdiction}
Data fragmented across regulators. No unified infrastructure connecting reporting regimes.
\end{block}

\column{0.32\textwidth}
\begin{block}{OP3: Real-Time Data}
Monthly/quarterly reporting introduces detection lags that sophisticated fraudsters exploit.
\end{block}
\end{columns}
\bottomnote{Source: Paper Section 6.1}
\end{frame}

% ----------------------------------------------------------
% SLIDE 4 -- OP1: Benchmark Datasets
% ----------------------------------------------------------
\begin{frame}{OP1: Benchmark Dataset Creation}
\begin{columns}[T]
\column{0.50\textwidth}
\textbf{Why It Is Critical}
\begin{itemize}\compactlist
\item \textbf{No public benchmark} exists for hedge fund fraud detection
\item Credit card fraud has widely used benchmarks $\Rightarrow$ hundreds of reproducible comparisons
\item Hedge fund studies: proprietary data, different labels, non-overlapping populations
\item \textbf{Cross-study comparison effectively impossible}
\item Structural obstacles: proprietary data, restrictive licenses, sparse monthly returns, only 10K--15K funds total, scarce confirmed fraud labels
\end{itemize}

\column{0.47\textwidth}
\textbf{Two-Track Approach}
\begin{enumerate}\compactlist
\item \textcolor{mlblue}{\textbf{Synthetic benchmark}}:
  \begin{itemize}\compactlist
  \item Regime-switching models calibrated to empirical return distributions
  \item Injected fraud patterns: Ponzi, NAV smoothing, style drift
  \item Rates calibrated to known enforcement cases
  \end{itemize}
\item \textcolor{mlblue}{\textbf{Anonymized regulatory data}}:
  \begin{itemize}\compactlist
  \item Differential privacy protocol for regulator-held labeled datasets
  \item VAEs and GANs trained on real data to produce synthetic releases
  \item Preserves aggregate statistics while protecting identities
  \end{itemize}
\end{enumerate}
\end{columns}
\vspace{2mm}
\textcolor{mlred}{OP1 is a \textbf{precondition} for OP4, OP6, OP7, and OP8.}
\bottomnote{Source: Kingma \& Welling (2014); Goodfellow et al.\ (2014); paper Section 6.1.1}
\end{frame}

% ----------------------------------------------------------
% SLIDE 5 -- OP2: Cross-Jurisdictional Integration
% ----------------------------------------------------------
\begin{frame}{OP2: Cross-Jurisdictional Data Integration}
\begin{columns}[T]
\column{0.52\textwidth}
\textbf{The Problem}
\begin{itemize}\compactlist
\item A single fund may be:
  \begin{itemize}\compactlist
  \item Domiciled in Cayman Islands
  \item Managed from New York
  \item Marketing passport in Europe
  \item Asian prime brokerage relationships
  \end{itemize}
\item Generates regulatory filings across \textbf{multiple jurisdictions} with different formats, frequencies, disclosures
\item No single regulator sees the complete picture
\item Funds exploit regulatory arbitrage \textbf{deliberately}
\item SEC Form ADV, FCA registry, ESMA/\aifmd{} not linked, different identifiers, different classification schemes
\end{itemize}

\column{0.45\textwidth}
\textbf{Suggested Approach: Federated Learning}
\begin{itemize}\compactlist
\item Multiple regulatory agencies train a \textbf{shared model} without exchanging raw data
\item Each jurisdiction trains local updates on its own data
\item Central server aggregates into a global model
\item Differential privacy layered on top for formal privacy bounds
\item Key research challenges:
  \begin{itemize}\compactlist
  \item \textbf{Entity resolution} across jurisdictions without sharing identifiable data
  \item Standardized feature extraction from \textbf{heterogeneous reporting formats}
  \end{itemize}
\end{itemize}
\end{columns}
\bottomnote{Source: Paper Section 6.1.2}
\end{frame}

% ----------------------------------------------------------
% SLIDE 6 -- OP3: Real-Time Alternative Data
% ----------------------------------------------------------
\begin{frame}{OP3: Real-Time Alternative Data Pipelines}
\begin{columns}[T]
\column{0.50\textwidth}
\textbf{The Detection Lag Problem}
\begin{itemize}\compactlist
\item Most detection operates on \textbf{monthly or quarterly data}
\item By the time a pattern becomes visible:
  \begin{itemize}\compactlist
  \item Months of additional investor capital at risk
  \item Manager may have adjusted behavior or absconded
  \end{itemize}
\item Hedge fund lag: \textbf{orders of magnitude greater} than other fraud domains
  \begin{itemize}\compactlist
  \item Banking: real-time, sub-second
  \item Credit card: per-transaction at point of sale
  \item Hedge funds: 30--60 day return lag, quarterly filings, 45-day holdings delay
  \end{itemize}
\end{itemize}

\column{0.47\textwidth}
\textbf{Alternative Data Sources}
\begin{itemize}\compactlist
\item \textbf{News sentiment}: transformer-based NLP (BERT) flags concerns within hours
\item \textbf{Social media}: investor complaints, whistleblower signals
\item \textbf{Web scraping}: marketing materials, employee turnover, litigation filings
\item \textbf{Early warning}: signals precede return changes by months
\end{itemize}
\vspace{2mm}
\textbf{Research Challenge}
\begin{itemize}\compactlist
\item Fusion architectures integrating high-frequency noisy alternative data with low-frequency reliable periodic data
\item Attention-based architectures that learn modality weighting by time horizon
\end{itemize}
\end{columns}
\bottomnote{Source: Devlin et al.\ (2019); Vaswani et al.\ (2017); paper Section 6.1.3}
\end{frame}

% ----------------------------------------------------------
% SLIDE 7 -- Methodological Challenges Overview
% ----------------------------------------------------------
\begin{frame}{Methodological Challenges: OP4--OP7}
\begin{block}{Beyond Standard ML Practice}
Even with adequate data, hedge fund fraud detection poses methodological challenges that \textbf{distinguish it from other fraud domains} and require tailored solutions.
\end{block}
\vspace{3mm}
\begin{columns}[T]
\column{0.24\textwidth}
\begin{block}{OP4: Imbalance}
\numfraudcases{} labeled cases total. Standard oversampling fails at this scale.
\end{block}

\column{0.24\textwidth}
\begin{block}{OP5: Cold-Start}
New funds lack history. Early period is when fraud risk may be highest.
\end{block}

\column{0.24\textwidth}
\begin{block}{OP6: Drift}
Fraud patterns evolve. Legitimate strategy changes mimic fraud signatures.
\end{block}

\column{0.24\textwidth}
\begin{block}{OP7: Fusion}
Returns, filings, holdings, networks, alt data: radically different modalities.
\end{block}
\end{columns}
\bottomnote{Source: Paper Section 6.2}
\end{frame}

% ----------------------------------------------------------
% SLIDE 8 -- OP4: Extreme Class Imbalance
% ----------------------------------------------------------
\begin{frame}{OP4: Extreme Class Imbalance at Small Scale}
\begin{columns}[T]
\column{0.50\textwidth}
\textbf{Why Standard Methods Fail}
\begin{itemize}\compactlist
\item Credit card fraud: 0.1--0.5\% base rate but \textbf{millions of transactions} $\Rightarrow$ thousands of positives
\item Hedge fund fraud: 0.5--3\% base rate but only \textbf{10K--15K funds total}
\item Result: only \textbf{\numfraudcases{} labeled fraud cases} across entire historical record
\item Small-$N$, high-dimensional, multi-modal problem
\item SMOTE fails: minority class is \textbf{fundamentally heterogeneous}
  \begin{itemize}\compactlist
  \item Ponzi schemes, NAV manipulation, style drift produce entirely different signatures
  \item Interpolating between fraud types generates nonexistent patterns
  \end{itemize}
\end{itemize}

\column{0.47\textwidth}
\textbf{Suggested Approaches}
\begin{itemize}\compactlist
\item \textcolor{mlblue}{\textbf{Few-shot learning}}: classify from 5--10 labeled cases per fraud type via meta-learned representations
\item \textcolor{mlblue}{\textbf{Meta-learning across fraud types}}: train on a distribution of detection tasks
\item \textcolor{mlblue}{\textbf{Semi-supervised contrastive learning}}: exploit large unlabeled pool by contrasting normal/anomalous in embedding space
\item \textcolor{mlblue}{\textbf{Transfer learning}}: from insurance fraud, financial statement fraud, AML
\item Evaluation: \textbf{fraud-type-stratified splits}, not pooled cross-validation
\end{itemize}
\end{columns}
\vspace{1mm}
\textcolor{mlred}{OP4 is a \textbf{precondition} for all supervised methods.}
\bottomnote{Source: Chawla et al.\ (2002); paper Section 6.2.1}
\end{frame}

% ----------------------------------------------------------
% SLIDE 9 -- OP5: Cold-Start Detection
% ----------------------------------------------------------
\begin{frame}{OP5: Cold-Start Detection for New and Emerging Funds}
\begin{columns}[T]
\column{0.50\textwidth}
\textbf{The Problem}
\begin{itemize}\compactlist
\item New funds lack historical return data (primary input to most models)
\item 3 months of track record $\Rightarrow$ \textbf{insufficient statistical power} for distributional tests, correlation analysis, style detection
\item Yet the \textbf{early period} is when fraud risk may be highest (credibility-building phase)
\item Genuinely informationally opaque: no return history, no filing history, no track record
\item Incubation structures allow \textbf{selective reporting} of successful histories (backfill bias)
\end{itemize}

\column{0.47\textwidth}
\textbf{Suggested Approaches}
\begin{itemize}\compactlist
\item \textcolor{mlblue}{\textbf{Operational due diligence features}} (available at registration):
  \begin{itemize}\compactlist
  \item Auditor quality, administrator independence
  \item Custody arrangements, governance structures
  \item Predictive power demonstrated by Dimmock \& Gerken (2012)
  \end{itemize}
\item \textcolor{mlblue}{\textbf{Transfer learning}} from similar strategy/manager profiles
\item \textcolor{mlblue}{\textbf{Network analysis}}: manager's prior fund relationships, service provider networks
\item \textcolor{mlblue}{\textbf{NLP on launch documents}}: assess plausibility and consistency of stated strategies before any returns
\end{itemize}
\end{columns}
\bottomnote{Source: Dimmock \& Gerken (2012); Brown et al.\ (2008); paper Section 6.2.2}
\end{frame}

% ----------------------------------------------------------
% SLIDE 10 -- OP6: Concept Drift
% ----------------------------------------------------------
\begin{frame}{OP6: Temporal Concept Drift and Adaptive Models}
\begin{columns}[T]
\column{0.50\textwidth}
\textbf{The Unique Challenge}
\begin{itemize}\compactlist
\item Fraud patterns evolve as perpetrators learn from detected schemes
\item Standard drift solutions: periodic retraining, online learning
\item Hedge fund complication: \textbf{legitimate strategy changes} produce the same statistical signatures as fraud
  \begin{itemize}\compactlist
  \item Equity L/S pivots to global macro
  \item Quant fund shifts from momentum to mean-reversion
  \item Changes in factor exposures, return distributions, risk profiles
  \end{itemize}
\item Credit card drift: slow, predictable (lifecycle/inflation)
\item Hedge fund drift: \textbf{sudden, large, strategically motivated}
\end{itemize}

\column{0.47\textwidth}
\textbf{Suggested Approaches}
\begin{itemize}\compactlist
\item \textcolor{mlblue}{\textbf{Strategy-aware drift detection}}: incorporate market regime and factor conditions
\item \textcolor{mlblue}{\textbf{ADWIN / DDM}} augmented with regime-switching models
\item \textcolor{mlblue}{\textbf{Factor-conditioned anomaly scores}}: normalize against peer funds pursuing similar strategies
\item Reduces false positives from legitimate strategy evolution
\item Key evaluation challenge: test sets with both \textbf{genuinely drifting fraud} and \textbf{legitimately adapting non-fraud}
\end{itemize}
\end{columns}
\bottomnote{Source: Paper Section 6.2.3}
\end{frame}

% ----------------------------------------------------------
% SLIDE 11 -- OP7: Multi-Modal Fusion
% ----------------------------------------------------------
\begin{frame}{OP7: Multi-Modal Fusion Architectures}
\begin{columns}[T]
\column{0.50\textwidth}
\textbf{The Data Modality Problem}
\begin{itemize}\compactlist
\item No other financial entity generates this combination:
  \begin{itemize}\compactlist
  \item Monthly returns: sparse, 36--120 observations
  \item Regulatory filings: unstructured narrative (NLP)
  \item Form 13F holdings: high-dimensional, 45-day lag
  \item Network data: relational (prime brokers, auditors)
  \item Alternative data: irregular, high-frequency
  \end{itemize}
\item Radically different frequencies, dimensionalities, missingness patterns
\item Existing approaches: single modality or hand-engineered feature concatenation
\end{itemize}

\column{0.47\textwidth}
\textbf{Suggested Architectures}
\begin{itemize}\compactlist
\item \textcolor{mlblue}{\textbf{Attention-based multi-modal transformers}}: learn dynamic modality weighting by informativeness
\item \textcolor{mlblue}{\textbf{Cross-modal contrastive learning}}: flag funds whose textual descriptions contradict quantitative behavior
\item \textcolor{mlblue}{\textbf{Hierarchical fusion}}: modality-specific encoders + higher-level combination
  \begin{itemize}\compactlist
  \item RNNs for return series
  \item Pre-trained LMs (BERT) for text
  \item GNNs for relational data
  \end{itemize}
\item Evaluate \textbf{marginal contribution} of each modality
\end{itemize}
\end{columns}
\bottomnote{Source: Vaswani et al.\ (2017); Devlin et al.\ (2019); Kipf \& Welling (2017); paper Section 6.2.4}
\end{frame}

% ----------------------------------------------------------
% SLIDE 12 -- Deployment Challenges Overview
% ----------------------------------------------------------
\begin{frame}{Deployment Challenges: OP8--OP10}
\begin{block}{From Proof-of-Concept to Operational Deployment}
Even a technically superior detection model is of limited practical value if it cannot \textbf{withstand adversarial manipulation}, \textbf{explain its decisions} to regulators and courts, or \textbf{integrate effectively} into human investigation workflows.
\end{block}
\vspace{3mm}
\begin{columns}[T]
\column{0.32\textwidth}
\begin{block}{OP8: Robustness}
Detection is inherently adversarial. Robustness is a \textbf{necessary condition}, not a desirable property.
\end{block}

\column{0.32\textwidth}
\begin{block}{OP9: Explainability}
EU AI Act mandates transparency. Best models resist explanation. Current post-hoc methods are insufficient.
\end{block}

\column{0.32\textwidth}
\begin{block}{OP10: Human-AI}
Models generate alerts; humans investigate. Effectiveness depends on \textbf{interaction quality}.
\end{block}
\end{columns}
\bottomnote{Source: Paper Section 6.3}
\end{frame}

% ----------------------------------------------------------
% SLIDE 13 -- OP8: Adversarial Robustness
% ----------------------------------------------------------
\begin{frame}{OP8: Adversarial Robustness Guarantees}
\begin{columns}[T]
\column{0.50\textwidth}
\textbf{Why Generic Robustness Is Insufficient}
\begin{itemize}\compactlist
\item Hedge fund adversaries are \textbf{qualitatively different} from credit card fraudsters
\item PhD-level quants with same tools as detector
\item Can simulate models, identify boundaries, engineer plausible evasion
\item Standard $\ell_p$-norm adversarial training is not enough
\item Perturbation budgets must be \textbf{economically meaningful}:
  \begin{itemize}\compactlist
  \item Returns must satisfy investor expectations
  \item Pass administrator review
  \item Maintain strategy consistency
  \end{itemize}
\end{itemize}

\column{0.47\textwidth}
\textbf{Suggested Approaches}
\begin{itemize}\compactlist
\item \textcolor{mlblue}{\textbf{Certified robustness bounds}}: formal guarantees on max output change under bounded perturbation
\item \textcolor{mlblue}{\textbf{Game-theoretic modeling}}: detector vs.\ fraudster as players in repeated game with asymmetric information
\item \textcolor{mlblue}{\textbf{Red-teaming}}: domain experts construct evasive return series against specific models
\item \textcolor{mlblue}{\textbf{Robust ensembles}}: diverse model families (statistical, tree, neural, network) force adversary to fool multiple independent mechanisms
\end{itemize}
\end{columns}
\bottomnote{Source: Paper Section 6.3.1}
\end{frame}

% ----------------------------------------------------------
% SLIDE 14 -- OP9: Explainability Without Performance Loss
% ----------------------------------------------------------
\begin{frame}{OP9: Explainability Without Sacrificing Performance}
\begin{columns}[T]
\column{0.50\textwidth}
\textbf{The Stringent Bar}
\begin{itemize}\compactlist
\item EU AI Act: ``sufficiently transparent'' outputs
\item Must satisfy \textbf{multiple audiences simultaneously}:
  \begin{itemize}\compactlist
  \item SEC/FCA examiners during routine examination
  \item Enforcement attorneys in proceedings
  \item Judges and juries in contested cases
  \end{itemize}
\item Materially higher bar than standard AML compliance dashboards
\item No standardized examination protocol for AI-assisted surveillance
\item Rudin (2019): high-stakes domains should prefer \textbf{inherently interpretable models}
\end{itemize}

\column{0.47\textwidth}
\textbf{Suggested Approaches}
\begin{itemize}\compactlist
\item \textcolor{mlblue}{\textbf{Neural additive models / explainable boosting machines}}: nonlinear capacity with per-feature transparency
\item \textcolor{mlblue}{\textbf{Faithful distillation}}: interpretable student model approximating complex teacher
\item \textcolor{mlblue}{\textbf{Attention-based explanations}}: highlight which return periods, filing passages, network connections contributed most
\item \textcolor{mlblue}{\textbf{Standardized explanation templates}}: structured reports mapping ML output to regulatory fraud indicators
\end{itemize}
\end{columns}
\bottomnote{Source: Rudin (2019); Lundberg \& Lee (2017); paper Section 6.3.2}
\end{frame}

% ----------------------------------------------------------
% SLIDE 15 -- OP10: Human-AI Collaboration
% ----------------------------------------------------------
\begin{frame}{OP10: Human-AI Collaboration in Fraud Investigation}
\begin{columns}[T]
\column{0.50\textwidth}
\textbf{The Operational Reality}
\begin{itemize}\compactlist
\item No AI system will replace human judgment in foreseeable future
\item Models generate alerts; humans \textbf{evaluate, investigate, decide}
\item Investigations: months of document review, forensic accounting, expert consultation
\item Low base rate $\Rightarrow$ even high-precision models generate more false positives than true positives in absolute terms
\item \textcolor{mlred}{\textbf{Alert fatigue}}: repeated false positives cause investigators to discount or ignore alerts -- negating detection benefits
\end{itemize}

\column{0.47\textwidth}
\textbf{Suggested Approaches}
\begin{itemize}\compactlist
\item \textcolor{mlblue}{\textbf{Active learning}}: model queries investigators about cases most likely to improve decision boundary
\item \textcolor{mlblue}{\textbf{Prioritized alert queues}}: rank by fraud probability $\times$ estimated financial impact
\item \textcolor{mlblue}{\textbf{Investigation-ready packages}}: fraud score, contributing features (\shap{}), historical comparisons, suggested inquiry lines
\item \textcolor{mlblue}{\textbf{Adaptive thresholds}}: adjust based on investigator capacity and historical precision
\item \textcolor{mlblue}{\textbf{Feedback loops}}: investigators label alerts as TP/FP/inconclusive $\Rightarrow$ retrain
\end{itemize}
\end{columns}
\bottomnote{Source: Paper Section 6.3.3}
\end{frame}

% ----------------------------------------------------------
% SLIDE 16 -- Open Problems Map
% ----------------------------------------------------------
\begin{frame}{Open Problems Map}

\chartplaceholder[5cm]{Chart: 01\_open\_problems\_map -- Visual map of 10 open problems organized by category (Data: OP1--3, Methods: OP4--7, Deployment: OP8--10) with dependency arrows showing how OP1 enables OP4/OP6/OP7/OP8 and OP4 is precondition for all supervised methods}

\vspace{2mm}
\begin{itemize}\compactlist
\item Three categories: \textbf{Data} (OP1--3), \textbf{Methods} (OP4--7), \textbf{Deployment} (OP8--10)
\item Dense interdependencies: data problems constrain methodological and deployment progress
\end{itemize}
\bottomnote{Source: Paper Section 6.4}
\end{frame}

% ----------------------------------------------------------
% SLIDE 17 -- Priority Matrix
% ----------------------------------------------------------
\begin{frame}{Priority Matrix: Impact $\times$ Feasibility}

\chartplaceholder[4cm]{Chart: 02\_priority\_matrix -- Scatter plot of 10 open problems positioned by Impact (y-axis: Medium to Critical) vs.\ Feasibility (x-axis: Low to High), with OP1 and OP4 in upper-right quadrant marked as critical preconditions}

\vspace{1mm}
{\footnotesize
\begin{center}
\begin{tabular}{l l c c l}
\toprule
\textbf{ID} & \textbf{Problem} & \textbf{Impact} & \textbf{Feasibility} & \textbf{Dependencies} \\
\midrule
OP1 & Benchmark dataset & Critical & Medium & Precondition for OP4, OP6, OP7, OP8 \\
OP4 & Class imbalance & Critical & Medium & Precondition for all supervised methods \\
OP9 & Explainability & High & Med--High & Required for OP10, regulatory deployment \\
OP5 & Cold-start & High & Medium & Benefits from OP1 \\
OP7 & Multi-modal fusion & High & Medium & Benefits from OP1 \\
OP8 & Adversarial robustness & High & Low--Med & Requires OP1 for evaluation \\
\bottomrule
\end{tabular}
\end{center}
}
\bottomnote{Source: Paper Table 6 / Section 6.4}
\end{frame}

% ----------------------------------------------------------
% SLIDE 18 -- Dependencies: OP1 and OP4
% ----------------------------------------------------------
\begin{frame}{Critical Preconditions: OP1 and OP4}
\begin{columns}[T]
\column{0.48\textwidth}
\begin{block}{OP1: Benchmark Datasets}
\textbf{Foundational} because without a shared evaluation resource:
\begin{itemize}\compactlist
\item Field cannot conduct \textbf{reproducible comparisons}
\item Progress on nearly every other problem is impeded:
  \begin{itemize}\compactlist
  \item OP4: class imbalance methods
  \item OP6: drift detection
  \item OP7: fusion architectures
  \item OP8: adversarial robustness
  \end{itemize}
\item Cannot benchmark against common data
\end{itemize}
\end{block}

\column{0.48\textwidth}
\begin{block}{OP4: Extreme Class Imbalance}
\textbf{Equally critical} because:
\begin{itemize}\compactlist
\item Small-$N$, heterogeneous-positive-class nature \textbf{invalidates standard supervised learning assumptions}
\item Any advance that does not address this will \textbf{fail to translate} from lab to deployment
\item Only \numfraudcases{} labeled cases across entire historical record
\item Too few cases \textit{per fraud type} for standard training
\end{itemize}
\end{block}
\end{columns}
\vspace{3mm}
\textcolor{mlred}{$\Rightarrow$ Solving OP1 and OP4 \textbf{unlocks} progress across the entire research agenda.}
\bottomnote{Source: Paper Section 6.4}
\end{frame}

% ----------------------------------------------------------
% SLIDE 19 -- Path Forward: Collaboration
% ----------------------------------------------------------
\begin{frame}{Path Forward: Collaboration Required}
\begin{block}{No Single Community Can Solve These Problems Alone}
The most impactful problems (OP2, OP10) require institutional collaboration that no single research group can orchestrate.
\end{block}
\vspace{2mm}
\begin{columns}[T]
\column{0.32\textwidth}
\begin{block}{Regulators Contribute}
\begin{itemize}\compactlist
\item Enforcement-labeled data (OP1)
\item Cross-border relationships (OP2)
\item Operational environments for human-AI evaluation (OP10)
\item Regulatory sandboxes for AI testing
\end{itemize}
\end{block}

\column{0.32\textwidth}
\begin{block}{Academics Contribute}
\begin{itemize}\compactlist
\item Methodological expertise in deep learning
\item Adversarial robustness methods
\item Explainability research
\item Federated learning protocols
\end{itemize}
\end{block}

\column{0.32\textwidth}
\begin{block}{Industry Contributes}
\begin{itemize}\compactlist
\item Domain knowledge of strategy evolution
\item Operational due diligence expertise
\item Practical realities of fund management
\item Red-teaming capabilities
\end{itemize}
\end{block}
\end{columns}
\vspace{2mm}
\textcolor{mlblue}{$\Rightarrow$ Shared data initiatives, regulatory sandboxes, and joint research programs are essential.}
\bottomnote{Source: Paper Section 6.4}
\end{frame}

% ----------------------------------------------------------
% SLIDE 20 -- Summary
% ----------------------------------------------------------
\begin{frame}{Summary: Research Agenda -- 10 Open Problems}
{\footnotesize
\begin{tabular}{l l l l}
\toprule
\textbf{Category} & \textbf{ID} & \textbf{Problem} & \textbf{Key Approach} \\
\midrule
\multirow{3}{*}{Data} & OP1 & Benchmark datasets & Synthetic generation + differential privacy \\
 & OP2 & Cross-jurisdictional integration & Federated learning \\
 & OP3 & Real-time alternative data & Attention-based fusion architectures \\
\midrule
\multirow{4}{*}{Methods} & OP4 & Extreme class imbalance & Few-shot / meta-learning \\
 & OP5 & Cold-start detection & Transfer learning + operational features \\
 & OP6 & Concept drift & ADWIN + regime-switching models \\
 & OP7 & Multi-modal fusion & Multi-modal transformers \\
\midrule
\multirow{3}{*}{Deployment} & OP8 & Adversarial robustness & Certified bounds + game theory \\
 & OP9 & Explainability & Neural additive models + templates \\
 & OP10 & Human-AI collaboration & Active learning + adaptive thresholds \\
\bottomrule
\end{tabular}
}
\vspace{3mm}
\begin{itemize}\compactlist
\item \textbf{OP1} (benchmarks) and \textbf{OP4} (class imbalance) are \textcolor{mlred}{critical preconditions}
\item Progress demands \textbf{collaboration} among regulators, academics, and industry
\end{itemize}
\bottomnote{Source: Paper Section 6 (Contribution C3)}
\end{frame}

\end{document}
