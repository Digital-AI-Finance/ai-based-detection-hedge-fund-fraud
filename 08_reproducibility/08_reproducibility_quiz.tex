\documentclass[8pt,aspectratio=169]{beamer}
\usetheme{Madrid}
\usepackage{graphicx}
\usepackage{booktabs}
\usepackage{adjustbox}
\usepackage{multicol}
\usepackage{amsmath}
\usepackage{amssymb}

% Color definitions
\definecolor{mlblue}{RGB}{0,102,204}
\definecolor{mlpurple}{RGB}{51,51,178}
\definecolor{mllavender}{RGB}{173,173,224}
\definecolor{mllavender2}{RGB}{193,193,232}
\definecolor{mllavender3}{RGB}{204,204,235}
\definecolor{mllavender4}{RGB}{214,214,239}
\definecolor{mlorange}{RGB}{255, 127, 14}
\definecolor{mlgreen}{RGB}{44, 160, 44}
\definecolor{mlred}{RGB}{214, 39, 40}
\definecolor{mlgray}{RGB}{127, 127, 127}

% Additional colors
\definecolor{lightgray}{RGB}{240, 240, 240}
\definecolor{midgray}{RGB}{180, 180, 180}

% Apply custom colors to Madrid theme
\setbeamercolor{palette primary}{bg=mllavender3,fg=mlpurple}
\setbeamercolor{palette secondary}{bg=mllavender2,fg=mlpurple}
\setbeamercolor{palette tertiary}{bg=mllavender,fg=white}
\setbeamercolor{palette quaternary}{bg=mlpurple,fg=white}

\setbeamercolor{structure}{fg=mlpurple}
\setbeamercolor{section in toc}{fg=mlpurple}
\setbeamercolor{subsection in toc}{fg=mlblue}
\setbeamercolor{title}{fg=mlpurple}
\setbeamercolor{frametitle}{fg=mlpurple,bg=mllavender3}
\setbeamercolor{block title}{bg=mllavender2,fg=mlpurple}
\setbeamercolor{block body}{bg=mllavender4,fg=black}

% Remove navigation symbols
\setbeamertemplate{navigation symbols}{}

% Clean itemize/enumerate
\setbeamertemplate{itemize items}[circle]
\setbeamertemplate{enumerate items}[default]

% Reduce margins for more content space
\setbeamersize{text margin left=5mm,text margin right=5mm}

% Command for bottom annotation
\newcommand{\bottomnote}[1]{%
\vfill
\vspace{-2mm}
\textcolor{mllavender2}{\rule{\textwidth}{0.4pt}}
\vspace{1mm}
\footnotesize
\textbf{#1}
}

% Command for compact list spacing
\newcommand{\compactlist}{%
\setlength{\itemsep}{0pt}%
\setlength{\parskip}{0pt}%
\setlength{\parsep}{0pt}%
}

% Notation macros
% Shared notation macros for AI-Based Detection of Hedge Fund Fraud
% Include this file in all Beamer slide decks via % Shared notation macros for AI-Based Detection of Hedge Fund Fraud
% Include this file in all Beamer slide decks via % Shared notation macros for AI-Based Detection of Hedge Fund Fraud
% Include this file in all Beamer slide decks via \input{notation}

% Performance metrics
\newcommand{\auc}{\ensuremath{\mathrm{AUC}}}
\newcommand{\fone}{\ensuremath{F_1}}
\newcommand{\shap}{\text{SHAP}}
\newcommand{\lime}{\text{LIME}}

% Autocorrelation
\newcommand{\rhoone}{\ensuremath{\rho_1}}

% Financial abbreviations
\newcommand{\nav}{\text{NAV}}
\newcommand{\aum}{\text{AUM}}

% Regulatory
\newcommand{\euaiact}{EU AI Act}
\newcommand{\aifmd}{AIFMD}
\newcommand{\sec}{SEC}
\newcommand{\edgar}{EDGAR}

% Key numbers from the paper (for consistency)
\newcommand{\aumdollar}{\$4.5\text{ trillion}}
\newcommand{\aucdeg}{10.6\%}
\newcommand{\numop}{10}
\newcommand{\numfraudcases}{50\text{--}100}
\newcommand{\systematicpapers}{105}


% Performance metrics
\newcommand{\auc}{\ensuremath{\mathrm{AUC}}}
\newcommand{\fone}{\ensuremath{F_1}}
\newcommand{\shap}{\text{SHAP}}
\newcommand{\lime}{\text{LIME}}

% Autocorrelation
\newcommand{\rhoone}{\ensuremath{\rho_1}}

% Financial abbreviations
\newcommand{\nav}{\text{NAV}}
\newcommand{\aum}{\text{AUM}}

% Regulatory
\newcommand{\euaiact}{EU AI Act}
\newcommand{\aifmd}{AIFMD}
\newcommand{\sec}{SEC}
\newcommand{\edgar}{EDGAR}

% Key numbers from the paper (for consistency)
\newcommand{\aumdollar}{\$4.5\text{ trillion}}
\newcommand{\aucdeg}{10.6\%}
\newcommand{\numop}{10}
\newcommand{\numfraudcases}{50\text{--}100}
\newcommand{\systematicpapers}{105}


% Performance metrics
\newcommand{\auc}{\ensuremath{\mathrm{AUC}}}
\newcommand{\fone}{\ensuremath{F_1}}
\newcommand{\shap}{\text{SHAP}}
\newcommand{\lime}{\text{LIME}}

% Autocorrelation
\newcommand{\rhoone}{\ensuremath{\rho_1}}

% Financial abbreviations
\newcommand{\nav}{\text{NAV}}
\newcommand{\aum}{\text{AUM}}

% Regulatory
\newcommand{\euaiact}{EU AI Act}
\newcommand{\aifmd}{AIFMD}
\newcommand{\sec}{SEC}
\newcommand{\edgar}{EDGAR}

% Key numbers from the paper (for consistency)
\newcommand{\aumdollar}{\$4.5\text{ trillion}}
\newcommand{\aucdeg}{10.6\%}
\newcommand{\numop}{10}
\newcommand{\numfraudcases}{50\text{--}100}
\newcommand{\systematicpapers}{105}


\title{Quiz: Reproducibility}
\subtitle{AI-Based Detection of Hedge Fund Fraud}
\author{Joerg Osterrieder}
\institute{Zurich University of Applied Sciences (ZHAW)}
\date{2025}

\begin{document}

\begin{frame}
\titlepage
\end{frame}

\begin{frame}{Question 1}
How many databases were searched for the systematic review?
\begin{itemize}\compactlist
\item[a)] 3
\item[b)] 4
\item[c)] 5
\item[d)] 7
\end{itemize}

\pause
\begin{block}{Answer}
\textbf{c)} 5\\[2mm]
The systematic review searched five databases: Web of Science, Scopus, IEEE Xplore, PubMed, and arXiv. This multi-database approach ensures comprehensive coverage of academic literature.
\end{block}
\bottomnote{Section 8: Reproducibility}
\end{frame}

\begin{frame}{Question 2}
What does SALSA stand for in the review methodology?
\begin{itemize}\compactlist
\item[a)] Systematic Analysis of Literature and Scientific Assessment
\item[b)] Search, AppraisaL, Synthesis, Analysis
\item[c)] Statistical Analysis for Literature Survey Approach
\item[d)] Structured Assessment of Literature Sources and Articles
\end{itemize}

\pause
\begin{block}{Answer}
\textbf{b)} Search, AppraisaL, Synthesis, Analysis\\[2mm]
The SALSA framework structures the systematic review into four phases: Search (identification), AppraisaL (quality assessment), Synthesis (data extraction), and Analysis (findings integration).
\end{block}
\bottomnote{Section 8: Review Methodology}
\end{frame}

\begin{frame}{Question 3}
How many initial papers were found in the database searches?
\begin{itemize}\compactlist
\item[a)] $\sim$100
\item[b)] $\sim$200
\item[c)] $\sim$350
\item[d)] $\sim$500
\end{itemize}

\pause
\begin{block}{Answer}
\textbf{d)} $\sim$500\\[2mm]
The initial database search identified approximately 500 papers. After removing duplicates, screening titles/abstracts, and full-text review, the final systematic corpus contained 105 papers.
\end{block}
\bottomnote{Section 8: Search Funnel}
\end{frame}

\begin{frame}{Question 4}
What is the final systematic corpus size after all screening stages?
\begin{itemize}\compactlist
\item[a)] 80
\item[b)] 105
\item[c)] 169
\item[d)] 297
\end{itemize}

\pause
\begin{block}{Answer}
\textbf{b)} 105\\[2mm]
After title/abstract screening, full-text assessment, and quality appraisal, exactly 105 papers were included in the final systematic corpus for detailed analysis and synthesis.
\end{block}
\bottomnote{Section 8: Final Corpus}
\end{frame}

\begin{frame}{Question 5}
What is the primary SEC data access system mentioned?
\begin{itemize}\compactlist
\item[a)] EDGAR
\item[b)] MIDAS
\item[c)] DERA
\item[d)] CRQA
\end{itemize}

\pause
\begin{block}{Answer}
\textbf{a)} EDGAR\\[2mm]
EDGAR (Electronic Data Gathering, Analysis, and Retrieval) is the SEC's primary system for accessing public filings, including Form ADV disclosures from hedge funds and investment advisers.
\end{block}
\bottomnote{Section 8: Data Sources}
\end{frame}

\end{document}
