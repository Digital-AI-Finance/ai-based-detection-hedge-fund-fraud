% ============================================================
%  Slide Deck 2 -- Background: The Hedge Fund Fraud Landscape
%  AI-Based Detection of Hedge Fund Fraud
% ============================================================
\documentclass[8pt,aspectratio=169]{beamer}
\usetheme{Madrid}
\usepackage{graphicx}
\usepackage{booktabs}
\usepackage{adjustbox}
\usepackage{multicol}
\usepackage{amsmath}
\usepackage{amssymb}

% ---- Color definitions ----
\definecolor{mlblue}{RGB}{0,102,204}
\definecolor{mlpurple}{RGB}{51,51,178}
\definecolor{mllavender}{RGB}{173,173,224}
\definecolor{mllavender2}{RGB}{193,193,232}
\definecolor{mllavender3}{RGB}{204,204,235}
\definecolor{mllavender4}{RGB}{214,214,239}
\definecolor{mlorange}{RGB}{255, 127, 14}
\definecolor{mlgreen}{RGB}{44, 160, 44}
\definecolor{mlred}{RGB}{214, 39, 40}
\definecolor{mlgray}{RGB}{127, 127, 127}
\definecolor{lightgray}{RGB}{240, 240, 240}
\definecolor{midgray}{RGB}{180, 180, 180}

% ---- Apply custom colors to Madrid theme ----
\setbeamercolor{palette primary}{bg=mllavender3,fg=mlpurple}
\setbeamercolor{palette secondary}{bg=mllavender2,fg=mlpurple}
\setbeamercolor{palette tertiary}{bg=mllavender,fg=white}
\setbeamercolor{palette quaternary}{bg=mlpurple,fg=white}
\setbeamercolor{structure}{fg=mlpurple}
\setbeamercolor{section in toc}{fg=mlpurple}
\setbeamercolor{subsection in toc}{fg=mlblue}
\setbeamercolor{title}{fg=mlpurple}
\setbeamercolor{frametitle}{fg=mlpurple,bg=mllavender3}
\setbeamercolor{block title}{bg=mllavender2,fg=mlpurple}
\setbeamercolor{block body}{bg=mllavender4,fg=black}

% ---- Navigation / itemize ----
\setbeamertemplate{navigation symbols}{}
\setbeamertemplate{itemize items}[circle]
\setbeamertemplate{enumerate items}[default]
\setbeamersize{text margin left=5mm,text margin right=5mm}

% ---- Custom commands ----
\newcommand{\bottomnote}[1]{%
\vfill
\vspace{-2mm}
\textcolor{mllavender2}{\rule{\textwidth}{0.4pt}}
\vspace{1mm}
\footnotesize
\textbf{#1}
}

\newcommand{\compactlist}{%
\setlength{\itemsep}{0pt}%
\setlength{\parskip}{0pt}%
\setlength{\parsep}{0pt}%
}

\newcommand{\chartplaceholder}[2][5cm]{%
\begin{center}
\begin{adjustbox}{max width=0.95\textwidth, max height=#1}
\framebox[\textwidth][c]{%
\rule{0pt}{#1}%
\textcolor{midgray}{[#2]}%
}
\end{adjustbox}
\end{center}
}

% ---- Notation ----
% Shared notation macros for AI-Based Detection of Hedge Fund Fraud
% Include this file in all Beamer slide decks via % Shared notation macros for AI-Based Detection of Hedge Fund Fraud
% Include this file in all Beamer slide decks via % Shared notation macros for AI-Based Detection of Hedge Fund Fraud
% Include this file in all Beamer slide decks via \input{notation}

% Performance metrics
\newcommand{\auc}{\ensuremath{\mathrm{AUC}}}
\newcommand{\fone}{\ensuremath{F_1}}
\newcommand{\shap}{\text{SHAP}}
\newcommand{\lime}{\text{LIME}}

% Autocorrelation
\newcommand{\rhoone}{\ensuremath{\rho_1}}

% Financial abbreviations
\newcommand{\nav}{\text{NAV}}
\newcommand{\aum}{\text{AUM}}

% Regulatory
\newcommand{\euaiact}{EU AI Act}
\newcommand{\aifmd}{AIFMD}
\newcommand{\sec}{SEC}
\newcommand{\edgar}{EDGAR}

% Key numbers from the paper (for consistency)
\newcommand{\aumdollar}{\$4.5\text{ trillion}}
\newcommand{\aucdeg}{10.6\%}
\newcommand{\numop}{10}
\newcommand{\numfraudcases}{50\text{--}100}
\newcommand{\systematicpapers}{105}


% Performance metrics
\newcommand{\auc}{\ensuremath{\mathrm{AUC}}}
\newcommand{\fone}{\ensuremath{F_1}}
\newcommand{\shap}{\text{SHAP}}
\newcommand{\lime}{\text{LIME}}

% Autocorrelation
\newcommand{\rhoone}{\ensuremath{\rho_1}}

% Financial abbreviations
\newcommand{\nav}{\text{NAV}}
\newcommand{\aum}{\text{AUM}}

% Regulatory
\newcommand{\euaiact}{EU AI Act}
\newcommand{\aifmd}{AIFMD}
\newcommand{\sec}{SEC}
\newcommand{\edgar}{EDGAR}

% Key numbers from the paper (for consistency)
\newcommand{\aumdollar}{\$4.5\text{ trillion}}
\newcommand{\aucdeg}{10.6\%}
\newcommand{\numop}{10}
\newcommand{\numfraudcases}{50\text{--}100}
\newcommand{\systematicpapers}{105}


% Performance metrics
\newcommand{\auc}{\ensuremath{\mathrm{AUC}}}
\newcommand{\fone}{\ensuremath{F_1}}
\newcommand{\shap}{\text{SHAP}}
\newcommand{\lime}{\text{LIME}}

% Autocorrelation
\newcommand{\rhoone}{\ensuremath{\rho_1}}

% Financial abbreviations
\newcommand{\nav}{\text{NAV}}
\newcommand{\aum}{\text{AUM}}

% Regulatory
\newcommand{\euaiact}{EU AI Act}
\newcommand{\aifmd}{AIFMD}
\newcommand{\sec}{SEC}
\newcommand{\edgar}{EDGAR}

% Key numbers from the paper (for consistency)
\newcommand{\aumdollar}{\$4.5\text{ trillion}}
\newcommand{\aucdeg}{10.6\%}
\newcommand{\numop}{10}
\newcommand{\numfraudcases}{50\text{--}100}
\newcommand{\systematicpapers}{105}


% ---- Title metadata ----
\title{Background: The Hedge Fund Fraud Landscape}
\subtitle{Section 2 -- AI-Based Detection of Hedge Fund Fraud}
\author{Joerg Osterrieder}
\institute{Zurich University of Applied Sciences (ZHAW)}
\date{2025}

% ============================================================
\begin{document}

% ----------------------------------------------------------
% SLIDE 1 -- Title
% ----------------------------------------------------------
\begin{frame}
\titlepage
\end{frame}

% ----------------------------------------------------------
% SLIDE 2 -- Outline
% ----------------------------------------------------------
\begin{frame}{Outline}
\begin{enumerate}\compactlist
\item Fraud Taxonomy Overview (5 types, difficulty 1--5)
\item Performance Fabrication
\item Allocation Fraud
\item Strategy Misrepresentation
\item Market Manipulation
\item Regulatory Fraud
\item Data Ecosystem: Four Layers
\item Return Data and Database Biases
\item Regulatory Filings (EDGAR, Forms ADV/D/13F)
\item Alternative Data
\item Synthetic Data
\item Regulatory Context: United States
\item Regulatory Context: European Union
\item Supervisory Technology (SupTech)
\item Section Summary and Key Takeaways
\end{enumerate}
\end{frame}

% ----------------------------------------------------------
% SLIDE 3 -- Fraud Taxonomy Overview
% ----------------------------------------------------------
\begin{frame}{Fraud Taxonomy Overview}
\begin{center}
\small
\begin{tabular}{lccp{5.5cm}}
\toprule
\textbf{Fraud Type} & \textbf{Difficulty} & \textbf{Key Case} & \textbf{Observable Signals} \\
\midrule
Performance fabrication & 3/5 & Madoff (2008) & Serial correlation, Benford violations, implausible Sharpe ratios \\
Allocation fraud & 4/5 & Petters (2008) & Cross-account return dispersion, win-rate asymmetry \\
Strategy misrepresentation & 3/5 & Platinum Partners (2016) & Style drift, factor exposure shifts, textual inconsistencies \\
Market manipulation & 5/5 & SAC Capital (2013) & Order-flow anomalies, network centrality, timing patterns \\
Regulatory fraud & 2/5 & Lancer Mgmt.\ (2003) & Filing inconsistencies, text anomalies, omission detection \\
\bottomrule
\end{tabular}
\end{center}
\vspace{2mm}
\begin{itemize}\compactlist
\item Difficulty scale: 1 (straightforward with available data) to 5 (requires privileged real-time data)
\item Ordered from most frequently studied to most difficult to detect
\end{itemize}
\bottomnote{Source: Paper Table 1; Section 2.1}
\end{frame}

% ----------------------------------------------------------
% SLIDE 4 -- Performance Fabrication
% ----------------------------------------------------------
\begin{frame}{Performance Fabrication (Difficulty 3/5)}
\begin{columns}[T]
\column{0.55\textwidth}
\textbf{Definition}
\begin{itemize}\compactlist
\item Deliberate misstatement of investment returns
\item Forms: Ponzi schemes, return smoothing, \nav{} manipulation
\item Inflating reported value of illiquid positions (OTC derivatives, distressed debt, private placements)
\end{itemize}
\vspace{2mm}
\textbf{Paradigmatic Case: Madoff}
\begin{itemize}\compactlist
\item Fabricated returns for $\geq$20 years
\item \textbf{Only 7 losing months} across 14 years
\item Nearly perfect 45-degree equity curve
\item Sharpe ratio exceeded plausible bounds
\end{itemize}

\column{0.42\textwidth}
\textbf{Detection Signals}
\begin{itemize}\compactlist
\item Serial correlation $\rhoone$ anomalies
\item Benford's law violations
\item Distributional ``kink'' at zero (Bollen \& Pool, 2012): $\sim$8\% of TASS funds flagged
\item Implausibly high Sharpe ratios
\item Low return volatility relative to stated strategy
\end{itemize}
\vspace{2mm}
\textcolor{mlblue}{Individual signals are noisy, but their \textbf{combination} via ML offers substantially improved discriminatory power}
\end{columns}
\bottomnote{Source: Gregoriou (2009); Markopolos (2010); Bollen \& Pool (2012); paper Section 2.1.1}
\end{frame}

% ----------------------------------------------------------
% SLIDE 5 -- Performance Fabrication Detection Signals
% ----------------------------------------------------------
\begin{frame}{Performance Fabrication: Statistical Red Flags}
\begin{columns}[T]
\column{0.48\textwidth}
\textbf{Getmansky et al.\ (2004)}
\begin{itemize}\compactlist
\item Econometric model: serial correlation from managed pricing of illiquid assets
\item Typical $\rhoone = 0.3$--$0.5$ for illiquid positions
\item \textbf{Abnormally high} $\rhoone$ for funds claiming liquid assets $\Rightarrow$ NAV manipulation signal
\end{itemize}
\vspace{2mm}
\textbf{Bollen \& Pool (2012)}
\begin{itemize}\compactlist
\item Distributional discontinuity at zero: excess small positive returns, deficit of small negatives
\item Correctly identified $\sim$50\% of funds that subsequently faced SEC enforcement
\end{itemize}

\column{0.48\textwidth}
\textbf{Benford's Law}
\begin{itemize}\compactlist
\item Tests leading-digit frequency: $P(d) = \log_{10}(1 + 1/d)$
\item Applied retroactively to Madoff: anomalies in 9/10 tests
\item Limited power for short histories ($<60$ months)
\item Defeated by knowledgeable fraudster who engineers digit conformity
\end{itemize}
\vspace{2mm}
\textbf{Key Insight}
\begin{itemize}\compactlist
\item Each test captures \textit{one} dimension
\item \textcolor{mlblue}{ML classifiers combine signals} for multi-dimensional detection
\end{itemize}
\end{columns}
\bottomnote{Source: Nigrini (2012); Getmansky et al.\ (2004); Bollen \& Pool (2012); paper Section 2.1.1}
\end{frame}

% ----------------------------------------------------------
% SLIDE 6 -- Allocation Fraud
% ----------------------------------------------------------
\begin{frame}{Allocation Fraud (Difficulty 4/5)}
\begin{columns}[T]
\column{0.55\textwidth}
\textbf{Definition}
\begin{itemize}\compactlist
\item Systematically directing profitable trades to \textbf{favored accounts} (proprietary, co-investment)
\item Routing losing trades to \textbf{client accounts}
\item Cherry-picking: delaying allocation until daily P\&L known
\end{itemize}
\vspace{2mm}
\textbf{SEC Evidence}
\begin{itemize}\compactlist
\item Favored accounts: \textbf{91\%} profitable trade allocations
\item Client accounts: only \textbf{31\%}
\item Disparity cannot arise by chance
\end{itemize}

\column{0.42\textwidth}
\textbf{Detection Challenges}
\begin{itemize}\compactlist
\item Requires \textbf{trade-level data} (order timestamps, fill assignments)
\item Rarely available in public databases
\item HFR/TASS report only monthly fund-level returns
\item Intra-fund allocation patterns entirely obscured
\end{itemize}
\vspace{2mm}
\textbf{Possible Approaches}
\begin{itemize}\compactlist
\item Cross-account return dispersion analysis
\item Win-rate asymmetry statistics
\item Network-based adviser--account mapping from filings (largely unexplored)
\end{itemize}
\end{columns}
\bottomnote{Source: SEC enforcement data; paper Section 2.1.2}
\end{frame}

% ----------------------------------------------------------
% SLIDE 7 -- Strategy Misrepresentation
% ----------------------------------------------------------
\begin{frame}{Strategy Misrepresentation (Difficulty 3/5)}
\begin{columns}[T]
\column{0.52\textwidth}
\textbf{Definition}
\begin{itemize}\compactlist
\item Actual investment behavior diverges materially from stated strategy without adequate disclosure
\item Includes:
  \begin{itemize}\compactlist
  \item Undisclosed \textbf{style drift} (e.g., equity L/S $\to$ illiquid credit)
  \item Leverage misreporting
  \item \textbf{AI-washing}: falsely claiming AI-driven decisions
  \end{itemize}
\end{itemize}
\vspace{2mm}
\textbf{Key Case: Platinum Partners (2016)}

\column{0.45\textwidth}
\textbf{Detection Methods}
\begin{itemize}\compactlist
\item \textbf{Change-point detection} (Patton \& Ramadorai, 2015): structural breaks in risk exposures often precede fund failure
\item Rolling-window factor regressions $\Rightarrow$ style drift detection
\item \textbf{NLP on Form ADV}: compare stated strategy descriptions vs.\ quantitative factor exposures
\item Cross-modal consistency checking: text vs.\ numbers
\end{itemize}
\vspace{2mm}
\textbf{Challenge}: distinguishing intentional misrepresentation from legitimate adaptive portfolio management
\end{columns}
\bottomnote{Source: Patton \& Ramadorai (2015); Fung \& Hsieh (2001); paper Section 2.1.3}
\end{frame}

% ----------------------------------------------------------
% SLIDE 8 -- Market Manipulation
% ----------------------------------------------------------
\begin{frame}{Market Manipulation (Difficulty 5/5)}
\begin{columns}[T]
\column{0.52\textwidth}
\textbf{Forms}
\begin{itemize}\compactlist
\item \textbf{Front-running} client orders
\item \textbf{Insider trading} (incl.\ ``shadow trading'' on economically related securities)
\item \textbf{Spoofing}: large orders placed and rapidly cancelled
\end{itemize}
\vspace{2mm}
\textbf{Key Case: SAC Capital (2013)}
\begin{itemize}\compactlist
\item Guilty plea to insider trading charges
\item \textbf{\$1.8 billion} penalty
\end{itemize}

\column{0.45\textwidth}
\textbf{Why Most Difficult}
\begin{itemize}\compactlist
\item Requires \textbf{real-time, tick-level} trade and order-book data
\item Not available in standard hedge fund databases
\item Detection needs:
  \begin{itemize}\compactlist
  \item Network analysis of communication patterns
  \item Temporal analysis of orders vs.\ MNPI events
  \item Cross-market surveillance for shadow trading
  \end{itemize}
\end{itemize}
\vspace{2mm}
\textbf{SEC MIDAS}
\begin{itemize}\compactlist
\item Processes $\sim$1 billion records/day
\item Academic research constrained by data access
\end{itemize}
\end{columns}
\bottomnote{Source: Lewis (2012); SEC MIDAS; paper Section 2.1.4}
\end{frame}

% ----------------------------------------------------------
% SLIDE 9 -- Regulatory Fraud
% ----------------------------------------------------------
\begin{frame}{Regulatory Fraud (Difficulty 2/5)}
\begin{columns}[T]
\column{0.52\textwidth}
\textbf{Definition}
\begin{itemize}\compactlist
\item Materially false or misleading information in mandatory filings
\item Key filings:
  \begin{itemize}\compactlist
  \item \textbf{Form ADV}: uniform registration (Investment Advisers Act)
  \item \textbf{Form D}: Reg D offering notices
  \item \textbf{Form 13F}: quarterly holdings (\$100M+ managers)
  \end{itemize}
\item Ranges from deliberate misstatements to material omissions (disciplinary history, conflicts)
\end{itemize}

\column{0.45\textwidth}
\textbf{Why Most Tractable}
\begin{itemize}\compactlist
\item Data are \textbf{structured / semi-structured}
\item \textbf{Publicly accessible} via \edgar{}
\item Amenable to traditional text analysis \textit{and} modern NLP
\item Cross-referencing filings with external data (e.g., AUM vs.\ implied fund flows)
\end{itemize}
\vspace{2mm}
\textbf{Key Results}
\begin{itemize}\compactlist
\item Dimmock \& Gerken (2012): Form ADV predicts SEC enforcement actions
\item Brown et al.\ (2008): filing data contain fraud-relevant signals complementing return-based tests
\end{itemize}
\end{columns}
\bottomnote{Source: Dimmock \& Gerken (2012); Brown et al.\ (2008); paper Section 2.1.5}
\end{frame}

% ----------------------------------------------------------
% SLIDE 10 -- Fraud Taxonomy Chart
% ----------------------------------------------------------
\begin{frame}{Fraud Taxonomy: Visual Summary}

\chartplaceholder[5.5cm]{Chart: 01\_fraud\_taxonomy -- Radar or matrix chart showing five fraud types with detection difficulty, data requirements, and AI method suitability}

\bottomnote{Source: Paper Table 1; Section 2.1}
\end{frame}

% ----------------------------------------------------------
% SLIDE 11 -- Data Ecosystem Overview
% ----------------------------------------------------------
\begin{frame}{Data Ecosystem: Four Layers}
\begin{columns}[T]
\column{0.48\textwidth}
\begin{block}{Layer 1: Return Data}
\begin{itemize}\compactlist
\item Lipper TASS, HFR, BarclayHedge, Morningstar
\item Monthly return series, self-reported characteristics
\item 7,000+ live and defunct funds in TASS
\end{itemize}
\end{block}
\begin{block}{Layer 2: Regulatory Filings}
\begin{itemize}\compactlist
\item SEC EDGAR: Forms ADV, D, 13F
\item Structured (XML) + free-text (PDF)
\item Post-Dodd-Frank: \$150M+ must register
\end{itemize}
\end{block}

\column{0.48\textwidth}
\begin{block}{Layer 3: Alternative Data}
\begin{itemize}\compactlist
\item News/social media sentiment, satellite imagery, web traffic, litigation records
\item Market: \textbf{\$7.5B} (2023), projected \$273B by 2032
\end{itemize}
\end{block}
\begin{block}{Layer 4: Synthetic Data}
\begin{itemize}\compactlist
\item Addresses extreme class imbalance
\item SMOTE, GANs, VAEs
\item Privacy-preserving generation for cross-institutional collaboration
\end{itemize}
\end{block}
\end{columns}
\vspace{2mm}
\textcolor{mlblue}{Detection effectiveness is bounded by data quality, coverage, and granularity}
\bottomnote{Source: Paper Section 2.2}
\end{frame}

% ----------------------------------------------------------
% SLIDE 12 -- Return Data Biases
% ----------------------------------------------------------
\begin{frame}{Return Data: Three Critical Biases}
\begin{columns}[T]
\column{0.32\textwidth}
\begin{block}{Survivorship Bias}
\begin{itemize}\compactlist
\item Defunct funds exit live databases
\item Overstates average returns by $\sim$\textbf{+242 bp/year}
\item Fung \& Hsieh (2009)
\end{itemize}
\end{block}

\column{0.32\textwidth}
\begin{block}{Backfill Bias}
\begin{itemize}\compactlist
\item Retroactive submission of favorable pre-reporting history
\item Overstates returns by $\sim$\textbf{+442 bp/year}
\item Also called ``instant history bias''
\end{itemize}
\end{block}

\column{0.32\textwidth}
\begin{block}{Selection Bias}
\begin{itemize}\compactlist
\item Voluntary reporting
\item Strong track records more likely to report
\item Distressed/fraudulent funds may stop reporting before detection
\end{itemize}
\end{block}
\end{columns}
\vspace{3mm}
\textcolor{mlred}{\textbf{Pernicious asymmetry for fraud detection:}} detected frauds enter graveyard; undetected frauds remain in live data, contaminating the ``clean'' training class.
\bottomnote{Source: Fung \& Hsieh (2009); Agarwal et al.\ (2011); paper Section 2.2.1}
\end{frame}

% ----------------------------------------------------------
% SLIDE 13 -- Regulatory Filings
% ----------------------------------------------------------
\begin{frame}{Regulatory Filings (EDGAR)}
\begin{columns}[T]
\column{0.48\textwidth}
\textbf{Key Filing Types}
\begin{itemize}\compactlist
\item \textbf{Form ADV Part 1}: structured XML via IARD
  \begin{itemize}\compactlist
  \item Business, ownership, clients, disciplinary history
  \end{itemize}
\item \textbf{Form ADV Part 2} (``brochure''): free-text PDF
  \begin{itemize}\compactlist
  \item Strategies, fees, risk factors, conflicts
  \item No standardized structure
  \end{itemize}
\item \textbf{Form 13F}: quarterly equity holdings (\$100M+ managers)
  \begin{itemize}\compactlist
  \item Machine-readable but contains known errors
  \end{itemize}
\end{itemize}

\column{0.48\textwidth}
\textbf{Practical Challenges}
\begin{itemize}\compactlist
\item Merging data across filing types
\item Linking to commercial return databases (TASS $\leftrightarrow$ SEC CRD numbers)
\item Substantial data engineering effort
\end{itemize}
\vspace{2mm}
\textbf{Value for Fraud Detection}
\begin{itemize}\compactlist
\item Cross-validation: return-based flags vs.\ filing-derived signals
\item Auditor changes, custody arrangements, disciplinary histories
\item Post-Dodd-Frank: dramatically expanded disclosure universe
\end{itemize}
\end{columns}
\bottomnote{Source: Brown et al.\ (2008); Dimmock \& Gerken (2012); paper Section 2.2.2}
\end{frame}

% ----------------------------------------------------------
% SLIDE 14 -- Alternative Data
% ----------------------------------------------------------
\begin{frame}{Alternative Data}
\begin{columns}[T]
\column{0.52\textwidth}
\textbf{Sources}
\begin{itemize}\compactlist
\item News and social media sentiment
\item Satellite imagery and geolocation data
\item Web traffic analytics
\item Patent filings and litigation records
\end{itemize}
\vspace{2mm}
\textbf{Market Size}
\begin{itemize}\compactlist
\item $\sim$\textbf{\$7.5 billion} (2023)
\item Projected \textbf{\$273 billion} by 2032
\item Driven by adoption among investment managers and regulators
\end{itemize}

\column{0.45\textwidth}
\textbf{Fraud Detection Uses}
\begin{itemize}\compactlist
\item Flag reputational signals \textit{before} regulatory actions
\item Independently verify economic claims (e.g., satellite foot traffic vs.\ reported performance)
\item Triangulate plausibility of stated strategies/returns
\end{itemize}
\vspace{2mm}
\textbf{Risks}
\begin{itemize}\compactlist
\item Sentiment signals are noisy, susceptible to manipulation
\item Acquisition/processing costs are substantial
\item Privacy concerns (intersect with \euaiact{})
\item Indirect relationship to fund-level fraud signals
\end{itemize}
\end{columns}
\bottomnote{Source: Paper Section 2.2.3}
\end{frame}

% ----------------------------------------------------------
% SLIDE 15 -- Synthetic Data
% ----------------------------------------------------------
\begin{frame}{Synthetic Data for Class Imbalance}
\begin{columns}[T]
\column{0.48\textwidth}
\textbf{The Problem}
\begin{itemize}\compactlist
\item Extreme class imbalance: confirmed frauds = small fraction of total
\item \numfraudcases{} confirmed cases vs.\ 10,000+ funds
\item Difficult to train supervised classifiers
\end{itemize}
\vspace{2mm}
\textbf{Methods}
\begin{itemize}\compactlist
\item \textbf{SMOTE}: interpolation between existing positive examples (most widely used)
\item \textbf{GANs}: learn distributional properties of known fraudulent funds
\item \textbf{VAEs}: probabilistic generation with calibrated uncertainty
\end{itemize}

\column{0.48\textwidth}
\textbf{Validation Challenges}
\begin{itemize}\compactlist
\item Must preserve statistical dependencies and temporal dynamics
\item Cannot amplify artifacts of training data
\item Generative models $>$ interpolation for realism and diversity
\end{itemize}
\vspace{2mm}
\textbf{Privacy-Preserving Generation}
\begin{itemize}\compactlist
\item Regulators cannot share enforcement-labelled data
\item Differentially private generative models could enable synthetic dataset release
\item Preserves aggregate statistics while protecting identities
\item Remains largely aspirational in hedge fund domain
\end{itemize}
\end{columns}
\bottomnote{Source: Chawla et al.\ (2002); paper Section 2.2.4}
\end{frame}

% ----------------------------------------------------------
% SLIDE 16 -- Data Ecosystem Chart
% ----------------------------------------------------------
\begin{frame}{Data Ecosystem: Visual Summary}

\chartplaceholder[5.5cm]{Chart: 02\_data\_ecosystem -- Four-layer data architecture diagram showing Return Data, Regulatory Filings, Alternative Data, and Synthetic Data with coverage, frequency, and fraud-type mappings}

\bottomnote{Source: Paper Section 2.2}
\end{frame}

% ----------------------------------------------------------
% SLIDE 17 -- US Regulatory Framework
% ----------------------------------------------------------
\begin{frame}{Regulatory Context: United States}
\begin{columns}[T]
\column{0.48\textwidth}
\textbf{Dodd-Frank Act (2010)}
\begin{itemize}\compactlist
\item Eliminated ``private adviser exemption''
\item Advisers with \aum{} $\geq$\$150M must register with SEC
\item Dramatically expanded universe of funds subject to disclosure
\item Created filing data underpinning many detection approaches
\end{itemize}
\vspace{2mm}
\textbf{SEC Divisions}
\begin{itemize}\compactlist
\item \textbf{DERA} (est.\ 2009): quantitative analysis for enforcement \& rulemaking
\item \textbf{MIDAS} (since 2013): $\sim$1B records/day from all equity exchanges
\item \textbf{CRQA}: trading pattern databases from past enforcement, prioritizes future exams
\end{itemize}

\column{0.48\textwidth}
\textbf{Whistleblower Program}
\begin{itemize}\compactlist
\item Established under Dodd-Frank \S922
\item Response to failure to act on Markopolos warnings
\item Over \textbf{\$1.5 billion} awarded to whistleblowers
\item Thousands of tips complementing algorithmic detection
\end{itemize}
\vspace{2mm}
\textbf{Hybrid Detection Paradigm}
\begin{itemize}\compactlist
\item Human intelligence (tips) $+$ machine intelligence (quantitative screening)
\item \textcolor{mlblue}{Underexplored synergy between these two channels}
\end{itemize}
\end{columns}
\bottomnote{Source: Brown et al.\ (2008); SEC (2023); paper Section 2.3.1}
\end{frame}

% ----------------------------------------------------------
% SLIDE 18 -- EU Regulatory Framework
% ----------------------------------------------------------
\begin{frame}{Regulatory Context: European Union}
\begin{columns}[T]
\column{0.48\textwidth}
\textbf{\aifmd{} (2011)}
\begin{itemize}\compactlist
\item Harmonized framework for alternative investment fund managers
\item Reporting obligations, leverage limits, investor disclosure
\item Structured data analogous to US Form ADV regime
\end{itemize}
\vspace{2mm}
\textbf{\euaiact{} (Regulation 2024/1689)}
\begin{itemize}\compactlist
\item Entered into force August 2024
\item AI for fraud detection classified as \textbf{high-risk}
\item Mandatory requirements:
  \begin{itemize}\compactlist
  \item Risk management systems
  \item Data governance standards
  \item Technical documentation
  \item Human oversight
  \item Transparency obligations (Art.\ 13)
  \end{itemize}
\end{itemize}

\column{0.48\textwidth}
\textbf{Implications for Model Selection}
\begin{itemize}\compactlist
\item Art.\ 13: outputs must be ``sufficiently transparent to enable deployers to interpret''
\item \textcolor{mlred}{Opaque models} (deep neural networks) may require post-hoc explainability (\shap{}, \lime{})
\item \textcolor{mlgreen}{Interpretable models} (logistic regression, decision trees) may be preferred despite potentially lower performance
\end{itemize}
\vspace{2mm}
\textbf{Accuracy vs.\ Explainability Tension}
\begin{itemize}\compactlist
\item Detection system that cannot articulate basis for suspicion $\Rightarrow$ limited regulatory use
\item EU AI Act codifies this intuition into law
\item Similar requirements likely to emerge globally
\end{itemize}
\end{columns}
\bottomnote{Source: EU AI Act (2024); paper Section 2.3.2}
\end{frame}

% ----------------------------------------------------------
% SLIDE 19 -- SupTech
% ----------------------------------------------------------
\begin{frame}{Supervisory Technology (SupTech)}
\begin{columns}[T]
\column{0.52\textwidth}
\textbf{Definition}
\begin{itemize}\compactlist
\item Advanced analytics and AI used by \textbf{financial regulators and central banks} to enhance oversight
\item Shift from \textit{reactive enforcement} (investigate after losses) to \textit{proactive surveillance} (detect anomalies before escalation)
\end{itemize}
\vspace{2mm}
\textbf{Adopters}
\begin{itemize}\compactlist
\item SEC (DERA, CRQA, MIDAS)
\item Bank of England
\item Monetary Authority of Singapore
\item De Nederlandsche Bank
\end{itemize}

\column{0.45\textwidth}
\textbf{Approach}
\begin{itemize}\compactlist
\item Establish baseline behavioral profiles for regulated entities
\item Flag deviations exceeding statistical thresholds
\item Conceptually similar to Bollen \& Pool (2012) but at institutional scale
\end{itemize}
\vspace{2mm}
\textbf{Challenges}
\begin{itemize}\compactlist
\item \textcolor{mlred}{False positives} consume scarce examination resources
\item Sophisticated fraudsters may reverse-engineer detection criteria
\item Shortage of staff with combined regulatory + ML expertise
\item \textbf{Cross-border coordination} limited: Cayman Islands registration, NY management, EU investors, Asian venues
\end{itemize}
\end{columns}
\bottomnote{Source: FSB (2017); BIS (2024); paper Section 2.3.3}
\end{frame}

% ----------------------------------------------------------
% SLIDE 20 -- Regulatory Comparison Chart
% ----------------------------------------------------------
\begin{frame}{Regulatory Landscape: Comparison}

\chartplaceholder[5.5cm]{Chart: 03\_regulatory\_landscape -- Side-by-side comparison of US (Dodd-Frank, SEC/DERA/MIDAS) vs.\ EU (AIFMD, EU AI Act) regulatory frameworks with SupTech adoption indicators}

\bottomnote{Source: Paper Section 2.3}
\end{frame}

% ----------------------------------------------------------
% SLIDE 21 -- Section Summary
% ----------------------------------------------------------
\begin{frame}{Section Summary}
\begin{columns}[T]
\column{0.48\textwidth}
\textbf{Fraud Taxonomy}
\begin{itemize}\compactlist
\item 5 types: performance fabrication, allocation fraud, strategy misrepresentation, market manipulation, regulatory fraud
\item Difficulty ranges from 2/5 (regulatory) to 5/5 (market manipulation)
\item Each type requires different data and detection approaches
\end{itemize}
\vspace{2mm}
\textbf{Data Ecosystem}
\begin{itemize}\compactlist
\item 4 layers: return data, regulatory filings, alternative data, synthetic data
\item Severe biases: survivorship (+242 bp), backfill (+442 bp), selection
\end{itemize}

\column{0.48\textwidth}
\textbf{Regulatory Context}
\begin{itemize}\compactlist
\item US: Dodd-Frank expanded disclosure; SEC investing in computational enforcement
\item EU: AIFMD for fund oversight; EU AI Act imposes high-risk requirements on detection AI
\item SupTech: proactive surveillance emerging but faces cross-border and capacity challenges
\end{itemize}
\vspace{2mm}
\textcolor{mlblue}{\textbf{These foundations establish \textit{what} must be detected, \textit{which} data are available, and \textit{what constraints} govern AI systems.}}
\end{columns}
\bottomnote{Source: Paper Section 2}
\end{frame}

% ----------------------------------------------------------
% SLIDE 22 -- Key Takeaways
% ----------------------------------------------------------
\begin{frame}{Key Takeaways}
\begin{enumerate}\compactlist
\item Hedge fund fraud spans \textbf{five distinct types} with fundamentally different data requirements and detection difficulty
\item The data ecosystem is \textbf{rich but biased} -- survivorship, backfill, and selection biases directly affect model training
\item Post-Dodd-Frank regulatory filings create a \textbf{valuable structured data source} that did not exist before 2012
\item The \textbf{alternative data revolution} (\$7.5B market) opens new detection avenues but introduces noise and privacy concerns
\item \textbf{Extreme class imbalance} (\numfraudcases{} confirmed cases) necessitates synthetic data generation
\item The \euaiact{} imposes \textbf{mandatory transparency requirements} on high-risk AI fraud detection systems
\item SupTech represents a paradigm shift from reactive to \textbf{proactive surveillance}, but cross-border coordination remains limited
\item \textbf{No single data layer or detection method} is sufficient -- multi-modal AI integration is essential
\end{enumerate}
\bottomnote{Source: Paper Section 2}
\end{frame}

\end{document}
