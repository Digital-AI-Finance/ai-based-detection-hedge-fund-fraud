\documentclass[8pt,aspectratio=169]{beamer}
\usetheme{Madrid}
\usepackage{graphicx}
\usepackage{booktabs}
\usepackage{adjustbox}
\usepackage{multicol}
\usepackage{amsmath}
\usepackage{amssymb}

% Color definitions
\definecolor{mlblue}{RGB}{0,102,204}
\definecolor{mlpurple}{RGB}{51,51,178}
\definecolor{mllavender}{RGB}{173,173,224}
\definecolor{mllavender2}{RGB}{193,193,232}
\definecolor{mllavender3}{RGB}{204,204,235}
\definecolor{mllavender4}{RGB}{214,214,239}
\definecolor{mlorange}{RGB}{255, 127, 14}
\definecolor{mlgreen}{RGB}{44, 160, 44}
\definecolor{mlred}{RGB}{214, 39, 40}
\definecolor{mlgray}{RGB}{127, 127, 127}

% Additional colors
\definecolor{lightgray}{RGB}{240, 240, 240}
\definecolor{midgray}{RGB}{180, 180, 180}

% Apply custom colors to Madrid theme
\setbeamercolor{palette primary}{bg=mllavender3,fg=mlpurple}
\setbeamercolor{palette secondary}{bg=mllavender2,fg=mlpurple}
\setbeamercolor{palette tertiary}{bg=mllavender,fg=white}
\setbeamercolor{palette quaternary}{bg=mlpurple,fg=white}

\setbeamercolor{structure}{fg=mlpurple}
\setbeamercolor{section in toc}{fg=mlpurple}
\setbeamercolor{subsection in toc}{fg=mlblue}
\setbeamercolor{title}{fg=mlpurple}
\setbeamercolor{frametitle}{fg=mlpurple,bg=mllavender3}
\setbeamercolor{block title}{bg=mllavender2,fg=mlpurple}
\setbeamercolor{block body}{bg=mllavender4,fg=black}

% Remove navigation symbols
\setbeamertemplate{navigation symbols}{}

% Clean itemize/enumerate
\setbeamertemplate{itemize items}[circle]
\setbeamertemplate{enumerate items}[default]

% Reduce margins for more content space
\setbeamersize{text margin left=5mm,text margin right=5mm}

% Command for bottom annotation
\newcommand{\bottomnote}[1]{%
\vfill
\vspace{-2mm}
\textcolor{mllavender2}{\rule{\textwidth}{0.4pt}}
\vspace{1mm}
\footnotesize
\textbf{#1}
}

% Command for compact list spacing
\newcommand{\compactlist}{%
\setlength{\itemsep}{0pt}%
\setlength{\parskip}{0pt}%
\setlength{\parsep}{0pt}%
}

% Notation macros
% Shared notation macros for AI-Based Detection of Hedge Fund Fraud
% Include this file in all Beamer slide decks via % Shared notation macros for AI-Based Detection of Hedge Fund Fraud
% Include this file in all Beamer slide decks via % Shared notation macros for AI-Based Detection of Hedge Fund Fraud
% Include this file in all Beamer slide decks via \input{notation}

% Performance metrics
\newcommand{\auc}{\ensuremath{\mathrm{AUC}}}
\newcommand{\fone}{\ensuremath{F_1}}
\newcommand{\shap}{\text{SHAP}}
\newcommand{\lime}{\text{LIME}}

% Autocorrelation
\newcommand{\rhoone}{\ensuremath{\rho_1}}

% Financial abbreviations
\newcommand{\nav}{\text{NAV}}
\newcommand{\aum}{\text{AUM}}

% Regulatory
\newcommand{\euaiact}{EU AI Act}
\newcommand{\aifmd}{AIFMD}
\newcommand{\sec}{SEC}
\newcommand{\edgar}{EDGAR}

% Key numbers from the paper (for consistency)
\newcommand{\aumdollar}{\$4.5\text{ trillion}}
\newcommand{\aucdeg}{10.6\%}
\newcommand{\numop}{10}
\newcommand{\numfraudcases}{50\text{--}100}
\newcommand{\systematicpapers}{105}


% Performance metrics
\newcommand{\auc}{\ensuremath{\mathrm{AUC}}}
\newcommand{\fone}{\ensuremath{F_1}}
\newcommand{\shap}{\text{SHAP}}
\newcommand{\lime}{\text{LIME}}

% Autocorrelation
\newcommand{\rhoone}{\ensuremath{\rho_1}}

% Financial abbreviations
\newcommand{\nav}{\text{NAV}}
\newcommand{\aum}{\text{AUM}}

% Regulatory
\newcommand{\euaiact}{EU AI Act}
\newcommand{\aifmd}{AIFMD}
\newcommand{\sec}{SEC}
\newcommand{\edgar}{EDGAR}

% Key numbers from the paper (for consistency)
\newcommand{\aumdollar}{\$4.5\text{ trillion}}
\newcommand{\aucdeg}{10.6\%}
\newcommand{\numop}{10}
\newcommand{\numfraudcases}{50\text{--}100}
\newcommand{\systematicpapers}{105}


% Performance metrics
\newcommand{\auc}{\ensuremath{\mathrm{AUC}}}
\newcommand{\fone}{\ensuremath{F_1}}
\newcommand{\shap}{\text{SHAP}}
\newcommand{\lime}{\text{LIME}}

% Autocorrelation
\newcommand{\rhoone}{\ensuremath{\rho_1}}

% Financial abbreviations
\newcommand{\nav}{\text{NAV}}
\newcommand{\aum}{\text{AUM}}

% Regulatory
\newcommand{\euaiact}{EU AI Act}
\newcommand{\aifmd}{AIFMD}
\newcommand{\sec}{SEC}
\newcommand{\edgar}{EDGAR}

% Key numbers from the paper (for consistency)
\newcommand{\aumdollar}{\$4.5\text{ trillion}}
\newcommand{\aucdeg}{10.6\%}
\newcommand{\numop}{10}
\newcommand{\numfraudcases}{50\text{--}100}
\newcommand{\systematicpapers}{105}


\title{Quiz: Background -- The Hedge Fund Fraud Landscape}
\subtitle{Section 02 -- Digital-AI-Finance}
\author{Joerg Osterrieder}
\institute{Zurich University of Applied Sciences (ZHAW)}
\date{2025}

\begin{document}

\begin{frame}
\titlepage
\end{frame}

\begin{frame}{Question 1: Fraud Categories}
How many fraud categories are in the taxonomy?

\vspace{0.5cm}
\begin{enumerate}[a)]
\item 3
\item 4
\item 5
\item 7
\end{enumerate}

\vspace{0.5cm}
\pause
\begin{block}{Answer}
\textbf{c) 5}

The taxonomy organizes hedge fund fraud into five categories: performance fabrication, allocation fraud, strategy misrepresentation, market manipulation, and regulatory fraud.
\end{block}
\bottomnote{Source: Section 2.1}
\end{frame}

\begin{frame}{Question 2: Detection Difficulty}
Which fraud type has the highest detection difficulty?

\vspace{0.5cm}
\begin{enumerate}[a)]
\item Performance Fabrication
\item Allocation Fraud
\item Market Manipulation
\item Regulatory Fraud
\end{enumerate}

\vspace{0.5cm}
\pause
\begin{block}{Answer}
\textbf{c) Market Manipulation}

Market manipulation has a detection difficulty rating of 5/5, the highest among all fraud types, as it requires privileged real-time data and advanced methods to detect.
\end{block}
\bottomnote{Source: Section 2.1}
\end{frame}

\begin{frame}{Question 3: Survivorship Bias}
What is the estimated survivorship bias in hedge fund databases?

\vspace{0.5cm}
\begin{enumerate}[a)]
\item 42 basis points/year
\item 142 basis points/year
\item 242 basis points/year
\item 442 basis points/year
\end{enumerate}

\vspace{0.5cm}
\pause
\begin{block}{Answer}
\textbf{c) 242 basis points/year}

Survivorship bias, estimated at approximately 242 basis points per year, arises because defunct funds exit live databases, resulting in return overstatement.
\end{block}
\bottomnote{Source: Section 2.2}
\end{frame}

\begin{frame}{Question 4: Backfill Bias}
What is the estimated backfill bias?

\vspace{0.5cm}
\begin{enumerate}[a)]
\item 142bp/year
\item 242bp/year
\item 342bp/year
\item 442bp/year
\end{enumerate}

\vspace{0.5cm}
\pause
\begin{block}{Answer}
\textbf{d) 442bp/year}

Backfill bias, estimated at 442 basis points per year, results from the retroactive inclusion of favorable pre-reporting return histories when funds begin reporting to databases.
\end{block}
\bottomnote{Source: Section 2.2}
\end{frame}

\begin{frame}{Question 5: Regulatory Changes}
Which regulatory change followed the 2008 financial crisis in the US?

\vspace{0.5cm}
\begin{enumerate}[a)]
\item Sarbanes-Oxley
\item Dodd-Frank Title IV
\item Glass-Steagall
\item Basel III
\end{enumerate}

\vspace{0.5cm}
\pause
\begin{block}{Answer}
\textbf{b) Dodd-Frank Title IV}

The Dodd-Frank Act's Title IV, passed in 2010, mandated that hedge fund advisers register with the SEC and file Form ADV, substantially improving transparency and entity resolution capabilities.
\end{block}
\bottomnote{Source: Section 2.3}
\end{frame}

\begin{frame}{Question 6: EU AI Act Classification}
Under the EU AI Act, how is fraud detection classified?

\vspace{0.5cm}
\begin{enumerate}[a)]
\item Low-risk
\item Limited-risk
\item High-risk
\item Unacceptable risk
\end{enumerate}

\vspace{0.5cm}
\pause
\begin{block}{Answer}
\textbf{c) High-risk}

Financial fraud detection systems fall under the EU AI Act's high-risk classification (Annex III), subjecting them to stringent transparency, human oversight, and risk management requirements.
\end{block}
\bottomnote{Source: Section 2.3}
\end{frame}

\begin{frame}{Question 7: Data Ecosystem Layers}
How many layers does the data ecosystem have?

\vspace{0.5cm}
\begin{enumerate}[a)]
\item 2
\item 3
\item 4
\item 5
\end{enumerate}

\vspace{0.5cm}
\pause
\begin{block}{Answer}
\textbf{c) 4}

The hedge fund data ecosystem spans four layers: return data from commercial databases, regulatory filings, alternative data from news and social media, and synthetic data generated for class imbalance.
\end{block}
\bottomnote{Source: Section 2.2}
\end{frame}

\begin{frame}{Question 8: SupTech Meaning}
What does SupTech stand for?

\vspace{0.5cm}
\begin{enumerate}[a)]
\item Superior Technology
\item Supervisory Technology
\item Supplementary Technology
\item Supply Technology
\end{enumerate}

\vspace{0.5cm}
\pause
\begin{block}{Answer}
\textbf{b) Supervisory Technology}

SupTech (Supervisory Technology) refers to the use of technology by regulatory authorities to enhance their supervisory capabilities, including AI-based fraud detection systems.
\end{block}
\bottomnote{Source: Section 2.3}
\end{frame}

\end{document}
