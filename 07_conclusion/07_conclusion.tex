% ============================================================
%  Slide Deck 7 -- Conclusion and Key Takeaways
%  AI-Based Detection of Hedge Fund Fraud
% ============================================================
\documentclass[8pt,aspectratio=169]{beamer}
\usetheme{Madrid}
\usepackage{graphicx}
\usepackage{booktabs}
\usepackage{adjustbox}
\usepackage{multicol}
\usepackage{amsmath}
\usepackage{amssymb}

% ---- Color definitions ----
\definecolor{mlblue}{RGB}{0,102,204}
\definecolor{mlpurple}{RGB}{51,51,178}
\definecolor{mllavender}{RGB}{173,173,224}
\definecolor{mllavender2}{RGB}{193,193,232}
\definecolor{mllavender3}{RGB}{204,204,235}
\definecolor{mllavender4}{RGB}{214,214,239}
\definecolor{mlorange}{RGB}{255,127,14}
\definecolor{mlgreen}{RGB}{44,160,44}
\definecolor{mlred}{RGB}{214,39,40}
\definecolor{mlgray}{RGB}{127,127,127}
\definecolor{lightgray}{RGB}{240,240,240}
\definecolor{midgray}{RGB}{180,180,180}

% ---- Apply custom colors to Madrid theme ----
\setbeamercolor{palette primary}{bg=mllavender3,fg=mlpurple}
\setbeamercolor{palette secondary}{bg=mllavender2,fg=mlpurple}
\setbeamercolor{palette tertiary}{bg=mllavender,fg=white}
\setbeamercolor{palette quaternary}{bg=mlpurple,fg=white}
\setbeamercolor{structure}{fg=mlpurple}
\setbeamercolor{section in toc}{fg=mlpurple}
\setbeamercolor{subsection in toc}{fg=mlblue}
\setbeamercolor{title}{fg=mlpurple}
\setbeamercolor{frametitle}{fg=mlpurple,bg=mllavender3}
\setbeamercolor{block title}{bg=mllavender2,fg=mlpurple}
\setbeamercolor{block body}{bg=mllavender4,fg=black}

% ---- Navigation / itemize ----
\setbeamertemplate{navigation symbols}{}
\setbeamertemplate{itemize items}[circle]
\setbeamertemplate{enumerate items}[default]
\setbeamersize{text margin left=5mm,text margin right=5mm}

% ---- Custom commands ----
\newcommand{\bottomnote}[1]{%
\vfill
\vspace{-2mm}
\textcolor{mllavender2}{\rule{\textwidth}{0.4pt}}
\vspace{1mm}
\footnotesize
\textbf{#1}
}

\newcommand{\compactlist}{%
\setlength{\itemsep}{0pt}%
\setlength{\parskip}{0pt}%
\setlength{\parsep}{0pt}%
}

\newcommand{\chartplaceholder}[2][5cm]{%
\begin{center}
\begin{adjustbox}{max width=0.95\textwidth, max height=#1}
\framebox[\textwidth][c]{%
\rule{0pt}{#1}%
\textcolor{midgray}{[#2]}%
}
\end{adjustbox}
\end{center}
}

% ---- Notation ----
% Shared notation macros for AI-Based Detection of Hedge Fund Fraud
% Include this file in all Beamer slide decks via % Shared notation macros for AI-Based Detection of Hedge Fund Fraud
% Include this file in all Beamer slide decks via % Shared notation macros for AI-Based Detection of Hedge Fund Fraud
% Include this file in all Beamer slide decks via \input{notation}

% Performance metrics
\newcommand{\auc}{\ensuremath{\mathrm{AUC}}}
\newcommand{\fone}{\ensuremath{F_1}}
\newcommand{\shap}{\text{SHAP}}
\newcommand{\lime}{\text{LIME}}

% Autocorrelation
\newcommand{\rhoone}{\ensuremath{\rho_1}}

% Financial abbreviations
\newcommand{\nav}{\text{NAV}}
\newcommand{\aum}{\text{AUM}}

% Regulatory
\newcommand{\euaiact}{EU AI Act}
\newcommand{\aifmd}{AIFMD}
\newcommand{\sec}{SEC}
\newcommand{\edgar}{EDGAR}

% Key numbers from the paper (for consistency)
\newcommand{\aumdollar}{\$4.5\text{ trillion}}
\newcommand{\aucdeg}{10.6\%}
\newcommand{\numop}{10}
\newcommand{\numfraudcases}{50\text{--}100}
\newcommand{\systematicpapers}{105}


% Performance metrics
\newcommand{\auc}{\ensuremath{\mathrm{AUC}}}
\newcommand{\fone}{\ensuremath{F_1}}
\newcommand{\shap}{\text{SHAP}}
\newcommand{\lime}{\text{LIME}}

% Autocorrelation
\newcommand{\rhoone}{\ensuremath{\rho_1}}

% Financial abbreviations
\newcommand{\nav}{\text{NAV}}
\newcommand{\aum}{\text{AUM}}

% Regulatory
\newcommand{\euaiact}{EU AI Act}
\newcommand{\aifmd}{AIFMD}
\newcommand{\sec}{SEC}
\newcommand{\edgar}{EDGAR}

% Key numbers from the paper (for consistency)
\newcommand{\aumdollar}{\$4.5\text{ trillion}}
\newcommand{\aucdeg}{10.6\%}
\newcommand{\numop}{10}
\newcommand{\numfraudcases}{50\text{--}100}
\newcommand{\systematicpapers}{105}


% Performance metrics
\newcommand{\auc}{\ensuremath{\mathrm{AUC}}}
\newcommand{\fone}{\ensuremath{F_1}}
\newcommand{\shap}{\text{SHAP}}
\newcommand{\lime}{\text{LIME}}

% Autocorrelation
\newcommand{\rhoone}{\ensuremath{\rho_1}}

% Financial abbreviations
\newcommand{\nav}{\text{NAV}}
\newcommand{\aum}{\text{AUM}}

% Regulatory
\newcommand{\euaiact}{EU AI Act}
\newcommand{\aifmd}{AIFMD}
\newcommand{\sec}{SEC}
\newcommand{\edgar}{EDGAR}

% Key numbers from the paper (for consistency)
\newcommand{\aumdollar}{\$4.5\text{ trillion}}
\newcommand{\aucdeg}{10.6\%}
\newcommand{\numop}{10}
\newcommand{\numfraudcases}{50\text{--}100}
\newcommand{\systematicpapers}{105}


% ---- Title metadata ----
\title{AI-Based Detection of Hedge Fund Fraud}
\subtitle{Section 7 -- Conclusion and Key Takeaways}
\author{Joerg Osterrieder}
\institute{Zurich University of Applied Sciences (ZHAW)}
\date{2025}

% ============================================================
\begin{document}

% ----------------------------------------------------------
% SLIDE 1 -- Title
% ----------------------------------------------------------
\begin{frame}
\titlepage
\end{frame}

% ----------------------------------------------------------
% SLIDE 2 -- Outline
% ----------------------------------------------------------
\begin{frame}{Outline}
\begin{enumerate}\compactlist
\item C1 Summary: Detection Pipeline Taxonomy
\item C2 Summary: Adversarial and Regulatory Readiness
\item C3 Summary: Research Roadmap
\item Contributions Chart
\item Key Takeaways for Practitioners
\item Key Takeaways for Regulators
\item Key Takeaways for Researchers
\item Takeaways Chart
\item The Co-Evolution of Fraud and Detection
\item From Supervised to Adversarial Learning
\item The Collaboration Imperative
\item The Field at an Inflection Point
\item Thank You / Questions
\item Selected References
\end{enumerate}
\end{frame}

% ----------------------------------------------------------
% SLIDE 3 -- C1 Summary
% ----------------------------------------------------------
\begin{frame}{C1: Detection Pipeline Taxonomy -- Five Stages}
\begin{block}{Contribution C1}
A \textbf{unified five-stage detection pipeline} framework that systematically maps hedge fund fraud types to appropriate AI detection methods.
\end{block}
\vspace{2mm}
\begin{enumerate}\compactlist
\item \textcolor{mlblue}{\textbf{Data Ingestion}}: return series, regulatory filings, alternative data, relational networks
\item \textcolor{mlblue}{\textbf{Feature Engineering}}: statistical, distributional, NLP-derived, graph-based features
\item \textcolor{mlblue}{\textbf{Model Selection}}: method family guided by fraud typology and data availability
\item \textcolor{mlblue}{\textbf{Explainability}}: post-hoc (\shap{}, \lime{}) or inherently interpretable architectures
\item \textcolor{mlblue}{\textbf{Deployment}}: operational integration, monitoring, feedback loops
\end{enumerate}
\vspace{2mm}
\begin{itemize}\compactlist
\item Makes explicit that \textbf{no single method covers all fraud types}
\item Effective surveillance requires a \textbf{multi-stage architecture} with method selection guided by specific fraud typology
\item Engineering blueprint for operational surveillance systems
\end{itemize}
\bottomnote{Source: Paper Section 3 / Section 7}
\end{frame}

% ----------------------------------------------------------
% SLIDE 4 -- C2 Summary
% ----------------------------------------------------------
\begin{frame}{C2: Adversarial and Regulatory Readiness Assessment}
\begin{block}{Contribution C2}
Systematic evaluation of adversarial vulnerabilities and regulatory compliance of current AI methods for hedge fund fraud detection.
\end{block}
\vspace{2mm}
\begin{columns}[T]
\column{0.48\textwidth}
\textbf{Adversarial Assessment}
\begin{itemize}\compactlist
\item Mean \auc{} degradation: \textcolor{mlred}{\textbf{\aucdeg{}}} under adversarial perturbation
\item Some techniques: $>$25\% degradation under informed adversaries
\item Adversaries are PhD-level quants, not generic attackers
\item Most detection models \textbf{untested} against hedge-fund-specific adversaries
\end{itemize}

\column{0.48\textwidth}
\textbf{Regulatory Assessment}
\begin{itemize}\compactlist
\item \euaiact{} classifies fraud detection as \textbf{high-risk AI} (transparency, human oversight, risk management)
\item SEC signaling increasing attention to AI governance
\item Explainability--performance trade-off: best performers are most opaque
\item Detection performance on historical data is \textbf{necessary but not sufficient}
\end{itemize}
\end{columns}
\bottomnote{Source: Paper Section 5 / Section 7}
\end{frame}

% ----------------------------------------------------------
% SLIDE 5 -- C3 Summary
% ----------------------------------------------------------
\begin{frame}{C3: Research Roadmap -- 10 Open Problems}
\begin{block}{Contribution C3}
Ten concrete open problems, each uniquely challenging in the hedge fund context, with suggested approaches, evaluation protocols, and feasibility assessments.
\end{block}
\vspace{2mm}
{\footnotesize
\begin{columns}[T]
\column{0.32\textwidth}
\textbf{Data (OP1--3)}
\begin{itemize}\compactlist
\item OP1: Benchmark datasets
\item OP2: Cross-jurisdictional integration
\item OP3: Real-time alternative data
\end{itemize}

\column{0.32\textwidth}
\textbf{Methods (OP4--7)}
\begin{itemize}\compactlist
\item OP4: Extreme class imbalance
\item OP5: Cold-start detection
\item OP6: Concept drift
\item OP7: Multi-modal fusion
\end{itemize}

\column{0.32\textwidth}
\textbf{Deployment (OP8--10)}
\begin{itemize}\compactlist
\item OP8: Adversarial robustness
\item OP9: Explainability
\item OP10: Human-AI collaboration
\end{itemize}
\end{columns}
}
\vspace{3mm}
\begin{itemize}\compactlist
\item \textbf{OP1} (benchmarks) and \textbf{OP4} (class imbalance) are \textcolor{mlred}{critical preconditions} that unlock progress on nearly all other problems
\item OP2 and OP10 require \textbf{institutional partnerships} beyond any single research group
\end{itemize}
\bottomnote{Source: Paper Section 6 / Section 7}
\end{frame}

% ----------------------------------------------------------
% SLIDE 6 -- Contributions Chart
% ----------------------------------------------------------
\begin{frame}{Three Contributions: Visual Summary}

\chartplaceholder[5cm]{Chart: 01\_contributions\_summary -- Three-panel visual summarizing C1 (pipeline taxonomy with 5 stages), C2 (adversarial/regulatory assessment with key metrics), and C3 (research roadmap with 10 open problems grouped by category)}

\vspace{2mm}
\begin{itemize}\compactlist
\item C1 provides the \textbf{organizational backbone} for C2 (method evaluation) and C3 (research priorities)
\item Together: a comprehensive analytical framework from survey to actionable agenda
\end{itemize}
\bottomnote{Source: Paper Section 7}
\end{frame}

% ----------------------------------------------------------
% SLIDE 7 -- Takeaways: Practitioners
% ----------------------------------------------------------
\begin{frame}{Key Takeaways for Practitioners}
\begin{enumerate}\compactlist
\item \textcolor{mlblue}{\textbf{Ensemble methods offer the best current balance}} of performance, interpretability, and deployment readiness
  \begin{itemize}\compactlist
  \item Gradient boosting, stacking ensembles, random forests
  \item Augment with \shap{} for investigation-ready outputs
  \item Accessible entry point without specialized hardware or deep learning expertise
  \end{itemize}
\vspace{2mm}
\item \textcolor{mlblue}{\textbf{Multi-stage pipeline with fraud-type-guided method selection}} is essential
  \begin{itemize}\compactlist
  \item Performance fabrication: serial correlation tests, Benford's law, Sharpe ratio plausibility (difficulty 3/5)
  \item Regulatory fraud: NLP on filings (difficulty 2/5)
  \item Market manipulation: order-level trade data, cross-account analysis (difficulty 5/5)
  \item Allocation fraud: win-rate asymmetry, timestamp analysis (difficulty 4/5)
  \end{itemize}
\vspace{2mm}
\item \textcolor{mlblue}{\textbf{Adversarial robustness testing should be standard practice}}
  \begin{itemize}\compactlist
  \item Mean \aucdeg{} degradation \textbf{understates} risk (generic perturbations, not domain-tailored)
  \item Domain-specific red-teaming: financial experts construct evasive return series
  \end{itemize}
\end{enumerate}
\bottomnote{Source: Paper Section 7 -- Takeaways for Practitioners}
\end{frame}

% ----------------------------------------------------------
% SLIDE 8 -- Takeaways: Regulators
% ----------------------------------------------------------
\begin{frame}{Key Takeaways for Regulators}
\begin{enumerate}\compactlist
\item \textcolor{mlblue}{\textbf{AI addresses the scale mismatch}} between regulatory resources and industry size
  \begin{itemize}\compactlist
  \item SEC Division of Examinations inspects only a fraction of funds per year
  \item AI processes thousands of return series, filings, and alternative data simultaneously
  \item \textbf{But}: models must satisfy emerging explainability requirements
  \item EU AI Act: ``sufficiently transparent'' outputs for high-risk systems
  \item May need to favor inherently interpretable architectures (neural additive models, explanation-augmented boosting) over black-box deep learning
  \item A fraud alert that \textbf{cannot be explained} to an enforcement attorney is of limited operational value
  \end{itemize}
\vspace{3mm}
\item \textcolor{mlblue}{\textbf{Cross-jurisdictional data sharing is a prerequisite}} for effective detection
  \begin{itemize}\compactlist
  \item Sophisticated fraudsters exploit regulatory arbitrage \textbf{deliberately}
  \item No single jurisdiction possesses a complete picture
  \item \textbf{Federated learning} offers a technical path: shared models without exchanging raw data
  \item Technical foundations exist; what remains is \textbf{institutional coordination}
  \end{itemize}
\end{enumerate}
\bottomnote{Source: Paper Section 7 -- Takeaways for Regulators}
\end{frame}

% ----------------------------------------------------------
% SLIDE 9 -- Takeaways: Researchers
% ----------------------------------------------------------
\begin{frame}{Key Takeaways for Researchers}
\begin{enumerate}\compactlist
\item \textcolor{mlblue}{\textbf{Benchmark datasets are the most critical enabler}}
  \begin{itemize}\compactlist
  \item Absence has fragmented the literature, prevented reproducible comparisons, impeded advancement
  \item Two-track approach: synthetic generation + differentially private regulatory data release
  \item Researchers with access to commercial databases or regulatory data can make a \textbf{foundational contribution}
  \end{itemize}
\vspace{2mm}
\item \textcolor{mlblue}{\textbf{Multi-modal fusion is underexplored and high-potential}}
  \begin{itemize}\compactlist
  \item Most methods operate on a single modality in isolation
  \item Sophisticated fraud exhibits anomalies across \textbf{multiple modalities simultaneously}
  \item Attention-based multi-modal transformers, cross-modal contrastive learning, hierarchical fusion
  \end{itemize}
\vspace{2mm}
\item \textcolor{mlblue}{\textbf{Adversarial robustness with domain-specific threat models}} deserves far more attention
  \begin{itemize}\compactlist
  \item Current literature focuses on image classification and unsophisticated adversaries
  \item Hedge fund fraud: highly quantitative adversary with economic constraints fundamentally different from $\ell_p$-norm budgets
  \item Certified bounds, game-theoretic modeling, red-teaming with domain experts
  \end{itemize}
\end{enumerate}
\bottomnote{Source: Paper Section 7 -- Takeaways for Researchers}
\end{frame}

% ----------------------------------------------------------
% SLIDE 10 -- Takeaways Chart
% ----------------------------------------------------------
\begin{frame}{Audience-Specific Takeaways: Visual Summary}

\chartplaceholder[5cm]{Chart: 02\_takeaways\_by\_audience -- Three-column visual mapping key takeaways to Practitioners (ensembles, multi-stage pipeline, adversarial testing), Regulators (scale mismatch, explainability, federated learning), and Researchers (benchmarks, multi-modal fusion, adversarial robustness)}

\vspace{2mm}
\begin{itemize}\compactlist
\item Each audience has distinct priorities, but \textbf{all three} must collaborate for operational progress
\end{itemize}
\bottomnote{Source: Paper Section 7}
\end{frame}

% ----------------------------------------------------------
% SLIDE 11 -- Co-Evolution of Fraud and Detection
% ----------------------------------------------------------
\begin{frame}{The Co-Evolution of Fraud and Detection}
\begin{block}{An Arms Race}
The history of financial fraud detection is a history of \textbf{co-evolution} between detection methods and evasion strategies.
\end{block}
\vspace{3mm}
\begin{itemize}\compactlist
\item Statistical anomaly detection $\Rightarrow$ \textit{engineered statistically plausible fabricated returns}
\item Benford's law analysis $\Rightarrow$ \textit{digit distributions that pass Benford tests}
\item Return smoothing detection $\Rightarrow$ \textit{sophisticated NAV manipulation with plausible serial correlation}
\item \textbf{AI/ML deployment} $\Rightarrow$ \textit{adaptive strategies designed to evade ML models} (inevitable)
\end{itemize}
\vspace{3mm}
\textcolor{mlred}{$\Rightarrow$ The research community must focus not only on methods that perform well on \textbf{historical data} but on methods that remain effective under \textbf{strategic manipulation} by sophisticated adversaries.}
\bottomnote{Source: Paper Section 7 -- The Co-Evolution of Fraud and Detection}
\end{frame}

% ----------------------------------------------------------
% SLIDE 12 -- Paradigm Shift
% ----------------------------------------------------------
\begin{frame}{From Supervised to Adversarial Learning}
\begin{columns}[T]
\column{0.48\textwidth}
\begin{block}{Standard Supervised Paradigm}
\begin{itemize}\compactlist
\item Data distribution assumed \textbf{stationary}
\item Train on historical labeled examples
\item Evaluate on held-out test set
\item Implicit assumption: future data looks like past data
\item \textcolor{mlred}{Ignores adversarial dynamics entirely}
\end{itemize}
\end{block}

\column{0.48\textwidth}
\begin{block}{Adversarial Learning Paradigm}
\begin{itemize}\compactlist
\item Defender and fraudster engaged in a \textbf{repeated game}
\item Data distribution shifts in response to detection
\item Must model adversary's strategic behavior
\item Robustness guarantees under worst-case manipulation
\item \textcolor{mlgreen}{Accounts for the reality of sophisticated adversaries}
\end{itemize}
\end{block}
\end{columns}
\vspace{3mm}
\textcolor{mlblue}{This paradigm shift -- from i.i.d.\ assumption to adversarial game -- is essential for operational deployment in the hedge fund context.}
\bottomnote{Source: Paper Section 7}
\end{frame}

% ----------------------------------------------------------
% SLIDE 13 -- Collaboration Imperative
% ----------------------------------------------------------
\begin{frame}{The Collaboration Imperative}
\begin{block}{No Single Community Has Data, Methods, and Operational Context}
The ten open problems represent not merely technical challenges but \textbf{institutional ones}.
\end{block}
\vspace{3mm}
\begin{columns}[T]
\column{0.32\textwidth}
\textbf{Regulatory Agencies}
\begin{itemize}\compactlist
\item Enforcement-labeled data
\item Cross-border relationships
\item Operational evaluation environments
\item Regulatory sandboxes
\end{itemize}

\column{0.32\textwidth}
\textbf{Academic Researchers}
\begin{itemize}\compactlist
\item Deep learning methods
\item Adversarial robustness
\item Explainability
\item Federated learning
\end{itemize}

\column{0.32\textwidth}
\textbf{Industry Practitioners}
\begin{itemize}\compactlist
\item Strategy evolution knowledge
\item Operational due diligence
\item Fund management realities
\item Domain-specific red-teaming
\end{itemize}
\end{columns}
\vspace{3mm}
Required: \textbf{shared data initiatives}, \textbf{regulatory sandboxes for AI testing}, and \textbf{joint research programs}.
\bottomnote{Source: Paper Section 7}
\end{frame}

% ----------------------------------------------------------
% SLIDE 14 -- Field at Inflection Point
% ----------------------------------------------------------
\begin{frame}{The Field at an Inflection Point}
\begin{center}
{\large\textcolor{mlpurple}{\textbf{Three conditions are now in place:}}}
\end{center}
\vspace{3mm}
\begin{enumerate}\compactlist
\item \textcolor{mlblue}{\textbf{Technical foundations established}}: AI methods can detect anomalies in returns, extract signals from filings, and model relational structures
\vspace{2mm}
\item \textcolor{mlblue}{\textbf{Regulatory imperative is clear}}: EU AI Act, SEC attention, global push for AI governance in financial surveillance
\vspace{2mm}
\item \textcolor{mlblue}{\textbf{Economic stakes are enormous}}: \aumdollar{} industry, multi-billion-dollar fraud cases, systemic risk implications
\end{enumerate}
\vspace{4mm}
\begin{block}{What Remains}
The coordinated effort to translate promising methods into \textbf{operationally deployed}, \textbf{robustly tested}, and \textbf{legally compliant} systems that can protect investors and preserve market integrity at the scale the modern hedge fund industry demands.
\end{block}
\bottomnote{Source: Paper Section 7}
\end{frame}

% ----------------------------------------------------------
% SLIDE 15 -- Thank You / Questions
% ----------------------------------------------------------
\begin{frame}{}
\vfill
\begin{center}
{\Large\textcolor{mlpurple}{\textbf{Thank You}}}

\vspace{5mm}
{\large Questions and Discussion}

\vspace{8mm}
\textbf{Joerg Osterrieder}\\
Zurich University of Applied Sciences (ZHAW)\\

\vspace{5mm}
\textcolor{mlblue}{Paper: \textit{AI-Based Detection of Hedge Fund Fraud: A Survey}}
\end{center}
\vfill
\end{frame}

% ----------------------------------------------------------
% SLIDE 16 -- Selected References
% ----------------------------------------------------------
\begin{frame}{Selected References}
{\scriptsize
\begin{itemize}\compactlist
\item Bollen, N.\ P.\ B.\ \& Pool, V.\ K.\ (2012). Suspicious patterns in hedge fund returns and the risk of fraud. \textit{Review of Financial Studies}, 25(9).
\item Cartella, F.\ et al.\ (2021). Adversarial attacks on financial fraud detection models. \textit{Expert Systems with Applications}.
\item Chen, Z.\ et al.\ (2024). Robust optimization for financial fraud detection. \textit{IEEE Trans.\ Neural Networks}.
\item Dimmock, S.\ G.\ \& Gerken, W.\ C.\ (2012). Predicting fraud by investment managers. \textit{Journal of Financial Economics}, 105(1).
\item EU AI Act (2024). Regulation 2024/1689 of the European Parliament and of the Council.
\item Getmansky, M., Lo, A.\ W., \& Makarov, I.\ (2004). An econometric model of serial correlation and illiquidity. \textit{Journal of Financial Economics}, 74(3).
\item Goodfellow, I.\ J.\ et al.\ (2015). Explaining and harnessing adversarial examples. \textit{ICLR}.
\item Lundberg, S.\ M.\ \& Lee, S.-I.\ (2017). A unified approach to interpreting model predictions. \textit{NeurIPS}.
\item Madry, A.\ et al.\ (2018). Towards deep learning models resistant to adversarial attacks. \textit{ICLR}.
\item Rudin, C.\ (2019). Stop explaining black box ML models for high stakes decisions. \textit{Nature Machine Intelligence}.
\item Vaswani, A.\ et al.\ (2017). Attention is all you need. \textit{NeurIPS}.
\end{itemize}
}
\bottomnote{Full bibliography available in the paper.}
\end{frame}

\end{document}
