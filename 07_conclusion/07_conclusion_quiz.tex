\documentclass[8pt,aspectratio=169]{beamer}
\usetheme{Madrid}
\usepackage{graphicx}
\usepackage{booktabs}
\usepackage{adjustbox}
\usepackage{multicol}
\usepackage{amsmath}
\usepackage{amssymb}

% Color definitions
\definecolor{mlblue}{RGB}{0,102,204}
\definecolor{mlpurple}{RGB}{51,51,178}
\definecolor{mllavender}{RGB}{173,173,224}
\definecolor{mllavender2}{RGB}{193,193,232}
\definecolor{mllavender3}{RGB}{204,204,235}
\definecolor{mllavender4}{RGB}{214,214,239}
\definecolor{mlorange}{RGB}{255, 127, 14}
\definecolor{mlgreen}{RGB}{44, 160, 44}
\definecolor{mlred}{RGB}{214, 39, 40}
\definecolor{mlgray}{RGB}{127, 127, 127}

% Additional colors
\definecolor{lightgray}{RGB}{240, 240, 240}
\definecolor{midgray}{RGB}{180, 180, 180}

% Apply custom colors to Madrid theme
\setbeamercolor{palette primary}{bg=mllavender3,fg=mlpurple}
\setbeamercolor{palette secondary}{bg=mllavender2,fg=mlpurple}
\setbeamercolor{palette tertiary}{bg=mllavender,fg=white}
\setbeamercolor{palette quaternary}{bg=mlpurple,fg=white}

\setbeamercolor{structure}{fg=mlpurple}
\setbeamercolor{section in toc}{fg=mlpurple}
\setbeamercolor{subsection in toc}{fg=mlblue}
\setbeamercolor{title}{fg=mlpurple}
\setbeamercolor{frametitle}{fg=mlpurple,bg=mllavender3}
\setbeamercolor{block title}{bg=mllavender2,fg=mlpurple}
\setbeamercolor{block body}{bg=mllavender4,fg=black}

% Remove navigation symbols
\setbeamertemplate{navigation symbols}{}

% Clean itemize/enumerate
\setbeamertemplate{itemize items}[circle]
\setbeamertemplate{enumerate items}[default]

% Reduce margins for more content space
\setbeamersize{text margin left=5mm,text margin right=5mm}

% Command for bottom annotation
\newcommand{\bottomnote}[1]{%
\vfill
\vspace{-2mm}
\textcolor{mllavender2}{\rule{\textwidth}{0.4pt}}
\vspace{1mm}
\footnotesize
\textbf{#1}
}

% Command for compact list spacing
\newcommand{\compactlist}{%
\setlength{\itemsep}{0pt}%
\setlength{\parskip}{0pt}%
\setlength{\parsep}{0pt}%
}

% Notation macros
% Shared notation macros for AI-Based Detection of Hedge Fund Fraud
% Include this file in all Beamer slide decks via % Shared notation macros for AI-Based Detection of Hedge Fund Fraud
% Include this file in all Beamer slide decks via % Shared notation macros for AI-Based Detection of Hedge Fund Fraud
% Include this file in all Beamer slide decks via \input{notation}

% Performance metrics
\newcommand{\auc}{\ensuremath{\mathrm{AUC}}}
\newcommand{\fone}{\ensuremath{F_1}}
\newcommand{\shap}{\text{SHAP}}
\newcommand{\lime}{\text{LIME}}

% Autocorrelation
\newcommand{\rhoone}{\ensuremath{\rho_1}}

% Financial abbreviations
\newcommand{\nav}{\text{NAV}}
\newcommand{\aum}{\text{AUM}}

% Regulatory
\newcommand{\euaiact}{EU AI Act}
\newcommand{\aifmd}{AIFMD}
\newcommand{\sec}{SEC}
\newcommand{\edgar}{EDGAR}

% Key numbers from the paper (for consistency)
\newcommand{\aumdollar}{\$4.5\text{ trillion}}
\newcommand{\aucdeg}{10.6\%}
\newcommand{\numop}{10}
\newcommand{\numfraudcases}{50\text{--}100}
\newcommand{\systematicpapers}{105}


% Performance metrics
\newcommand{\auc}{\ensuremath{\mathrm{AUC}}}
\newcommand{\fone}{\ensuremath{F_1}}
\newcommand{\shap}{\text{SHAP}}
\newcommand{\lime}{\text{LIME}}

% Autocorrelation
\newcommand{\rhoone}{\ensuremath{\rho_1}}

% Financial abbreviations
\newcommand{\nav}{\text{NAV}}
\newcommand{\aum}{\text{AUM}}

% Regulatory
\newcommand{\euaiact}{EU AI Act}
\newcommand{\aifmd}{AIFMD}
\newcommand{\sec}{SEC}
\newcommand{\edgar}{EDGAR}

% Key numbers from the paper (for consistency)
\newcommand{\aumdollar}{\$4.5\text{ trillion}}
\newcommand{\aucdeg}{10.6\%}
\newcommand{\numop}{10}
\newcommand{\numfraudcases}{50\text{--}100}
\newcommand{\systematicpapers}{105}


% Performance metrics
\newcommand{\auc}{\ensuremath{\mathrm{AUC}}}
\newcommand{\fone}{\ensuremath{F_1}}
\newcommand{\shap}{\text{SHAP}}
\newcommand{\lime}{\text{LIME}}

% Autocorrelation
\newcommand{\rhoone}{\ensuremath{\rho_1}}

% Financial abbreviations
\newcommand{\nav}{\text{NAV}}
\newcommand{\aum}{\text{AUM}}

% Regulatory
\newcommand{\euaiact}{EU AI Act}
\newcommand{\aifmd}{AIFMD}
\newcommand{\sec}{SEC}
\newcommand{\edgar}{EDGAR}

% Key numbers from the paper (for consistency)
\newcommand{\aumdollar}{\$4.5\text{ trillion}}
\newcommand{\aucdeg}{10.6\%}
\newcommand{\numop}{10}
\newcommand{\numfraudcases}{50\text{--}100}
\newcommand{\systematicpapers}{105}


\title{Quiz: Conclusion}
\subtitle{AI-Based Detection of Hedge Fund Fraud}
\author{Joerg Osterrieder}
\institute{Zurich University of Applied Sciences (ZHAW)}
\date{2025}

\begin{document}

\begin{frame}
\titlepage
\end{frame}

\begin{frame}{Question 1}
Which methods offer the best balance of performance and interpretability?
\begin{itemize}\compactlist
\item[a)] Deep neural networks
\item[b)] Ensemble methods
\item[c)] Logistic regression
\item[d)] Graph neural networks
\end{itemize}

\pause
\begin{block}{Answer}
\textbf{b)} Ensemble methods\\[2mm]
Random forests and gradient boosting provide the best balance, achieving high AUC (0.85--0.92) while maintaining interpretability through feature importance scores and SHAP values.
\end{block}
\bottomnote{Section 7: Conclusion}
\end{frame}

\begin{frame}{Question 2}
What are the three audience groups for the takeaways?
\begin{itemize}\compactlist
\item[a)] Students, Teachers, Admins
\item[b)] Developers, Testers, Managers
\item[c)] Practitioners, Regulators, Researchers
\item[d)] Investors, Brokers, Auditors
\end{itemize}

\pause
\begin{block}{Answer}
\textbf{c)} Practitioners, Regulators, Researchers\\[2mm]
The conclusion provides targeted takeaways for: (1) Practitioners (fund managers, compliance officers), (2) Regulators (SEC, ESMA, FCA), and (3) Researchers (academics, data scientists).
\end{block}
\bottomnote{Section 7: Takeaways by Audience}
\end{frame}

\begin{frame}{Question 3}
What is the most critical enabler for field progress?
\begin{itemize}\compactlist
\item[a)] Benchmark datasets
\item[b)] Better algorithms
\item[c)] More funding
\item[d)] Regulatory reform
\end{itemize}

\pause
\begin{block}{Answer}
\textbf{a)} Benchmark datasets\\[2mm]
The conclusion emphasizes that creating shared, privacy-preserving benchmark datasets is the single most critical enabler. Without accessible data, researchers cannot compare methods or validate improvements.
\end{block}
\bottomnote{Section 7: Key Takeaways}
\end{frame}

\begin{frame}{Question 4}
What collaboration model is required for production deployment?
\begin{itemize}\compactlist
\item[a)] Government only
\item[b)] Industry only
\item[c)] Academic only
\item[d)] Academic + Regulatory + Industry
\end{itemize}

\pause
\begin{block}{Answer}
\textbf{d)} Academic + Regulatory + Industry\\[2mm]
The conclusion stresses that production-grade fraud detection requires three-way collaboration: academics (methodology), regulators (data access, requirements), and industry (deployment, validation).
\end{block}
\bottomnote{Section 7: Path Forward}
\end{frame}

\begin{frame}{Question 5}
What paradigm shift is needed for adversarial fraud?
\begin{itemize}\compactlist
\item[a)] Batch to real-time processing
\item[b)] Supervised to adversarial learning
\item[c)] Cloud to edge computing
\item[d)] Rule-based to machine learning
\end{itemize}

\pause
\begin{block}{Answer}
\textbf{b)} Supervised to adversarial learning\\[2mm]
The conclusion calls for a shift from static supervised learning to adversarial learning frameworks that can adapt to evolving fraud tactics, recognizing fraud detection as an arms race.
\end{block}
\bottomnote{Section 7: Future Directions}
\end{frame}

\end{document}
