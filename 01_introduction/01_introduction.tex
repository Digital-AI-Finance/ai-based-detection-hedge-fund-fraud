% ============================================================
%  Slide Deck 1 -- Introduction
%  AI-Based Detection of Hedge Fund Fraud
% ============================================================
\documentclass[8pt,aspectratio=169]{beamer}
\usetheme{Madrid}
\usepackage{graphicx}
\usepackage{booktabs}
\usepackage{adjustbox}
\usepackage{multicol}
\usepackage{amsmath}
\usepackage{amssymb}

% ---- Color definitions ----
\definecolor{mlblue}{RGB}{0,102,204}
\definecolor{mlpurple}{RGB}{51,51,178}
\definecolor{mllavender}{RGB}{173,173,224}
\definecolor{mllavender2}{RGB}{193,193,232}
\definecolor{mllavender3}{RGB}{204,204,235}
\definecolor{mllavender4}{RGB}{214,214,239}
\definecolor{mlorange}{RGB}{255, 127, 14}
\definecolor{mlgreen}{RGB}{44, 160, 44}
\definecolor{mlred}{RGB}{214, 39, 40}
\definecolor{mlgray}{RGB}{127, 127, 127}
\definecolor{lightgray}{RGB}{240, 240, 240}
\definecolor{midgray}{RGB}{180, 180, 180}

% ---- Apply custom colors to Madrid theme ----
\setbeamercolor{palette primary}{bg=mllavender3,fg=mlpurple}
\setbeamercolor{palette secondary}{bg=mllavender2,fg=mlpurple}
\setbeamercolor{palette tertiary}{bg=mllavender,fg=white}
\setbeamercolor{palette quaternary}{bg=mlpurple,fg=white}
\setbeamercolor{structure}{fg=mlpurple}
\setbeamercolor{section in toc}{fg=mlpurple}
\setbeamercolor{subsection in toc}{fg=mlblue}
\setbeamercolor{title}{fg=mlpurple}
\setbeamercolor{frametitle}{fg=mlpurple,bg=mllavender3}
\setbeamercolor{block title}{bg=mllavender2,fg=mlpurple}
\setbeamercolor{block body}{bg=mllavender4,fg=black}

% ---- Navigation / itemize ----
\setbeamertemplate{navigation symbols}{}
\setbeamertemplate{itemize items}[circle]
\setbeamertemplate{enumerate items}[default]
\setbeamersize{text margin left=5mm,text margin right=5mm}

% ---- Custom commands ----
\newcommand{\bottomnote}[1]{%
\vfill
\vspace{-2mm}
\textcolor{mllavender2}{\rule{\textwidth}{0.4pt}}
\vspace{1mm}
\footnotesize
\textbf{#1}
}

\newcommand{\compactlist}{%
\setlength{\itemsep}{0pt}%
\setlength{\parskip}{0pt}%
\setlength{\parsep}{0pt}%
}

\newcommand{\chartplaceholder}[2][5cm]{%
\begin{center}
\begin{adjustbox}{max width=0.95\textwidth, max height=#1}
\framebox[\textwidth][c]{%
\rule{0pt}{#1}%
\textcolor{midgray}{[#2]}%
}
\end{adjustbox}
\end{center}
}

% ---- Notation ----
% Shared notation macros for AI-Based Detection of Hedge Fund Fraud
% Include this file in all Beamer slide decks via % Shared notation macros for AI-Based Detection of Hedge Fund Fraud
% Include this file in all Beamer slide decks via % Shared notation macros for AI-Based Detection of Hedge Fund Fraud
% Include this file in all Beamer slide decks via \input{notation}

% Performance metrics
\newcommand{\auc}{\ensuremath{\mathrm{AUC}}}
\newcommand{\fone}{\ensuremath{F_1}}
\newcommand{\shap}{\text{SHAP}}
\newcommand{\lime}{\text{LIME}}

% Autocorrelation
\newcommand{\rhoone}{\ensuremath{\rho_1}}

% Financial abbreviations
\newcommand{\nav}{\text{NAV}}
\newcommand{\aum}{\text{AUM}}

% Regulatory
\newcommand{\euaiact}{EU AI Act}
\newcommand{\aifmd}{AIFMD}
\newcommand{\sec}{SEC}
\newcommand{\edgar}{EDGAR}

% Key numbers from the paper (for consistency)
\newcommand{\aumdollar}{\$4.5\text{ trillion}}
\newcommand{\aucdeg}{10.6\%}
\newcommand{\numop}{10}
\newcommand{\numfraudcases}{50\text{--}100}
\newcommand{\systematicpapers}{105}


% Performance metrics
\newcommand{\auc}{\ensuremath{\mathrm{AUC}}}
\newcommand{\fone}{\ensuremath{F_1}}
\newcommand{\shap}{\text{SHAP}}
\newcommand{\lime}{\text{LIME}}

% Autocorrelation
\newcommand{\rhoone}{\ensuremath{\rho_1}}

% Financial abbreviations
\newcommand{\nav}{\text{NAV}}
\newcommand{\aum}{\text{AUM}}

% Regulatory
\newcommand{\euaiact}{EU AI Act}
\newcommand{\aifmd}{AIFMD}
\newcommand{\sec}{SEC}
\newcommand{\edgar}{EDGAR}

% Key numbers from the paper (for consistency)
\newcommand{\aumdollar}{\$4.5\text{ trillion}}
\newcommand{\aucdeg}{10.6\%}
\newcommand{\numop}{10}
\newcommand{\numfraudcases}{50\text{--}100}
\newcommand{\systematicpapers}{105}


% Performance metrics
\newcommand{\auc}{\ensuremath{\mathrm{AUC}}}
\newcommand{\fone}{\ensuremath{F_1}}
\newcommand{\shap}{\text{SHAP}}
\newcommand{\lime}{\text{LIME}}

% Autocorrelation
\newcommand{\rhoone}{\ensuremath{\rho_1}}

% Financial abbreviations
\newcommand{\nav}{\text{NAV}}
\newcommand{\aum}{\text{AUM}}

% Regulatory
\newcommand{\euaiact}{EU AI Act}
\newcommand{\aifmd}{AIFMD}
\newcommand{\sec}{SEC}
\newcommand{\edgar}{EDGAR}

% Key numbers from the paper (for consistency)
\newcommand{\aumdollar}{\$4.5\text{ trillion}}
\newcommand{\aucdeg}{10.6\%}
\newcommand{\numop}{10}
\newcommand{\numfraudcases}{50\text{--}100}
\newcommand{\systematicpapers}{105}


% ---- Title metadata ----
\title{AI-Based Detection of Hedge Fund Fraud}
\subtitle{Section 1 -- Introduction}
\author{Joerg Osterrieder}
\institute{Zurich University of Applied Sciences (ZHAW)}
\date{2025}

% ============================================================
\begin{document}

% ----------------------------------------------------------
% SLIDE 1 -- Title
% ----------------------------------------------------------
\begin{frame}
\titlepage
\end{frame}

% ----------------------------------------------------------
% SLIDE 2 -- Outline
% ----------------------------------------------------------
\begin{frame}{Outline}
\begin{enumerate}\compactlist
\item The Scale of Hedge Fund Fraud
\item Major Fraud Cases
\item Why Hedge Funds Are Uniquely Vulnerable
\item Limitations of Traditional Detection
\item Cognitive Limitations of Human Analysts
\item Traditional Statistical Methods and Their Limits
\item Four AI/ML Advantages
\item Survey Scope and Prior Work
\item Three Principal Contributions (C1--C3)
\item What This Survey Does NOT Do
\item Paper Organization
\item Summary and Key Takeaways
\end{enumerate}
\end{frame}

% ----------------------------------------------------------
% SLIDE 3 -- Scale of Hedge Fund Fraud
% ----------------------------------------------------------
\begin{frame}{The Scale of Hedge Fund Fraud}
\begin{itemize}\compactlist
\item Global hedge fund industry: \textbf{\aumdollar{} AUM} as of 2025
\item Substantial growth from roughly \$2 trillion at the onset of the 2008 crisis
\item Strategies range from quantitative stat-arb to activist equity and illiquid credit
\item Unlike mutual funds, hedge funds benefit from \textbf{broad regulatory exemptions}:
  \begin{itemize}\compactlist
  \item Limited disclosure
  \item Voluntary performance reporting
  \item Minimal portfolio-level transparency
  \end{itemize}
\item Strategic flexibility attracts institutional capital -- but simultaneously \textbf{enables fraud to persist undetected for years or decades}
\end{itemize}
\bottomnote{Source: Stulz (2007); paper Section 1.1}
\end{frame}

% ----------------------------------------------------------
% SLIDE 4 -- Major Fraud Cases
% ----------------------------------------------------------
\begin{frame}{Major Fraud Cases}
\begin{columns}[T]
\column{0.33\textwidth}
\begin{block}{Madoff (2008)}
\begin{itemize}\compactlist
\item \textbf{\$65 billion} in stated account value
\item Largest financial fraud in history
\item Ponzi scheme sustained for $\sim$20 years
\item Only 7 losing months over 14 years
\end{itemize}
\end{block}

\column{0.33\textwidth}
\begin{block}{Bayou Group (2003--05)}
\begin{itemize}\compactlist
\item \textbf{\$450 million} in concealed losses
\item Fabricated financial statements
\item Created a sham auditing firm
\end{itemize}
\end{block}

\column{0.33\textwidth}
\begin{block}{Archegos (2021)}
\begin{itemize}\compactlist
\item \textbf{\$10+ billion} counterparty losses
\item Prime broker failures
\item Concentrated position reporting gaps
\end{itemize}
\end{block}
\end{columns}

\vspace{4mm}
\begin{itemize}\compactlist
\item The SEC brings \textbf{dozens of enforcement actions} each year
\item Violations span: return misrepresentation, asset misappropriation, insider trading, valuation manipulation
\end{itemize}
\bottomnote{Source: Markopolos (2010); Gregoriou (2009); paper Section 1.1}
\end{frame}

% ----------------------------------------------------------
% SLIDE 5 -- Why Hedge Funds Are Uniquely Vulnerable
% ----------------------------------------------------------
\begin{frame}{Why Hedge Funds Are Uniquely Vulnerable}
\begin{enumerate}\compactlist
\item \textbf{Illiquid / hard-to-value assets}
  \begin{itemize}\compactlist
  \item Distressed debt, private equity co-investments, bespoke derivatives
  \item Independent pricing difficult or impossible $\Rightarrow$ \nav{} inflation, return smoothing
  \end{itemize}
\item \textbf{Voluntary reporting to databases}
  \begin{itemize}\compactlist
  \item HFR, Lipper TASS, Morningstar -- all voluntary
  \item Survivorship, backfill, and self-selection biases
  \end{itemize}
\item \textbf{Lock-up periods and redemption gates}
  \begin{itemize}\compactlist
  \item Restrict investor liquidity, delay fraud discovery
  \end{itemize}
\item \textbf{Concentrated authority}
  \begin{itemize}\compactlist
  \item Limited partnership structures: small group of general partners
  \item Minimal independent oversight
  \end{itemize}
\end{enumerate}
\vspace{2mm}
\textcolor{mlred}{$\Rightarrow$ Agency problem of unusual severity (Stulz, 2007)}
\bottomnote{Source: Getmansky et al.\ (2004); Fung \& Hsieh (2009); paper Section 1.1}
\end{frame}

% ----------------------------------------------------------
% SLIDE 6 -- Limitations of Traditional Detection
% ----------------------------------------------------------
\begin{frame}{Limitations of Traditional Detection}
\begin{columns}[T]
\column{0.55\textwidth}
\textbf{Regulatory Capacity Mismatch}
\begin{itemize}\compactlist
\item SEC: $\sim$\textbf{4,600 staff} overseeing thousands of advisers, broker-dealers, fund complexes
\item Division of Examinations inspects only a \textit{fraction} of possible funds per year
\item A single examiner can assess at most \textbf{a handful of funds per quarter}
\item Most hedge funds receive scrutiny only \textit{infrequently}
\item Long windows for fraudulent schemes to operate undiscovered
\end{itemize}

\column{0.42\textwidth}
\textbf{The Markopolos Case}
\begin{itemize}\compactlist
\item Harry Markopolos submitted detailed analyses to the SEC \textbf{starting in 2000}
\item Argued Madoff's returns were statistically implausible
\item \textcolor{mlred}{SEC failed to act for nearly a decade}
\item Reflects institutional shortcomings \textit{and} the difficulty of distinguishing skill from fabrication
\end{itemize}
\end{columns}
\bottomnote{Source: Markopolos (2010); paper Section 1.2}
\end{frame}

% ----------------------------------------------------------
% SLIDE 7 -- Cognitive Limitations
% ----------------------------------------------------------
\begin{frame}{Cognitive Limitations of Human Analysts}
\begin{itemize}\compactlist
\item Even experienced auditors face \textbf{fundamental cognitive constraints}
\item Well-documented biases that impair fraud signal identification:
\end{itemize}
\vspace{2mm}
\begin{columns}[T]
\column{0.32\textwidth}
\begin{block}{Hindsight Bias}
After fraud is revealed, signals seem ``obvious'' -- but they were not ex ante
\end{block}

\column{0.32\textwidth}
\begin{block}{Confirmation Bias}
Analysts seek evidence confirming initial assessment, ignoring contradictory signals
\end{block}

\column{0.32\textwidth}
\begin{block}{Anchoring Effects}
Prior expectations anchor judgment -- deviations from established templates are under-weighted
\end{block}
\end{columns}
\vspace{3mm}
\begin{itemize}\compactlist
\item These biases compound the difficulty of evaluating \textbf{complex, opaque strategies}
\item Throughput bottleneck $+$ cognitive limitations $\Rightarrow$ systematic detection gaps
\end{itemize}
\bottomnote{Source: Paper Section 1.2}
\end{frame}

% ----------------------------------------------------------
% SLIDE 8 -- Traditional Statistical Methods
% ----------------------------------------------------------
\begin{frame}{Traditional Statistical Methods and Their Limits}
\begin{columns}[T]
\column{0.48\textwidth}
\textbf{Established Methods}
\begin{itemize}\compactlist
\item \textbf{Benford's law}: tests leading-digit distribution of returns
  \begin{itemize}\compactlist
  \item Can identify data fabrication
  \item Easily defeated by knowledgeable fraudster
  \end{itemize}
\item \textbf{Serial correlation}: detects suspicious smoothness in return series (Bollen \& Pool 2012; Getmansky et al.\ 2004)
  \begin{itemize}\compactlist
  \item Captures only one dimension of fraud
  \end{itemize}
\item \textbf{Forensic ratio / outlier detection}: flags individual anomalies
\end{itemize}

\column{0.48\textwidth}
\textbf{Fundamental Limitations}
\begin{itemize}\compactlist
\item Each method detects a \textit{single} signature of a \textit{single} fraud type
\item Cannot capture \textbf{complex, multi-dimensional patterns}
\item Logistic regression on Form ADV (Dimmock \& Gerken 2012):
  \begin{itemize}\compactlist
  \item \auc{} $\approx 0.65$--$0.70$
  \item Interpretable but limited feature space
  \item Does not scale to modern data volumes
  \end{itemize}
\item High false positive rates when deployed independently
\end{itemize}
\end{columns}
\bottomnote{Source: Nigrini (2012); Dimmock \& Gerken (2012); paper Section 1.2}
\end{frame}

% ----------------------------------------------------------
% SLIDE 9 -- Four AI/ML Advantages
% ----------------------------------------------------------
\begin{frame}{Four AI/ML Advantages for Fraud Detection}
\begin{enumerate}\compactlist
\item \textcolor{mlblue}{\textbf{Scalability}}
  \begin{itemize}\compactlist
  \item Process thousands of return series, filings, and alternative data \textit{simultaneously}
  \item Surveillance at a scale human analysts cannot achieve
  \end{itemize}
\item \textcolor{mlblue}{\textbf{Pattern Recognition}}
  \begin{itemize}\compactlist
  \item Detect subtle, nonlinear, multi-dimensional anomalies
  \item E.g., random forest on dozens of return features: suspicious \textit{combinations} no single test flags
  \end{itemize}
\item \textcolor{mlblue}{\textbf{Real-Time Monitoring}}
  \begin{itemize}\compactlist
  \item Once deployed, models evaluate incoming data continuously
  \item Early warning systems alert before losses compound
  \end{itemize}
\item \textcolor{mlblue}{\textbf{Multi-Modal Data Integration}}
  \begin{itemize}\compactlist
  \item Fuse structured (returns, ratios), unstructured (news, sentiment), and relational data (networks)
  \item Richer, more holistic picture of fund behavior
  \end{itemize}
\end{enumerate}
\bottomnote{Source: Paper Section 1.2}
\end{frame}

% ----------------------------------------------------------
% SLIDE 10 -- Survey Scope
% ----------------------------------------------------------
\begin{frame}{Survey Scope}
\begin{itemize}\compactlist
\item \textbf{First systematic, qualitative survey} of AI-based approaches to hedge fund fraud detection
\item No existing survey addresses AI fraud detection with a \textit{specific} focus on:
  \begin{itemize}\compactlist
  \item The hedge fund context
  \item Its unique data challenges
  \item Its distinctive regulatory environment
  \end{itemize}
\item Current literature is \textbf{fragmented}:
  \begin{itemize}\compactlist
  \item Spans computer science, finance, accounting, law
  \item Divergent datasets, evaluation metrics, fraud definitions
  \item Rarely addresses adversarial dynamics
  \end{itemize}
\item This survey \textbf{synthesizes} the scattered literature into a coherent analytical framework
\end{itemize}
\bottomnote{Source: Paper Section 1.3}
\end{frame}

% ----------------------------------------------------------
% SLIDE 11 -- Prior Survey Comparison
% ----------------------------------------------------------
\begin{frame}{Prior Survey Comparison}

\chartplaceholder[4.5cm]{Chart: 01\_survey\_comparison -- Comparison table of prior surveys vs.\ this survey across six dimensions (Hedge Fund Focus, AI/ML Methods, Fraud Taxonomy, Adversarial Robustness, Regulatory Readiness, Research Agenda)}

\vspace{2mm}
\begin{itemize}\compactlist
\item Prior surveys: Ngai et al.\ (2011), Abdallah et al.\ (2016), West \& Bhattacharya (2016), Pourhabibi et al.\ (2020), Bao et al.\ (2020), Hilal et al.\ (2022), Ahmed et al.\ (2024)
\item \textcolor{mlred}{None} substantively covers all six dimensions -- \textbf{this survey} does
\end{itemize}
\bottomnote{Source: Paper Table 1; Section 1.3}
\end{frame}

% ----------------------------------------------------------
% SLIDE 12 -- Contribution C1
% ----------------------------------------------------------
\begin{frame}{Contribution C1: Detection Pipeline Taxonomy}
\begin{block}{C1: Unified Five-Stage Detection Pipeline}
A framework spanning \textbf{data ingestion, feature engineering, model selection, explainability, and deployment} that systematically maps hedge fund fraud types to appropriate AI detection methods.
\end{block}

\vspace{3mm}
\begin{itemize}\compactlist
\item Provides researchers and practitioners with a \textbf{structured lens}:
  \begin{itemize}\compactlist
  \item Which methods apply to which fraud scenarios?
  \item Where do methodological gaps remain?
  \end{itemize}
\item \textbf{No existing survey} provides this hedge-fund-specific mapping
\item Engineering blueprint for operational surveillance systems
\end{itemize}
\bottomnote{Source: Paper Section 1.3 -- Contribution C1}
\end{frame}

% ----------------------------------------------------------
% SLIDE 13 -- Contribution C2
% ----------------------------------------------------------
\begin{frame}{Contribution C2: Adversarial and Regulatory Readiness}
\begin{block}{C2: Adversarial and Regulatory Readiness Assessment}
Systematic evaluation of how \textbf{robust} current AI methods are to adversarial manipulation by sophisticated hedge fund managers, and whether they satisfy \textbf{emerging regulatory requirements}.
\end{block}

\vspace{3mm}
\begin{itemize}\compactlist
\item \textbf{Adversarial lens}: hedge fund managers are sophisticated actors who adapt behavior to evade detection
\item \textbf{Regulatory lens}:
  \begin{itemize}\compactlist
  \item \euaiact{} (Regulation 2024/1689) -- classifies fraud detection AI as \textit{high-risk}
  \item SEC guidance on predictive analytics
  \end{itemize}
\item Bridges the gap between technical ML literature and practical demands of regulators / compliance
\item \textcolor{mlred}{No prior survey} evaluates AI fraud detection through this dual lens
\end{itemize}
\bottomnote{Source: Paper Section 1.3 -- Contribution C2}
\end{frame}

% ----------------------------------------------------------
% SLIDE 14 -- Contribution C3
% ----------------------------------------------------------
\begin{frame}{Contribution C3: Actionable Research Roadmap}
\begin{block}{C3: Ten Open Research Problems}
Each problem is differentiated by the specific characteristics of the hedge fund context, with suggested methodological approaches, evaluation protocols, and feasibility considerations.
\end{block}

\vspace{3mm}
\begin{itemize}\compactlist
\item \textbf{\numop{} concrete open problems} identified
\item For each problem:
  \begin{itemize}\compactlist
  \item Suggested methodological approaches
  \item Evaluation protocols
  \item Feasibility considerations
  \end{itemize}
\item Designed to guide:
  \begin{itemize}\compactlist
  \item Academic researchers seeking \textbf{impactful problems}
  \item Industry practitioners seeking \textbf{evidence-based solutions}
  \end{itemize}
\end{itemize}
\bottomnote{Source: Paper Section 1.3 -- Contribution C3}
\end{frame}

% ----------------------------------------------------------
% SLIDE 15 -- What This Survey Does NOT Do
% ----------------------------------------------------------
\begin{frame}{What This Survey Does NOT Do}
\begin{itemize}\compactlist
\item \textbf{Not a quantitative meta-analysis} of detection performance across studies
\item Why not?
  \begin{itemize}\compactlist
  \item Heterogeneity of datasets, fraud definitions, evaluation protocols, reporting standards
  \item Meaningful statistical aggregation is \textit{precluded}
  \end{itemize}
\item Instead: a \textbf{qualitative synthesis} approach
  \begin{itemize}\compactlist
  \item Critical analysis of methodological strengths, limitations, contextual applicability
  \end{itemize}
\item Appropriate given the current state of the field:
  \begin{itemize}\compactlist
  \item Standardization of benchmarks and evaluation procedures remains an \textbf{open challenge}
  \item Addressed explicitly in the research agenda (Section 6)
  \end{itemize}
\end{itemize}
\bottomnote{Source: Paper Section 1.3}
\end{frame}

% ----------------------------------------------------------
% SLIDE 16 -- Paper Organization
% ----------------------------------------------------------
\begin{frame}{Paper Organization}
\begin{enumerate}\compactlist
\item \textbf{Section 2 -- Background}: Fraud taxonomy, data ecosystem, regulatory context
\item \textbf{Section 3 -- Detection Pipeline} (C1): Five-stage framework from raw data to actionable assessments
\item \textbf{Section 4 -- Literature Review}: AI/ML methods organized by method family, mapped onto pipeline
\item \textbf{Section 5 -- Adversarial \& Regulatory} (C2): Robustness to manipulation, EU AI Act, ethical considerations
\item \textbf{Section 6 -- Research Agenda} (C3): Ten open problems with approaches and evaluation criteria
\item \textbf{Section 7 -- Conclusion}: Synthesis of findings and implications
\end{enumerate}
\vspace{3mm}
\textcolor{mlblue}{Each section builds on the previous -- the pipeline taxonomy (C1) provides the organizational backbone for the literature review and the research agenda.}
\bottomnote{Source: Paper Section 1.4}
\end{frame}

% ----------------------------------------------------------
% SLIDE 17 -- AUM Growth Timeline
% ----------------------------------------------------------
\begin{frame}{Hedge Fund Industry: AUM Growth Timeline}

\chartplaceholder[5cm]{Chart: 02\_hedge\_fund\_growth -- Timeline of hedge fund AUM growth from $\sim$\$2T (2008 crisis) to \$4.5T (2025), annotated with major fraud cases (Bayou 2005, Madoff 2008, SAC 2013, Archegos 2021)}

\vspace{2mm}
\begin{itemize}\compactlist
\item Industry \aum{} has more than doubled since 2008
\item Major fraud revelations clustered around crises and regulatory inflection points
\end{itemize}
\bottomnote{Source: Paper Section 1.1}
\end{frame}

% ----------------------------------------------------------
% SLIDE 18 -- Summary and Key Takeaways
% ----------------------------------------------------------
\begin{frame}{Summary and Key Takeaways}
\begin{enumerate}\compactlist
\item Hedge fund fraud is a \textbf{multi-billion-dollar problem} enabled by structural opacity, voluntary reporting, and concentrated authority
\item Traditional detection is limited by \textbf{regulatory capacity constraints}, cognitive biases, and univariate statistical methods
\item AI/ML offers four fundamental advantages: \textbf{scalability, pattern recognition, real-time monitoring, multi-modal integration}
\item This survey makes \textbf{three contributions}:
  \begin{itemize}\compactlist
  \item[\textbf{C1}] Detection pipeline taxonomy (5 stages)
  \item[\textbf{C2}] Adversarial and regulatory readiness assessment
  \item[\textbf{C3}] Research roadmap (10 open problems)
  \end{itemize}
\item The field is \textbf{fragmented and early-stage} -- no prior survey addresses AI fraud detection specifically for hedge funds
\item Qualitative synthesis approach is appropriate given the current lack of standardized benchmarks
\end{enumerate}
\bottomnote{Source: Paper Section 1}
\end{frame}

\end{document}
